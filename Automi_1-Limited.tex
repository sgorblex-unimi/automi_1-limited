\documentclass[a4paper,twoside]{thesis}
% \documentclass[a4paper]{thesis}
% \documentclass[a4paper,compact]{thesis}	% compact is used for drafts

\usepackage{Automi_1-Limited}

\begin{document}
\title{Automi 1-Limited}
\author{Alessandro Clerici Lorenzini}
\matr{941784}
\logo{img/unimi_logo}
\university{Università degli Studi di Milano}
\dept{Dipartimento di Informatica}
\degr{Corso di Laurea in Informatica}
\superv{Prof. Giovanni Pighizzini}
\cosuperv{Dr. Luca Prigioniero}
\date{Anno Accademico 2021/2022}

\pagestyle{plain}
\maketitle
\thispagestyle{empty}
\cleardoublepage
\pagenumbering{roman}
\tableofcontents
\preface
La teoria dei linguaggi formali è un campo dell'informatica teorica che studia i linguaggi, cioè insiemi di stringhe, e i modi di rappresentarli finitamente. In quest'ambito si studiano diversi modelli di calcolo, in grado di riconoscere varie classi di linguaggi. Un classico oggetto di studio di questo campo sono i modelli computazionali con risorse limitate, di cui si studia la potenza riconoscitiva e il rapporto con altri modelli.

Gli automi limited sono stati introdotti da Hibbard nel 1967 \cite{Hibbard:67:CFdet} come strumento per costruire una gerarchia che caratterizzasse il determinismo all'interno della classe dei linguaggi liberi dal contesto.
Un automa $d$-limited è una macchina di Turing nondeterministica in cui lo spazio di lavoro è limitato alle celle che inizialmente contengono l'input e la scrittura di una cella è possibile solo durante le sue prime $d$ visite. Sebbene Hibbard abbia dimostrato che gli automi $d$-limited nondeterministici caratterizzano i linguaggi liberi dal contesto quando $d>1$, gli automi $0$-limited (cioè con $d=0$) corrispondono alla definizione di automi a stati finiti nondeterministici \eng{two-way}, riconoscitori di linguaggi regolari. Wagner e Wechsung \cite{Wagner:86:compCompl} hanno dimostrato che permettendo la scrittura alla prima visita di una cella la classe riconosciuta è la stessa, ossia gli automi $1$-limited caratterizzano i linguaggi regolari.

L'individuazione di diversi riconoscitori equivalenti per la stessa classe di linguaggi ha portato la ricerca a chiedersi quale sia il costo delle diverse rappresentazioni. La complessità descrizionale, che ha avuto origine con l'articolo fondatore di Meyer e Fischer \cite{Meyer:71:ecodescription}, studia quanto una descrizione di un linguaggio formale, ad esempio una grammatica o un automa, può essere succinta, e le relazioni tra le dimensioni di diverse descrizioni equivalenti. Per esempio, la costruzione per sottoinsiemi di Rabin e Scott \cite{Rabin:59:NFA} ha dimostrato che pagando un costo esponenziale nel numero di stati è possibile rimuovere il nondeterminismo da un automa a stati finiti. Meyer e Fischer hanno dimostrato che esistono linguaggi per cui tale costo è inevitabile \cite{Meyer:71:ecodescription}.

Nel 2014 Pighizzini e Pisoni hanno ripreso lo studio degli automi limited trattandoli dal punto di vista della complessità descrizionale, ottenendo in \cite{Pighizzini:14:limitedRE} l'upper bound del numero di stati necessari a un automa a stati finiti per simulare un automa $1$-limited. Tale simulazione, a partire da un automa $1$-limited di $n$ stati, produce un automa a stati finiti nondeterministico con un numero di stati esponenziale in $n$, o uno deterministico con un numero di stati doppiamente esponenziale. Lo stesso articolo ha dimostrato che per qualunque $n$ esistono casi in cui tale costo non può essere evitato. In altre parole, gli automi $1$-limited possono fornire descrizioni molto più compatte rispetto ai riconoscitori standard per la classe dei regolari. Negli anni successivi sono state sviluppate diverse tecniche applicabili agli automi $1$-limited che hanno portato allo studio di altri linguaggi che dimostrano i bound che li riguardano (ad esempio \cite{Pighizzini:22:limitedwitness}). Al contempo sono state presentate e studiate altre varianti di automi limited, con risultati interessanti specialmente in relazione alla complessità descrizionale.

In questo elaborato riassumiamo innanzitutto (capitolo \ref{cha:prel}) le conoscenze che consistono in prerequisiti per la comprensione del cuore della tesi, toccando le principali definizioni della teoria dei linguaggi formali. Il capitolo \ref{cha:a1l} introduce il modello degli automi $d$-limited e degli $1$-limited in particolare, quindi ne studia la potenza computazionale e descrizionale, presentando le principali simulazioni da parte degli altri riconoscitori di linguaggi regolari e dimostrando l'ottimalità dei rispettivi costi. Il capitolo \ref{cha:wit} approfondisce alcune tecniche applicabili agli $1$-limited per riconoscere linguaggi con particolari caratteristiche. Per i linguaggi a blocchi, cioè composti da concatenazioni di stringhe di uguale lunghezza legate da una particolare relazione, viene costruito un riconoscitore che implementa un algoritmo basato sul confronto simbolo a simbolo, mentre per i lower bound viene usata una dimostrazione basata su distinguibilità (tecnica nota in letteratura). Per i linguaggi unari, cioè costruiti su un alfabeto di un solo simbolo, viene sviluppata una tecnica basata sulla binary carry sequence (la successione dei massimi esponenti di $2$ che dividono ogni naturale) che permette a un automa $1$-limited di verificare se la lunghezza di una stringa è multipla di un naturale $n$ usando un numero logaritmico di stati in $n$. Infine viene studiato un adattamento della stessa tecnica utile a simulare alcuni automi a stati finiti che hanno una transizione che riporta allo stato iniziale. Il capitolo \ref{cha:prob} conclude con una panoramica dei risultati sul modello, evidenziandone i problemi ancora aperti e le possibili ricerche future, e discute brevemente gli altri risultati che riguardano gli automi limited e le loro varianti.

\pagenumbering{arabic}
\pagestyle{headings}

\chapter{Nozioni preliminari}
La teoria dei linguaggi formali è un campo fondamentale dell'informatica teorica e studia i linguaggi, cioè insiemi di parole, e i loro generatori o riconoscitori. Questo campo, sebbene a prima vista appaia strettamente legato alla linguistica, ha implicazioni enormi nell'informatica e nella matematica, per esempio nei campi della computabilità, della programmazione, della crittografia e della logica. In questo capitolo ricordiamo al lettore le nozioni fondamentali della teoria dei linguaggi, delle grammatiche e dei principali riconoscitori, di cui studieremo il rapporto con gli automi \eng{$1$-limited}.


\subsection*{Convenzioni di notazione}
Dato un insieme $S$, indicheremo con $\card{S}$ la sua cardinalità e con $\subsets{S}$ l'insieme dei suoi sottoinsiemi.
% TODO: aggiungere qui eventuali altre notazioni
La restante notazione è descritta nelle sezioni seguenti.


\section{Linguaggi}
In questa sezione introduciamo le definizioni fondamentali della teoria dei linguaggi, inclusi alfabeti, parole e linguaggi stessi. La maggior parte dei concetti è espressa seguendo la notazione usata in \cite{Hopcroft:79:introLFA}, che invitiamo il lettore a consultare per approfondimenti.

\paragraph{Alfabeti} Un \emph{alfabeto} è un insieme finito e non vuoto arbitrario, i cui elementi sono detti \emph{simboli}. Indicheremo alfabeti mediante simboli come $\Sigma$, $\Gamma$ o esplicitando direttamente l'insieme ($\set{a,b}$). I simboli di un alfabeto sono spesso indicati con lettere minuscole o cifre numeriche, tuttavia può essere utile usare altri simboli quando la semantica ne viene semplificata. Chiamiamo unario un alfabeto di esattamente un elemento.

\paragraph{Parole} Una \emph{parola} (o \emph{stringa}) $w$ su un alfabeto $\Sigma$ è una sequenza finita di simboli appartenenti a $\Sigma$: $w:=x_1 x_2 \cdots x_n$ con $x_1,x_2,\dots,x_n\in\Sigma$. Le parole si indicano comunemente con lettere latine minuscole. La parola non contenente simboli è detta parola vuota e si indica con $\emptyword$. La lunghezza di una parola $w$ è il numero di simboli che la compongono e viene indicata con $\len w$. Si indica con $\Sigma^n$, con $n\in\N$, l'insieme di parole di lunghezza $n$ su $\Sigma$, con $\Sigma^0:=\set{\emptyword}$ per qualunque $\Sigma$. Spesso si usa semplicemente $\Sigma$ per indicare l'insieme $\Sigma^1$ delle parole di lunghezza $1$ su $\Sigma$. Si indica con $\Sigma\star$ l'insieme di tutte le parole sull'alfabeto $\Sigma$, cioè l'unione di tutte le sue potenze, e con $\Sigma^+$ l'insieme delle parole non vuote su $\Sigma$: $\Sigma^+ := \Sigma\star\setminus\set{\emptyword}$. Date due parole $v=x_1\cdots x_n$ e $w=y_1\cdots y_m$ si dice concatenazione di $v$ e $w$ la parola $vw:=x_1\cdots x_n y_1\cdots y_m$ composta dai simboli di $v$ seguiti da quelli di $w$. Si noti che $\len{vw}=\len v+\len w$.

\paragraph{Linguaggi} Un linguaggio $L$ su un alfabeto $\Sigma$ è un insieme di parole su $\Sigma$, ossia $L\subseteq\Sigma\star$ (le potenze di un alfabeto sono dunque linguaggi). Il simbolo $\emptyset$ indica il linguaggio vuoto, da non confondere con il linguaggio $\set{\emptyword}$.  A volte omettiamo le parentesi graffe per indicare un linguaggio singoletto ($a$ al posto di $\set{a}$). Chiamiamo unario un linguaggio su un alfabeto unario. Il prodotto di due linguaggi $L_1$ e $L_2$ è il linguaggio in cui ogni parola è la concatenazione di una parola di $L_1$ e una di $L_2$:
\begin{equation*}
	L_1\cdot L_2 := \set{xy\mid x\in L_1 \land y\in L_2}
\end{equation*}
Dato un linguaggio $L$, viene indicato con $L^0$ il linguaggio $\set{\emptyword}$, e con $L^n$ il prodotto di $L$ con se stesso $n$ volte. La chiusura (di Kleene) di un linguaggio $L$ è il linguaggio $L\star$ delle concatenazioni di un numero arbitrario di parole di $L$:
\begin{equation*}
	L\star := L^0\cup L^1\cup\dots=\bigcup_{k=0}^\infty L^k
\end{equation*}
Si indica inoltre con $L^+$ il linguaggio $\bigcup_{k=1}^\infty L^k$. Si noti che $L^+=L\star$ se e solo se $\emptyword\in L$.


\subsection*{Riconoscitori e generatori}
Esistono diversi modi di rappresentare un linguaggio $L$: se è finito è sufficiente elencarne le parole ($L:=\set{w_1,w_2,\dots,w_n}$), se è infinito e le sue parole possiedono una proprietà caratterizzante $P$, può essere descritto da essa ($L:=\set{w\mid P(w)}$). Tuttavia, non sempre è facile o rappresentativo usare una di queste rappresentazioni. I metodi generativo e riconoscitivo forniscono un ulteriore modo di descrivere i linguaggi.

% TODO: aggiungere un riferimento bibliografico che permetta al lettore di approfondire il concetto di sistema formale (e, volendo, calcolo logico)
\begin{description}
	\item[Generatori] un generatore per un linguaggio $L\subseteq\Sigma\star$ è un sistema formale che produce parole appartenenti a $L$, cioè un metodo per costruirle tramite regole.
	\item[Riconoscitori] un riconoscitore per un linguaggio $L\subseteq\Sigma\star$ è un algoritmo o una procedura che determina se una data parola $w\in\Sigma\star$ appartiene a $L$.
\end{description}

\noindent Dato un riconoscitore o generatore $M$, si indica con $\generated M$ il linguaggio riconosciuto o generato da $M$.



\section{Grammatiche}
Le \emph{grammatiche} sono il principale generatore studiato nella teoria dei linguaggi formali. Una grammatica su un alfabeto $\Sigma$ consiste in un insieme di regole che permettono di costruire le parole di un linguaggio trasformando in più passi un \emph{assioma} di partenza. Queste trasformazioni seguono delle regole, dette di produzione, nella forma $\alpha\to\beta$, che permettono di trasformare una parola $u\alpha v$ in $u\beta v$. Le parole derivate applicando le regole di produzione possono contenere simboli detti \emph{nonterminali} da un alfabeto ausiliario (a cui appartiene l'assioma), ma solo quelle composte esclusivamente da simboli \emph{terminali}, cioè appartenenti a $\Sigma$, fanno parte del linguaggio generato.


\subsection{La Classificazione di Chomsky}\label{subs:prel:chom}
La gerarchia Chomsky (\cite{Chomsky:56:hier}) è una classificazione delle grammatiche, basata sulla forma delle loro regole di produzione, da cui deriva una gerarchia di linguaggi fondamentale nella teoria dei linguaggi formali. La gerarchia si compone di quattro classi:
\begin{description}
	\item[Tipo 0] tutte le grammatiche sono di \emph{tipo 0};
	\item[Tipo 1] in una grammatica di tipo 1, ogni regola di produzione è nella forma $\alpha A\beta\to\alpha\gamma\beta$, dove $\gamma$ è non vuota e $A$ è un simbolo nonterminale. È ammessa la regola $S\to\emptyword$, se $S$ è l'assioma, solo se $S$ non compare nella parte destra di alcuna altra regola. I linguaggi che possono essere generati da grammatiche di tipo 1 sono detti \emph{dipendenti dal contesto} (\eng{context-sensitive});
	\item[Tipo 2] in una grammatica di tipo 2, ogni regola di produzione $\alpha\to\beta$ è tale che $\alpha$ è un simbolo nonterminale. Valgono le stesse restrizioni sull'assioma delle grammatiche di tipo 1. I linguaggi che possono essere generati da grammatiche di tipo 2 sono detti \emph{liberi dal contesto} (\eng{context-free});
	\item[Tipo 3] in una grammatica di tipo 3, ogni regola di produzione è in una delle forme $A\to\sigma B$, $A\to\sigma$, $A\to\emptyword$, dove $A$ e $B$ sono simboli nonterminali e $\sigma$ è un simbolo terminale. I linguaggi che possono essere generati da grammatiche di tipo 3 sono detti \emph{regolari}.
\end{description}
Esiste un'inclusione tra le classi di grammatiche di tipo più alto e quelle di tipo più basso, da cui deriva un'inclusione (propria) tra le rispettive classi di linguaggi. Come vedremo, queste classi corrispondono inoltre a classi di riconoscitori.



\section{Riconoscitori}
I riconoscitori sono modelli matematici che, data una parola in input, determinano se questa appartiene al linguaggio (accettazione) o non vi appartiene (rifiuto). Formalmente, data una macchina $M$ che lavora su un alfabeto $\Sigma$:
\begin{equation*}
	\generated M := \set{w\in\Sigma\star\mid M\text{ accetta }w}
\end{equation*}
Per questo motivo di ogni modello si definisce il concetto di accettazione.
Per approfondimenti, dimostrazioni e altri riconoscitori suggeriamo la lettura di \cite{Hopcroft:79:introLFA} e \cite{Shallit:09:secondLFA}.


\subsection{Macchine di Turing}
Una macchina di Turing si compone di un nastro, una testina e un controllo finito. Il nastro è infinito a destra ed è diviso in celle, le prime celle contenenti l'input (un simbolo per cella), le seguenti il simbolo vuoto. La testina punta a una cella del nastro (visita) ed è in grado di leggere e scrivere su di essa, scegliendo da un predeterminato alfabeto. Il controllo finito consiste in un insieme finito di stati di cui uno corrente. L'evoluzione della computazione di una macchina di Turing procede per istanti successivi, secondo una legge detta \emph{funzione di transizione}. Precisamente, la macchina, in funzione dello stato corrente e del simbolo letto dalla testina, sovrascrive il simbolo nella cella con un altro, cambia stato, e si muove in una direzione (sinistra o destra) in una cella adiacente. Lo stato corrente e i simboli sul nastro (più implicitamente la posizione della testina) consistono nell'unica memoria della macchina.

\begin{defin}[macchina di Turing]
	Una \emph{macchina di Turing} (TM) è una settupla $M:=\tuple{Q,\Sigma,\Gamma,\blank,\delta,q_0,F}$, dove:
	\begin{itemize}
		\item $Q$ è un insieme finito e non vuoto di stati;
		\item $\Sigma$ è l'alfabeto di input, cioè dei simboli che si possono trovare sul nastro all'inizio della computazione (insieme a $\blank$);
		\item $\Gamma\supseteq\Sigma\cup\set{\blank}$ è un alfabeto di simboli per il nastro;
		\item $\blank\in\Gamma$ è il simbolo vuoto (\emph{blank}), che ricorre in ogni cella del nastro a destra dell'input;
		\item $\delta:Q\times\Gamma\to Q\times\Gamma\times\set{\Left,\Right}$ è una funzione parziale detta di transizione. Se a un dato passo la macchina è nello stato $p$, la cella puntata dalla testina contiene $\sigma$, e $\delta(p,\sigma)=(q,\gamma,D)$, allora:
		      \begin{itemize}
			      \item $q$ è il prossimo stato;
			      \item $\gamma$ è il simbolo che verrà scritto in sostituzione a $\sigma$ nella cella corrente, allo spostamento della testina;
			      \item $D$ è la direzione in cui si muoverà la testina, $\Left$ se a sinistra e $\Right$ se a destra.
		      \end{itemize}
		\item $q_0\in Q$ è lo stato iniziale;
		\item $F\subseteq Q$ è l'insieme degli stati finali (o accettanti).
	\end{itemize}
	Una macchina di Turing accetta una parola $w\in\Sigma\star$ se e solo se la sua computazione, a partire dallo stato $q_0$ e con la testina sul primo simbolo di $w$, termina in uno stato finale.
\end{defin}
\noindent Sia l'arresto in uno stato non finale sia il non arresto sono considerati rifiuti.

\begin{figure}
	\centering
	\begin{tikzpicture}[cell/.style={minimum height=1.5em,minimum width=1.5em,outer sep=0pt,rectangle,draw,node distance=0pt}]
	\node[cell] (first) {$\sigma_0$};
	\node[cell] (pointed) [right=of first] {$\sigma_1$};
	\node[cell] (third) [right=of pointed] {$\sigma_2$};
	\node[cell, minimum width=2.5em] (worddots) [right=of third] {$\dots$};
	\node[cell] (last) [right=of worddots] {$\sigma_n$};
	\node[cell] (b1) [right=of last] {$b$};
	\node[cell] (b2) [right=of b1] {$b$};
	\node[node distance=0pt] (dots) [right=0.4 cm of b2] {$\dots$};
	\node[cell] (control) [above=0.75cm of pointed,thick] {$q$};
	\draw[-latex] (control) -- (pointed);
	\draw (b2.north east) -- ++(1.5cm,0) (b2.south east) -- ++ (1.5cm,0);
\end{tikzpicture}

	\caption{Rappresentazione di una macchina di Turing di esempio in un dato istante.}
\end{figure}

\subsubsection{Nondeterminismo}
È fondamentale citare il modello nondeterministico delle macchine di Turing e, in generale, delle macchine riconoscitrici. Una macchina di Turing nondeterministica (NTM) ha per funzione di transizione una relazione del tipo $\delta:Q\times\Gamma\to \subsets{Q\times\Gamma\times\set{\Left,\Right}}$. Un modello di questo tipo descrive multiple possibilità per un passo dell'evoluzione della macchina, ciascuna descritta da uno degli elementi di un'immagine della funzione. Una parola in input è accettata se esiste una computazione, tra quelle coerenti con la funzione di transizione, che termina in uno stato accettante.
In generale, una macchina riconoscitrice si dice nondeterministica quando la sua funzione di transizione fornisce più possibilità per un passo evolutivo, e l'accettazione corrisponde all'esistenza di una computazione che termini accettando.

Poiché per ogni TM nondeterministica è possibile costruire una TM deterministica che riconosce lo stesso linguaggio e viceversa, i due modelli riconoscono la stessa classe di linguaggi. La classe dei linguaggi accettati da macchine di Turing è quella dei linguaggi ricorsivamente enumerabili, che equivale alla classe dei linguaggi generabili da grammatiche di tipo 0 della classificazione di Chomsky (paragrafo \ref{subs:prel:chom}).

% TODO: aggiungere estensione con limitazione da una funzione lineare nell'input? Serve fonte
\subsubsection{Automi limitati linearmente}
Gli \emph{automi linearmente limitati} (LBA) sono macchine di Turing in cui la lunghezza del nastro, invece che essere infinita, è limitata dalla lunghezza dell'input. La classe dei linguaggi accettati da automi linearmente limitati è quella dei linguaggi dipendenti da contesto.


\subsection{Automi a pila}\label{subs:prel:PDA}
Un \emph{automa a pila} (PDA) è una macchina nondeterministica composta da un controllo finito, un nastro in sola lettura e una pila. Una pila è una struttura dati che permettere di leggere e scrivere in maniera LIFO (\eng{Last In, First Out}). A ogni passo l'automa può leggere un simbolo dell'input, procedendo da sinistra verso destra, oppure non leggere nulla ($\emptyword$-mossa). In funzione del simbolo letto sul nastro e di quello sulla cima della pila, l'automa cambia stato e sostituisce il simbolo in cima alla pila con una parola. Un PDA accetta una parola $w\in\Sigma\star$ se e solo se esiste una computazione che, a partire dallo stato iniziale, con la pila contenente il simbolo iniziale e con la lettura del primo simbolo di $w$, termina la lettura dell'input in uno stato finale. Si può definire in alternativa l'accettazione per pila vuota, cioè terminando la lettura dell'input con la pila non contenente simboli, che risulta equivalente nella potenza riconoscitiva.

Come le macchine di Turing, gli automi a pila hanno una controparte deterministica: gli \emph{automi a pila deterministici} (DPDA). Tuttavia, contrariamente alle TM, essa non equivale, nel potere riconoscitivo, alla versione nondeterministica. Si può dimostrare che la classe di linguaggi riconosciuta da PDA coincide con la classe dei linguaggi liberi da contesto. I DPDA riconoscono una sottoclasse propria dei linguaggi liberi da contesto, i cui membri sono detti semplicemente linguaggi liberi da contesto deterministici. Rispetto alla classificazione di Chomsky questa classe è un sottoinsieme proprio della classe dei context-free e un soprainsieme di quella dei regolari. Una gerarchia più ampia di linguaggi context-free derivante dal determinismo è stata introdotta da Hibbard in \cite{Hibbard:67:CFdet} tramite gli automi \eng{$d$-limited}.


\subsection{Automi a stati finiti}\label{subs:prel:NFA}

\subsubsection{NFA e DFA}
\begin{defin}[automa a stati finiti nondeterministico]
	Un \emph{automa a stati finiti nondeterministico} (NFA) è una quintupla $A=\tuple{Q,\Sigma,\delta,q_0,F}$, dove:
	\begin{itemize}
		\item $Q$ è un insieme finito e non vuoto di stati;
		\item $\Sigma$ è l'alfabeto di input;
		\item $\delta:Q\times\Sigma\to\subsets{Q}$ è la funzione di transizione. Se a un dato passo l'automa è nello stato $p$, legge il simbolo $\sigma$, e $\delta(p,\sigma)\ni q$, allora l'automa può passare allo stato $q$ e alla lettura del simbolo successivo;
		\item $q_0\in Q$ è lo stato iniziale;
		\item $F\subseteq Q$ è l'insieme degli stati finali.
	\end{itemize}
	Un NFA accetta una parola $w\in\Sigma\star$ se e solo se esiste una computazione che, a partire dallo stato $q_0$ e dalla lettura del primo simbolo di $w$, termina in uno stato finale dopo aver letto tutti i simboli dell'input.
\end{defin}
\begin{defin}
	Un \emph{automa a stati finiti deterministico} (DFA) è un NFA in cui $\card{\delta(q,\sigma)}\leq 1 ~ \forall q\in Q,\sigma\in\Sigma$.
\end{defin}

\begin{figure}
	\centering
	\begin{tikzpicture}[shorten >=1pt,node distance=1.1cm,auto,initial text=]
	\node[state,initial]	(q_0)			{$q_0$};
	\node[state]		(q_1)	[right=of q_0]	{$q_1$};
	\node[state,accepting]	(q_2)	[right=of q_1]	{$q_2$};
	\path[->]
	(q_0)	edge			node		{$0$} (q_1)
		edge [loop above]	node 		{$0,1$} ()
	(q_1)	edge			node		{$1$} (q_2);
\end{tikzpicture}

	\caption{Diagramma di transizione per un NFA che accetta le stringhe che finiscono in $01$}
\end{figure}

Si può dimostrare che NFA e DFA riconoscono la stessa classe di linguaggi e che essa coincide con la classe dei linguaggi regolari. La tabella \ref{tab:prel:chomskyhier} riassume la classificazione di Chomsky completa di corrispondenza con le rispettive classi di linguaggi e riconoscitori.

\begin{table}
	\caption{Classificazione di Chomsky con corrispondenza con le rispettive classi di linguaggi e riconoscitori. $a$ è un simbolo terminale, $A$ e $B$ nonterminali, $\alpha$, $\beta$ e $\gamma$ parole qualunque, con $\gamma$ non vuota.}
	\label{tab:prel:chomskyhier}
	\centering
	\begin{tabularx}{\textwidth}{lXXl}
		\toprule
		\textbf{Grammatica} & \textbf{Linguaggi generabili} & \textbf{Riconoscitore}  & \textbf{Regole di produzione}         \\
		\midrule
		Tipo 0              & Ricorsivamente enumerabili    & Macchine di Turing      & $\gamma\to\alpha$                     \\
		Tipo 1              & Dipendenti dal contesto       & Automi lineari limitati & $\alpha A\beta\to\alpha\gamma\beta$   \\
		Tipo 2              & Liberi dal contesto           & Automi a pila           & $A\to\alpha$                          \\
		Tipo 3              & Regolari                      & Automi a stati finiti   & $A\to a$, $A\to aB$, $A\to\emptyword$ \\
		\bottomrule
	\end{tabularx}
\end{table}

\subsubsection{Automi \eng{two-way}}
Un altro sistema equivalente a quello degli NFA per riconoscere i linguaggi regolari è quello degli automi \eng{two-way}, simili ad essi ma con la capacità di muoversi in ambe le direzioni tra i simboli dell'input. Useremo la formalizzazione di \cite{Pighizzini:14:limitedRE}, poiché si avvicina molto a quella per gli automi \eng{limited}. L'equivalenza degli automi two-way deterministici e DFA è discussa in \cite{Shallit:09:secondLFA}.

Un automa a stati finiti two-way ha le stesse componenti di una macchina di Turing: controllo finito, nastro e testina. Tuttavia, esso non può effettuare operazioni di scrittura; inoltre, può visitare unicamente le celle che in origine contengono l'input. Questo è infatti delimitato da due simboli speciali, il \eng{left} ($\lem$) e il \eng{right} ($\rem$) \eng{end-marker}, oltre i quali la testina non può muoversi.
\begin{defin}
	Un \emph{automa a stati finiti \eng{2-way} nondeterministico} (2NFA) è una quintupla $A=\tuple{Q,\Sigma,\delta,q_0,F}$, dove:
	\begin{itemize}
		\item $Q$ è un insieme finito e non vuoto di stati;
		\item $\Sigma$ è l'alfabeto di input;
		\item $\delta:Q\times(\Sigma\cup\set{\lem,\rem})\to\subsets{Q\times\set{\Left,\Right}}$ è la funzione di transizione. Se a un dato passo l'automa è nello stato $p$, legge il simbolo $\sigma$, e $\delta(p,\sigma)\ni (q,D)$, allora l'automa può passare allo stato $q$ e alla lettura del simbolo alla sinistra di quello corrente se $D=\Left$, o alla sua destra se $D=\Right$. Non è possibile muovere la testina a sinistra di $\lem$ e a destra di $\rem$, se non per accettare;
		\item $q_0\in Q$ è lo stato iniziale;
		\item $F\subseteq Q$ è l'insieme degli stati finali.
	\end{itemize}
	I simboli speciali $\lem$ e $\rem$ circoscrivono l'input nonché lo spazio di lavoro sul nastro.

	Un 2NFA accetta una parola $w\in\Sigma\star$ se e solo se esiste una computazione che, a partire dallo stato $q_0$ e dalla lettura del primo simbolo dell'input (il nastro contenente $\lem w\rem$), termina in uno stato finale $q\in F$ violando il \eng{right end-marker}.
\end{defin}

\begin{defin}
	Un \emph{automa a stati finiti \eng{two-way} deterministico} (2DFA) è un 2NFA in cui $\card{\delta(q,\sigma)}\leq 1 ~ \forall q\in Q,\sigma\in\Sigma\cup\set{\lem,\rem}$.
\end{defin}

Talvolta, per distinguerli dalle loro controparti two-way, gli NFA \eng{one-way} si abbreviano con 1NFA e, analogamente, i DFA con 1DFA.


\section{Complessità descrizionale}
Come visto, i linguaggi possono essere descritti da diversi sistemi: automi, grammatiche, etc. Una descrizione può essere codificata in simboli e misurata, ottenendo una rappresentazione della sua complessità. Il campo della complessità descrizionale studia le descrizioni da questo punto di vista, cercando quelle più concise e studiando quale sia il \eng{trade-off} di complessità quando si passa da una descrizione a un'altra equivalente.

Nella pratica, ogni sistema di descrizione ha caratteristiche che influenzano la sua complessità descrizionale. Per esempio, la complessità di un NFA o DFA è proporzionale al numero di stati che lo compongono. Dato un NFA di $n$ stati, la \eng{subset construction} (\cite{Rabin:59:NFA}) permette di ottenere un DFA equivalente con $2^n$ stati. Questo \eng{upper bound}, che limita la crescita in complessità nella conversione, è anche un \eng{lower bound}, cioè esistono linguaggi per cui è inevitabile che la simulazione raggiunga tale complessità. Ciò permette di concludere che la descrizione di un linguaggio da parte di un NFA può arrivare a essere esponenzialmente più efficace di quella di un DFA dal punto di vista della complessità descrizionale.

Una panoramica sulla complessità descrizionale, le sue declinazioni e i suoi sviluppi recenti è presentata in \cite{Kutrib:21:descriptional}.

\chapter{Automi \eng{\texorpdfstring{$1$}{1}-limited}}
Come accennato al paragrafo \ref{subs:prel:PDA}, il fatto che gli automi a pila deterministici riconoscano una classe intermedia tra i linguaggi liberi da contesto e quelli regolari ha portato Hibbard a voler definire una generalizzazione del determinismo nei linguaggi liberi da contesto e i loro riconoscitori. Per fare ciò, egli introdusse in \cite{Hibbard:67:CFdet} gli \eng{scan limited automata}, che oggi, con qualche piccola modifica al modello, chiamiamo automi \eng{limited}. Lo studio degli automi limited ha ormai superato il suo scopo iniziale: non riguarda più unicamente il caso deterministico e si presta particolarmente a risultati nell'ambito della complessità descrizionale.

In questo capitolo presentiamo innanzitutto gli automi \eng{$d$-limited} e il loro potere riconoscitivo, poi entriamo nel merito degli automi \eng{$1$-limited}, il nostro oggetto di studio, con una panoramica dei principali risultati che li riguardano.



\section{Automi \eng{d-limited}}
Gli automi \eng{$d$-limited} sono un caso particolare di automi linearmente limitati. Un automa \eng{$d$-limited}, con $d\in\N$, è una macchina di Turing nondeterministica in cui lo spazio di lavoro, che all'inizio contiene l'input, è circoscritto dagli \eng{end-marker} $\lem$ e $\rem$ (come già visto per i 2NFA al paragrafo \ref{subs:prel:NFA}). Inoltre, la capacità di scrittura della macchina è limitata, potendo scrivere su una cella solo durante le prime $d$ visite.
\begin{defin}[automa \eng{$d$-limited}]
	Dato un intero $d\geq 0$, un automa \eng{$d$-limited} ($d$-LA) è una tupla $A=\tuple{Q,\Sigma,\Gamma,\delta,q_0,F}$ dove:
	\begin{itemize}
		\item $Q$ è un insieme finito e non vuoto di stati;
		\item $\Sigma$ è l'alfabeto di input;
		\item $\Gamma\supseteq\Sigma\cup\set{\lem,\rem}$ è l'alfabeto di lavoro (\eng{working alphabet}), dove $\lem$ e $\rem$ circoscrivono l'input nonché lo spazio di lavoro sul nastro. $\Gamma$ è partizionato in $d+1$ sottoinsiemi $\Gamma_0,\Gamma_1,\dots,\Gamma_d$, con $\Gamma_0=\Sigma$ e $\lem,\rem\in\Gamma_d$. L'insieme $\Gamma_k$ rappresenta l'alfabeto a cui appartiene ogni simbolo alla $k$-esima visita della cella che lo contiene. Dopo la $d$-esima visita, la cella rimane invariata (\eng{frozen});
		\item $\delta:Q\times\Gamma\to \subsets{Q\times(\Gamma\setminus\set{\lem,\rem})\times\set{\Left,\Right}}$ è la funzione di transizione. Se a un dato passo l'automa è nello stato $p$, legge il simbolo $\sigma$, e $\delta(p,\sigma)\ni (q,\gamma,D)$, allora:
		      \begin{itemize}
			      \item $q$ è il prossimo stato;
			      \item $\gamma$ è il simbolo che verrà scritto in sostituzione a $\sigma$ nella cella corrente, allo spostamento della testina;
			      \item $D$ e la direzione in cui si muoverà la testina, $\Left$ se a sinistra e $\Right$ se a destra.
		      \end{itemize}
		      La natura del simbolo $\gamma$ è soggetta al partizionamento di $\Gamma$: un simbolo in $\Gamma_k$ viene sostituito con un simbolo in $\Gamma_{k+1}$ (ma un simbolo in $\Gamma_d$ viene sostituito con se stesso). Le visite in cui si cambia direzione contano doppio\footnote{Ciò è una conseguenza del fatto che, in realtà, si contano per ogni cella non le visite, ma le scansioni da sinistra a destra (visite di numero dispari) e quelle da destra a sinistra (visite di numero pari), per cui un cambio di direzione comprende entrambe.}, sostituendo un simbolo in $\Gamma_k$ con un simbolo in $\Gamma_{k+2}$. Le celle contenenti gli \eng{end-marker} non sono mai modificate, né è possibile muoversi a sinistra di $\lem$ e a destra di $\rem$, se non per accettare.
		\item $q_0\in Q$ è lo stato iniziale;
		\item $F\subseteq Q$ è l'insieme degli stati finali.
	\end{itemize}
	Una \emph{configurazione} di un automa $d$-limited si indica con $xqy$, dove $x$ è la parola prima della cella corrente (può essere omessa se $x=\emptyword$), $q$ è lo stato corrente, e $y$ è la parola che inizia con il simbolo nella cella corrente (si omettono $\lem$ e $\rem$). Si scrive $xpy\trans zqw$ se esiste una transizione che porta dalla configurazione $xpy$ a $zqw$ e $xpy\transs zqw$ se esiste una computazione in zero o più passi che porta dalla configurazione $xpy$ a $zqw$.

	Un automa $d$-limited accetta una parola $w\in\Sigma\star$ se e solo se esiste una computazione che, a partire dalla configurazione $q_0w$ (il nastro contenente $\lem w\rem$), termina in uno stato finale $q\in F$ violando il \eng{right end-marker}.

	Un automa $d$-limited si dice deterministico (D$d$-LA) se $\card{\delta(q,\gamma)}\leq 1 ~ \forall q\in Q,\gamma\in\Gamma$. Un automa si dice \eng{limited} se è \eng{$d$-limited} per qualche $d$.
\end{defin}
Sebbene formalmente il simbolo corrente viene sempre sovrascritto, nella pratica ciò non accade. Quando un simbolo deve rimanere invariato, esso viene sostituito con un simbolo equivalente rispetto alla funzione di transizione (si immagini che, ad esempio, $a$ venga sostituito con $a'$ ma comunque rappresentato con $a$). In questo modo si preserva l'informazione originale ma si mantiene la correttezza rispetto al modello. Una tecnica analoga viene usata quando si segna (\eng{mark}) una cella, per esempio sostituendo $a$ con $\bar a$, preservando il simbolo ma dandovi un valore aggiuntivo.

Si noti che gli 0-LA corrispondono esattamente ai 2NFA.

\begin{figure}
	\centering
	\begin{tikzpicture}[cell/.style={minimum height=1.5em,minimum width=1.5em,outer sep=0pt,rectangle,draw,node distance=0pt}]
	\node (lem) {\Large $\lem$};
	\node[cell] (0) [right=of lem]{$\sigma_0$};
	\node[cell] (1) [right=of 0] {$\sigma_1$};
	\node[cell] (2) [right=of 1] {$\sigma_2$};
	\node[cell] (3) [right=of 2] {$\sigma_3$};
	\node[cell, minimum width=2.5em] (worddots) [right=of 3] {$\dots$};
	\node[cell] (last) [right=of worddots] {$\sigma_n$};
	\node[node distance=0pt] (rem) [right=of last]{\Large $\rem$};
	\node[cell] (control) [above=0.75cm of 2,thick] {$q$};
	\draw[-latex] (control) -- (2);
\end{tikzpicture}

	\caption{Rappresentazione di un \la d di esempio in un dato istante.}
\end{figure}



\section{Potenza riconoscitiva}
In \cite{Hibbard:67:CFdet} Hibbard studia il diverso potere riconoscitivo dei D\la d al variare di $d$, con i D\la2 che riconoscono esattamente i linguaggi liberi da contesto deterministici. Nonostante ciò, e nonostante gli \la d siano una restrizione dei LBA, che riconoscono i \eng{context-sensitive}, è noto che i \la d nondeterministici riconoscono esattamente la classe dei linguaggi \eng{context-free}, per qualunque $d\geq2$. Questo risultato deriva dalla costruzione di alcune trasformazioni da modelli equivalenti agli automi a pila ai \la2 e viceversa, combinati con trasformazioni da \la{d+1} a \la d, per $d\geq2$. Una dimostrazione più semplice del potere riconoscitivo dei \la2 si ottiene dal teorema di Chomsky-Schützenberger (\cite{Chomsky:63:algebraCF}) e un \la2 molto semplice che riconosce i linguaggi di Dyck (dettagli in \cite{Pighizzini:19:limited}).

Per quanto riguarda $d<2$, poiché gli \la0 sono esattamente i 2NFA è chiaro che questi riconoscano esattamente i linguaggi regolari. Wagner e Wechsung hanno dimostrato in \cite{Wagner:86:compCompl} che la possibilità di riscrivere una cella durante la sua prima visita non aumenta il potere riconoscitivo, ergo anche gli \la1 riconoscono i linguaggi regolari. Come vedremo, infatti, il vero potere degli \la1 non è nella classe riconosciuta, ma nella complessità della loro descrizione, notevolmente ridotta rispetto ai riconoscitori standard per i linguaggi regolari.

Una panoramica più approfondita dei risultati ottenuti per gli automi $d$-limited, specialmente per $d\geq2$ che qui non trattiamo, si può trovare in \cite{Pighizzini:19:limited}.



\section{Complessità descrizionale}
Non sono necessari particolari costruzioni perché un \la1 simuli un'altra macchina tra i tipici riconoscitori di linguaggi regolari: è sufficiente che ignori la sua capacità di scrivere (2NFA) o quella di muoversi in ambe le direzioni (1NFA), eventualmente aggiungendo uno stato che permetta di muoversi verso destra per accettare violando il right end-marker. È invece non triviale la simulazione opposta, cioè quella di un \la1 da parte di una di tali macchine. Gli estensivi studi di queste simulazioni ci permettono non solo di dimostrare che gli \la1 riconoscano effettivamente i linguaggi regolari, ma anche quale aumento di complessità descrizionale derivi dalla costruzione.


\subsection{Upper bound}
Con una rivisitazione della costruzione di Wagner e Wechsung (teorema 12.1 in \cite{Wagner:86:compCompl}), Pighizzini e Pisoni hanno ottenuto in \cite{Pighizzini:14:limitedRE} l'upper bound del numero di stati necessari a un automa a stati finiti per simulare un automa 1-limited. La simulazione si basa sul concetto di tabella di transizione, già usato nella conversione da 2DFA in 1DFA presentata in \cite{Shepherdson:59:reduction2to1way}. Dato un \la1 di $n$ stati, la costruzione produce un 1NFA di un numero di stati esponenziale in $n$, oppure un 1DFA doppiamente esponenziale, indipendentemente dalla dimensione dell'alfabeto di lavoro. Se il \la1 è deterministico, inoltre, la stessa costruzione produce un 1DFA di numero di stati semplicemente esponenziale.
\begin{theor}[Teorema 2 in \cite{Pighizzini:14:limitedRE}]\label{thm:a1l:upper}
	Sia $M$ un \la1 di $n$ stati.
	\begin{enumerate}[(a)]
		\item \label{itm:a1l:up:NFA} $M$ può essere simulato da un 1NFA con $n\cdot2^{n^2}$ stati;
		\item \label{itm:a1l:up:DFA} $M$ può essere simulato da un 1DFA con $2^{n\cdot2^{n^2}}$ stati;
		\item \label{itm:a1l:up:det} se $M$ è deterministico, può essere simulato da un 1DFA con al più $n\cdot (n+1)^n$ stati.
	\end{enumerate}
\end{theor}
\begin{proof}
	Sia $M=\tuple{Q,\Sigma,\Gamma,\delta,q_0,F}$ un \la1, con $\card{Q}=n$.

	\ref{itm:a1l:up:NFA} Per costruire un 1NFA $A$ che simuli $M$ ci occorre uno strumento che trasferisca nella memoria a stati finiti le due facoltà degli 1-LA che $A$ non possiede: la scrittura alla prima visita e il movimento della testina verso sinistra. Ogni transizione di $A$ legge un simbolo nuovo, perciò dovrà simulare una o più transizioni di $M$.
	Dal momento che $M$ accetta violando l'end-marker destro, ogni cella deve essere visitata, e la prima visita di ogni cella segue l'ordine delle celle stesse, da sinistra verso destra. Tra la prima visita di una cella $\sigma_k$ e quella della successiva $\sigma_{k+1}$, l'unica computazione possibile è una computazione two-way in sola lettura, poiché tutte le celle che precedono $\sigma_{k+1}$ sono ormai immutabili. Una tabella di transizione rappresenta questa computazione come una relazione tra gli stati in cui può iniziare e quelli in cui può finire. Formalmente, data una stringa $z\in\Gamma_1^+$ una tabella di transizione è una relazione binaria $\tau_z\subseteq Q\times Q$ tale che
	\begin{equation*}
		(p,q)\in\tau_z \iff z'pYw \transs zqw = z'Yqw
	\end{equation*}
	con $Y\in\Gamma_1,p,q\in Q,z=z'Y,w\in\Gamma\star$. Se la testina punta a $Y$ ed è preceduta da $z'$, la tabella di transizione di $z$ indica quindi, per ogni stato in cui $M$ può trovarsi, i possibili stati in cui può uscire dalla porzione di nastro contenente $z'Y$ per visitare la cella alla destra di $Y$. La figura \ref{fig:a1l:ttt} dà una rappresentazione di $\tau_z$. Una tabella di transizione dipende ovviamente dalla stringa su cui è costruita, ma non è, in generale, unica di tale stringa (infatti le possibili tabelle sono finite).

	\begin{figure}
		\centering
		\subfloat[][$(p,q)\in \tau_{z}$ oppure $(p,q)\in t_Y(\tau_{z'})$\label{fig:a1l:ttt}]
		{
			\begin{tikzpicture}[tapeseg/.style={minimum height=1.5em,minimum width=1.5em,outer sep=0pt,node distance=0pt},cell/.style={rectangle,draw,tapeseg}]
	\footnotesize
	\node[cell] (Y) {$Y$};
	\draw (Y.north west) -- ++(-3cm,0) (Y.south west) -- ++ (-3cm,0);
	\node[tapeseg,node distance=10pt] (z) [left=of Y]{$z'$};
	\node[tapeseg,node distance=8pt] (w) [right=of Y]{$w$};
	\draw (Y.north east) -- ++(1cm,0) (Y.south east) -- ++ (1cm,0);
	\node (p) [below=.35 cm of Y] {$p$};
	\node[tapeseg,node distance=1pt] (qalign) [right=of Y] {};
	\node[tapeseg] (qalign2) [below=.2 of qalign] {};
	\node (q) [below=.35 of qalign2] {$q$};
	\draw[-latex,shorten >=1pt] (p) -- (Y);
	\draw	(p.south) -- ++(-2cm,-.12cm)
		-- ++(+.9cm,-.09cm) -- ++(-1.8cm,-.10cm)
		-- ++(2.9cm,-.12cm) -- ++(-1.55cm,-.10cm)
		-- (q.south);
	\draw[-latex] (q) -- (qalign2);
	\draw[dashed,shorten <=.1cm]
		(Y.south west) -- ++(0cm,-1.5cm);
	\draw[dashed,shorten <=.1cm]
		(Y.south east) -- ++(0cm,-1.5cm);
\end{tikzpicture}

		} \qquad
		\subfloat[][$(p,q)\in m_X(\tau_z)$\label{fig:a1l:ttm}]{
			\begin{tikzpicture}[tapeseg/.style={minimum height=1.5em,minimum width=1.5em,outer sep=0pt,node distance=0pt},cell/.style={rectangle,draw,tapeseg}]
	\footnotesize
	\node[cell] (Y) {$Y$};
	\draw (Y.north west) -- ++(-2.5cm,0) (Y.south west) -- ++ (-2.5cm,0);
	\node[tapeseg,node distance=10pt] (z) [left=of Y]{$z'$};
	\node[cell] (X) [right=of Y]{$X$};
	\node[tapeseg,node distance=8pt] (w) [right=of X]{$w$};
	\draw (X.north east) -- ++(1cm,0) (X.south east) -- ++ (1cm,0);
	\node (p) [below=.35 cm of Y] {$p$};
	\node[tapeseg,node distance=1pt] (qalign) [right=of X] {};
	\node[tapeseg] (qalign2) [below=of qalign] {};
	\node (q) [below=.35 of qalign2] {$q$};
	\draw[-latex,shorten >=1pt] (p) -- (Y);
	\draw(p.south) -- ++(-2.3cm,-.10cm)
		-- ++(2.85cm,-.12cm) -- ++(-1.55cm,-.10cm)
		-- (q.south);
	\draw[-latex] (q) -- (qalign2);
	\draw[dashed,shorten <=.1cm]
		(Y.south west) -- ++(0cm,-1.5cm);
	\draw[dashed,shorten <=.1cm]
		(Y.south east) -- ++(0cm,-1.5cm);
	\draw[dashed,shorten <=.1cm]
		(X.south east) -- ++(0cm,-1.5cm);
\end{tikzpicture}
}
		\caption{Rappresentazione delle computazioni che determinano l'appartenenza della coppia di stati $(p,q)$ a diverse tabelle di transizione. $z=z'Y$.}
	\end{figure}

	Fissato $X\in\Gamma_1$, introduciamo ora due funzioni $t_X,m_X:\subsets{Q\times Q}\to\subsets{Q\times Q}$ che trasformano tabelle di transizione. Intuitivamente, $t_X$ trasforma una tabella $\tau_z$ in $\tau_{zX}$, mentre $m_X(\tau_z)$ produce una tabella leggermente diversa, che descrive la computazione dall'ultimo simbolo di $z$ al simbolo dopo $zX$:
	\begin{equation*}
		(p,q)\in m_X(\tau_z) \iff z'pYXw \transs zXqw = z'YXqw
	\end{equation*}
	Le funzioni $t$ e $m$ sono rappresentate rispettivamente nelle figure \ref{fig:a1l:ttt} e \ref{fig:a1l:ttm}.

	Queste definizioni sono ben poste dal momento che è sufficiente conoscere $\tau_z$, senza necessariamente conoscere $z$, per produrre $m_X(\tau_z)$ e $t_X(\tau_z)$. Ciò è possibile perché, muovendosi a sinistra di $X$, si può consultare $\tau_z$ per ottenere lo stato in cui si torna in $X$, mentre in $X$ è sufficiente calcolare $\delta$. In altre parole, se $\tau_z=\tau_w$ allora $t_X(\tau_z)=t_X(\tau_w)$ e $m_X(\tau_z)=t_X(\tau_w)$. Definizioni più formali sono presenti in \cite{Pighizzini:14:limitedRE}.

	Siamo ora pronti a descrivere la forma e il comportamento di $A$.
	Ogni stato di $A$ corrisponde in $M$ alla prima visita di una cella, e una transizione deve simulare una computazione di $M$ che termina con la prima visita della successiva. Precisamente, ogni stato di $A$ è una coppia $[q,\tau]$:
	\begin{itemize}
		\item la prima componente è lo stato in cui si trova la macchina simulata $M$ durante la prima visita del simbolo corrente;
		\item la seconda componente è una tabella di transizione che permette, qualora la macchina simulata si muova a sinistra, di sapere in che stati può tornare nella posizione attuale.
	\end{itemize}
	Si ipotizzi ora che $M$ sia nella configurazione $zraw$, durante la prima visita della cella contenente $a$ ($z\in\Gamma_1\star,r\in Q,a\in\Sigma,w\in\Sigma\star$). Simulando $M$, $A$ avrà uno stato corrente $[r,\tau]$, dove $\tau=\tau_z$ (ma $A$ non ha traccia di $z$). Poiché questa è la prima visita di $a$, la prossima transizione di $M$ sostituirà il simbolo con un altro, diciamo $X$. Separiamo ora i casi di movimento della testina a destra e a sinistra.
	\begin{itemize}
		\item se la testina si muove a destra in uno stato $s$ (fig. \ref{fig:a1l:mright}), ci troviamo in corrispondenza della prima visita della cella successiva, perciò $A$ può spostarsi direttamente in un nuovo stato. La prima componente del nuovo stato sarà ovviamente $s$, mentre la tabella va aggiornata in modo da contenere l'informazione sulla cella appena lasciata, che è stata anche riscritta. La tabella per il nuovo stato sarà dunque $t_X(\tau)$, che rappresenta le possibili computazioni da $X$ per uscire dalla porzione di nastro che vi termina qualora $M$ ci dovesse tornare. Estendendo il caso al nondeterminismo il movimento a destra può variare nello stato $s$ e nel simbolo scritto $X$, sono quindi necessarie le corrispondenti transizioni in ogni stato $[s,t_X(\tau)]$;
		\item se la testina si muove verso sinistra in uno stato $s$ (fig. \ref{fig:a1l:mleft}), occorre "aspettare" che $M$ effettui una computazione che termini a destra di $X$, nella prima visita di una nuova cella. La tabella dello stato risultante, come nel caso precedente, sarà $t_X(\tau)$, poiché la computazione a sinistra della nuova cella non può essere influenzata da ulteriori scritture. Lo stato risultante $q$ ci viene invece dato da $(s,q)\in m_X(\tau)$, che descrive il comportamento di $M$ all'arrivo nella nuova cella a partire dallo stato $s$ e dal simbolo a sinistra di $X$. Considerando anche il nondeterminismo, per ogni stato $s$ e simbolo $X$ vanno quindi costruite transizioni in $[q,t_X(\tau)]$, per ogni stato $q$ tale che $(s,q)\in m_X(\tau)$.
	\end{itemize}

	\begin{figure}
		\centering
		\subfloat[][$M$ muove la testina a destra. $A$ passa nello stato ${[s,t_X(\tau)]}$.\label{fig:a1l:mright}]{
			\hspace{.7cm}\begin{tikzpicture}[tapeseg/.style={minimum height=1.5em,minimum width=1.5em,outer sep=0pt,node distance=0pt},cell/.style={rectangle,draw,tapeseg}]
	\footnotesize
	\node[cell] (a) {$a$};
	\draw[shorten >=3pt,shorten <=3pt] (a.north east) -- (a.south west);
	\draw (a.north west) -- ++(-1cm,0) (a.south west) -- ++(-1cm,0);
	\draw (a.north east) -- ++(+1cm,0) (a.south east) -- ++(+1cm,0);
	\node[tapeseg] (X) [above=of a] {$X$};
	\node[tapeseg,node distance=6pt] (z) [left=of a]{$\cdots$};
	\node[tapeseg,node distance=6pt] (w) [right=of a]{$\cdots$};
	% \node[tapeseg] (r) [below=.35 cm of a] {$r$};
	\node (r) [below=.35 cm of a] {$r$};
	\draw[-latex,shorten >=1pt] (r) -- (a);
	\node[tapeseg,node distance=1pt] (salign) [right=of a] {};
	% \node[tapeseg] (s) [below=.35 of salign] {$s$};
	\node (s) [below=.35 of salign] {$s$};
	\draw[-latex,shorten >=1pt] (s) -- (salign);
	\draw[->] (r.south) .. controls +(down:2mm) and +(down:2mm) .. (s.south);
	\draw[dashed,shorten <=.1cm]
		(a.south west) -- ++(0cm,-1.5cm);
	\draw[dashed,shorten <=.1cm]
		(a.south east) -- ++(0cm,-1.5cm);
\end{tikzpicture}
\hspace{.7cm}}
		\qquad
		\subfloat[][$M$ muove la testina a sinistra. $A$ passa nello stato ${[q,t_X(\tau)]}$.\label{fig:a1l:mleft}]{
			\hspace{.3cm}\begin{tikzpicture}[tapeseg/.style={minimum height=1.5em,minimum width=1.5em,outer sep=0pt,node distance=0pt},cell/.style={rectangle,draw,tapeseg}]
	\footnotesize
	\node[cell] (a) {$a$};
	\node[tapeseg] (X) [above=of a] {$X$};
	\draw[shorten >=3pt,shorten <=3pt] (a.north east) -- (a.south west);
	\node[tapeseg,node distance=13pt] (z) [left=of a]{$\cdots$};
	\node[tapeseg,node distance=8pt] (w) [right=of a]{$\cdots$};
	\draw (a.north west) -- ++(-2cm,0) (a.south west) -- ++ (-2cm,0);
	\node[tapeseg,node distance=1pt] (salign) [left=of a]{};
	% \node[tapeseg] (s) [below=.35 of salign] {$s$};
	\node (s) [below=.35 of salign] {$s$};
	\draw[-latex,shorten >=1pt] (s) -- (salign);
	\draw[->] (r.south) .. controls +(down:2mm) and +(down:2mm) .. (s.south);
	\draw (a.north east) -- ++(1cm,0) (a.south east) -- ++ (1cm,0);
	\node (r) [below=.35 cm of a] {$r$};
	\node[tapeseg,node distance=1pt] (qalign) [right=of a] {};
	\node[tapeseg] (qalign2) [below=.2cm of qalign] {};
	\node (q) [below=.35 of qalign2] {$q$};
	\draw[-latex,shorten >=1pt] (r) -- (a);
	\draw(s.south west) -- ++(-1cm,-.15cm)
		-- ++(1.65cm,-.17cm) -- ++(-2cm,-.12cm)
		-- (q.south);
	\draw[-latex] (q) -- (qalign2);
	\draw[dashed,shorten <=.1cm]
		(a.south west) -- ++(0cm,-1.5cm);
	\draw[dashed,shorten <=.1cm]
		(a.south east) -- ++(0cm,-1.5cm);
\end{tikzpicture}
\hspace{.3cm}}
		\caption{I due tipi di computazione di $M$ che corrispondono a una transizione in $A$.}
	\end{figure}

	Lo stato iniziale è la coppia $[q_0,\tau_\lem]$. Poiché un 1NFA non può leggere il simbolo $\rem$, è necessario un altro meccanismo per rendere uno stato finale. Uno stato $[s,\tau]$ è finale se e solo se $M$, giunto a $\rem$ nello stato $s$, possiede una computazione (in sola lettura e su tutto il nastro) che termina in uno stato finale $f\in F$ superato l'end-marker, cioè se $(s,f)\in t_\rem(\tau)$.

	Formalmente, $A:=\tuple{Q',\Sigma,\delta',q_0',F'}$ con
	\begin{itemize}
		\item $Q':=Q\times\subsets{Q\times Q}$;
		\item $q_0':=[q_0,\tau_\lem]$;
		\item $F':=\set{[p,\tau]\mid \exists q\in F: (p,q)\in t_\rem(\tau)}$
		\item dati $r,s\in Q,a\in\Sigma,X\in\Gamma_1,\tau\in\subsets{Q\times Q}$ le seguenti transizioni:
		      \begin{itemize}
			      \item per ogni $(s,X,\Right)\in\delta(r,a)$, $\delta'([r,\tau],a)$ contiene $[s,t_X(\tau)]$;
			      \item per ogni $(s,X,\Left)\in\delta(r,a)$, $\delta'([r,\tau],a)$ contiene $[q,t_X(\tau)]$ per ogni $q\in Q$ tale che $(s,q)\in m_X(\tau)$.
		      \end{itemize}
	\end{itemize}
	L'1NFA risultante ha $n\times 2^{n^2}$ stati. Benché $t_X$ e $m_X$ restituiscano diverse tabelle al variare di $X$, il numero di tabelle possibili, e quindi di stati di $A$, non dipende da $\card{\Gamma}$.

	\ref{itm:a1l:up:DFA} usando la subset construction (\cite{Rabin:59:NFA}), $A$ può essere convertito in un 1DFA con $2^{2^{n^2}}$ stati.

	\ref{itm:a1l:up:det} se $M$ è deterministico, ogni tabella di transizione è una funzione parziale $Q\to Q$, perciò $A$ ha al più $n\cdot(n+1)^n$ stati ($n+1$ immagini poiché si include la possibilità di non uscire dal segmento di nastro). Inoltre, per ogni tabella $\tau$, ogni stato $[r,\tau]$ possiede esattamente una transizione derivante da una transizione di $M$ per ciascun simbolo di input. $A$ è quindi un 1DFA.
\end{proof}


\subsection{Lower bound}
Studieremo ora i lower bound di conversione, facendo uso di \eng{witness languages} per dimostrare che esistono determinati casi in cui le complessità di conversione ottenute negli upper bound non possono essere significativamente migliorate. Per fare ciò, si cerca una famiglia di linguaggi per il cui riconoscimento esiste un \la1 con certo numero di stati $n$ e, al contempo, per cui un 1DFA necessita di un numero doppiamente esponenziale in $n$ di stati e un 1NFA, oppure 2NFA, 2DFA o D\la1 un numero esponenziale\footnote{Non è necessario, in generale, che sia la stessa famiglia di linguaggi a dimostrare l'ottimalità delle diverse conversioni; tuttavia presenteremo un caso in cui ciò si verifica.}.
\begin{theor}
	Esistono infiniti \la1 di $n$ stati la cui conversione in 1DFA richiede un numero doppiamente esponenziale in $n$ di stati.
\end{theor}

Si consideri il linguaggio $L_n$, dove $1\leq n\in\N$ è un parametro, delle parole binarie a blocchi di $n$ simboli in cui $n$ blocchi sono uguali:
\begin{align*}
	L_n := \{ & x_1x_2\cdots x_k\mid k\geq0, x_1,x_2,\dots,x_k\in\{0,1\}^n,                                   \\
	          & \exists i_1,i_2,\dots,i_n\in\{1,\dots,k\},i_1<i_2<\dots<i_n, x_{i_1}=x_{i_2}=\dots=x_{i_n}\}
\end{align*}

\begin{theor}
	$L_n$ può essere riconosciuto da un \la1 con un numero di stati di $O(n)$ e un alfabeto di lavoro indipendente da $n$.
\end{theor}
\begin{proof}
	Un \la1 $A_n$ può riconoscere $L_n$ come segue:
	\begin{enumerate}
		\SetKwData{B}{b}\SetKwData{C}{c}
		\item $A_n$ effettua una scansione preliminare dell'input, da sinistra verso destra, che verifichi tramite un contatore modulo $n$ che la parola sia effettivamente composta da blocchi di $n$ simboli, ossia che la sua lunghezza sia multipla di $n$. Se si raggiunge il right end-marker con il contatore non a zero, la macchina si ferma rifiutando. Durante questa scansione, l'unica in cui $A_n$ ha facoltà di scrittura, la macchina sceglie inoltre in modo nondeterministico delle celle, provando a indovinare quelle che danno inizio ai blocchi uguali. Ciò viene fatto sostituendo i simboli con una loro versione segnata, che però mantenga l'informazione originale (per esempio $0$ in $\hat 0$), e in corrispondenza dell'azzeramento del contatore, in modo che i simboli segnati siano necessariamente all'inizio di blocchi. Questa fase richiede $O(n)$ stati per il contatore;
		\item viene poi effettuata una scansione verso sinistra che verifichi che il numero di celle segnate, e quindi di blocchi candidati, sia $n$. Anche questa scansione richiede $O(n)$ stati;
		\item \label{itm:a1l:lowLn:LA3} la macchina prosegue poi verificando in sola lettura l'uguaglianza dei blocchi candidati, confrontando coppie di blocchi simbolo a simbolo. Per fare ciò la macchina implementa l'algoritmo \ref{alg:a1l:lowLn:3f}, una versione rivista e corretta dell'algoritmo presentato in \cite{Pighizzini:14:limitedRE}, che presentava qualche errore. Si ipotizzi di prendere in considerazione una coppia di blocchi candidati successivi. L'algoritmo fa uso di due variabili: la variabile intera \C, il cui dominio è $\set{0,\dots,n-1}$, rappresenta l'indice del simbolo che dev'essere confrontato nei due blocchi; la variabile \B contiene una copia del simbolo all'indice \C del primo blocco per confrontarla con il corrispondente simbolo nel secondo. Inizialmente $A_n$ scansiona da $\lem$ verso destra alla ricerca del primo simbolo segnato, e \C è inizializzata a $0$. A questo punto si entra nel ciclo principale, che confronta due blocchi adiacenti tra quelli segnati. La porzione di codice a righe \refalgrange{algln:a1l:lowLn:3f}{1}{2} sposta la testina alla posizione di indice \C del primo blocco decrementando \C, salva una copia del simbolo in \B, quindi ripristina il valore di \C (modificando direttamente \C si risparmia l'utilizzo di un ulteriore indice variabile). Quindi la macchina prosegue individuando l'inizio del secondo blocco, accettando se trova invece $\rem$. Effettua quindi una procedura simile alla precedente per localizzare il simbolo in posizione \C (righe \refalgrange{algln:a1l:lowLn:3f}{3}{4}), questa volta però confrontandolo con quello contenuto in \B. Se i due sono diversi la macchina rifiuta, altrimenti prosegue. A questo punto, se $\C=n-1$ significa che i due blocchi sono stati confrontati, quindi si passa al confronto del secondo con il prossimo dei candidati. Per fare ciò è necessario unicamente resettare \C a $0$, dal momento che la testina è già nella corretta posizione. Se invece $\C<n-1$, si incrementa \C e si torna all'inizio del primo blocco per confrontare il prossimo simbolo. Questa fase richiede un numero di stati di $O(n)$ per memorizzare i valori di \B e \C.
	\end{enumerate}

	\IncMargin{1em}
	\begin{algorithm}
		\DontPrintSemicolon
\SetKwData{B}{b}\SetKwData{C}{c}
\Repeat{la cella corrente è segnata}{
	sposta la testina a destra di una cella\;
}
$\C \leftarrow 0$\;
\Repeat{la cella corrente contiene $\rem$}{
	\label{algln:a1l:lowLn:3f:1} \While{$\C>0$}{
		sposta la testina a destra di una cella\;
		$\C \leftarrow \C-1$\;
	}
	$\B \leftarrow$ simbolo corrente\;
	\While{la cella corrente non è segnata}{
		sposta la testina a sinistra di una cella\;
		\label{algln:a1l:lowLn:3f:2} $\C \leftarrow \C+1$\;
	}
	\Repeat{la cella corrente è segnata o contiene $\rem$}{
		sposta la testina a destra di una cella\;
	}
	\If{la cella corrente non contiene $\rem$}{
		\label{algln:a1l:lowLn:3f:3} \While{$\C>0$}{
			sposta la testina a destra di una cella\;
			$\C \leftarrow \C-1$\;
		}
		\If{il simbolo corrente è diverso da \B}{
			RIFIUTA\;
		}
		\While{la cella corrente non è segnata}{
			sposta la testina a sinistra di una cella\;
			\label{algln:a1l:lowLn:3f:4} $\C \leftarrow \C+1$\;
		}
		\If{$\C<n-1$}{
			\Repeat{la cella corrente è segnata}{
				sposta la testina a sinistra di una cella\;
			}
			$\C\leftarrow \C+1$\;
		}
		\Else{
			$\C\leftarrow 0$\;
		}
	}
}
ACCETTA\;

		\caption{\hyperref[itm:a1l:lowLn:LA3]{Terza fase} del riconoscimento di $L_n$ da parte di $A_n$}
		\label{alg:a1l:lowLn:3f}
	\end{algorithm}
	\DecMargin{1em}
\end{proof}

Per dimostrare l'ottimalità dell'upper bound trovato in precedenza, dimostriamo che qualunque conversione non porterebbe alla costruzione di un 1DFA equivalente ad $A_n$ che abbia meno di un numero doppiamente esponenziale in $n$ di stati, poiché un tale 1DFA non può esistere. Il risultato viene poi esteso per l'ottimalità dei bound relativi a 1NFA, 2NFA, 2DFA e D\la1.
\begin{theor}\label{thm:a1l:lowLn}
	Sia $1\leq n\in\N$. Allora:
	\begin{enumerate}[(a)]
		\item \label{itm:a1l:lowLn:DFA} un 1DFA che riconosca $L_n$ necessita di un numero di stati doppiamente esponenziale in $n$;
		\item \label{itm:a1l:lowLn:1LA} un \la1 che riconosca $L_n$ necessita di un numero di stati almeno polinomiale in $n$;
		\item \label{itm:a1l:lowLn:NFA} un D\la1, un 2NFA, 2DFA o 1NFA che riconosca $L_n$ necessita di un numero di stati esponenziale in $n$.
	\end{enumerate}
\end{theor}
\begin{proof}
	\ref{itm:a1l:lowLn:DFA} Per dimostrare che un 1DFA che riconosca $L_n$ necessita di $2^{2^n}$ stati usiamo la tecnica delle stringhe distinguibili. Fissato un alfabeto, due stringhe $x$ e $y$ si dicono \emph{distinguibili} rispetto a un linguaggio $L$ se e solo se esiste una stringa $z$ tale che esattamente una tra le parole $xz$ e $yz$ appartiene a $L$. La cardinalità di un qualunque insieme di stringhe a due a due distinguibili è lower bound per il numero di stati necessari a un 1DFA per riconoscere il relativo linguaggio\footnote{Questa proprietà è una conseguenza del teorema di Myhill-Nerode se si sceglie come relazione di equivalenza l'indistinguibilità di stringhe.}.

	Sia $x_1,x_2,\dots,x_{2^n}$, una lista in qualunque ordine di tutti i possibili blocchi, cioè stringhe in $\set{a,b}^n$. Sia $F:=\set{f: \set{0,\dots,2^n}\to\set{0,\dots,n-1}}$ l'insieme delle funzioni che mappano gli indici dei blocchi a interi tra $0$ e $n-1$. Scelto $f\in F$, sia $w_f$ la parola formata dalla giustapposizione dei blocchi $x_1,x_2,\dots,x_{2^n}$, il blocco di indice $i$ ripetuto $f(i)$ volte, ossia $w_f:=x_1^{f(1)}x_2^{f(2)}\cdots x_{2^n}^{f(2^n)}$. Si considerino ora due funzioni $f,g\in F$ tali che per un certo indice $j$, $f(j)\neq g(j)$. Le stringhe $w_f$ e $w_g$ sono distinguibili dalla stringa $z:=x_j^{n-\max(f(j),g(j))}$, poiché una e una sola delle due parole $w_fz$ e $w_gz$ raggiunge $n$ ripetizioni del blocco $x_j$. Per esempio, se $f(j)>g(j)$:
	\begin{equation*}
		w_fz=x_1\cdots \underbrace{x_j\cdots x_j}_{f(j)} \cdots x_{2^n}\underbrace{x_j\cdots x_j}_{n-f(j)}\in L_n \qquad w_gz=x_1\cdots \underbrace{x_j\cdots x_j}_{g(j)} \cdots x_{2^n}\underbrace{x_j\cdots x_j}_{n-f(j)}\notin L_n
	\end{equation*}
	Dove il totale di ripetizioni del blocco $x_j$ sono $n$ per $w_fz$ e $g(j)+n-f(j)<n$ per $w_gz$.

	Ne si deduce, quindi, che la cardinalità dell'insieme delle parole $w_f$ è lower bound per il numero di stati di una 1DFA che riconosca $L_n$. Tale numero coincide con il numero di possibili funzioni $f$, ossia con $|F|=n^{2^n}=\Omega(2^{2^n})$.

	\ref{itm:a1l:lowLn:1LA} Se il numero di stati di un \la1 che riconosca $L_n$ fosse meno che polinomiale in $n$, allora in virtù del teorema \ref{thm:a1l:upper} si potrebbe costruire un 1DFA meno che doppiamente esponenziale in $n$, il che contraddirebbe \ref{itm:a1l:lowLn:DFA}.

	\ref{itm:a1l:lowLn:NFA} Con un ragionamento analogo, considerando le costruzioni che permettono ai 1DFA di simulare 1NFA, 2NFA, 2DFA e D\la1, tutte di crescita almeno esponenziale, dimostriamo che è impossibile che una di queste macchine che riconosca $L_n$ abbia un numero di stati meno che esponenziale in $n$.
\end{proof}
\noindent La dimostrazione di Pighizzini e Pisoni (\cite{Pighizzini:14:limitedRE}) estende il lower bound per i 1DFA a $((2^n-2)\cdot(\frac{n-1}{n})^2+1)\cdot n^{2^n}+1$ (tramite distinguibilità) e quello dei 1NFA a $n^2\cdot2^n$ (tramite \eng{fooling set}).

Il teorema \ref{thm:a1l:lowLn}, oltre a dimostrare che le conversioni da \la1 non possono essere migliorate significativamente (ossia così da rimuovere un livello di esponenzialità), trova un lower bound esponenziale alla conversione da \la1 a D\la1:
\begin{corol}
	La simulazione di \la1 nondeterministici da parte di \la1 deterministici richiede un fattore esponenziale di stati.
\end{corol}
\begin{proof}
	L'esistenza di un D\la1 deterministico meno che esponenziale in $n$ che riconosca $L_n$ permetterebbe la costruzione di un 1DFA equivalente meno che doppiamente esponenziale in $n$ con la simulazione di cui al teorema \ref{thm:a1l:upper}, il che alla luce del teorema \ref{thm:a1l:lowLn} è assurdo.
\end{proof}

\chapter{Witness languages}\label{cha:wit}
Quando esiste un linguaggio con una certa proprietà, si dice che tale linguaggio è un \eng{witness language} ("linguaggio testimone") di quella proprietà. I witness language rappresentano casi peggiori, e sono utili ad ottenere lower bound, risolvere problemi aperti e in generale ad avere una conoscenza più approfondita di un modello oggetto di studio.
Nel paragrafo \ref{subs:a1l:low} abbiamo studiato il linguaggio $L_n$, testimone del lower bound esponenziale della simulazione di \la1 da parte di 1NFA, 2NFA, 2DFA e D\la1 e doppiamente esponenziale da parte di 1DFA.
In questo capitolo studieremo alcune tecniche che permettono di costruire riconoscitori per vari linguaggi, e dimostreremo che questi linguaggi sono testimoni di diversi lower bound che riguardano gli automi $1$-limited.
Classifichiamo tali linguaggi in base a una caratteristica comune che ne rende lo studio (almeno in parte) uniforme: \emph{a blocchi}, \emph{unari}, \emph{con reset}. I linguaggi unari, in particolare, sono considerati un caso speciale, in quanto non hanno una struttura vera e propria e sono caratterizzati unicamente dalla lunghezza delle loro parole.



\section{Linguaggi a blocchi}\label{sec:wit:blk}
In un linguaggio a blocchi di parametro $n$, ogni parola è composta dalla concatenazione di stringhe di lunghezza $n$, dette blocchi. Condizioni diverse sulla relazione tra i blocchi danno origine a diverse famiglie di linguaggi, ad esempio:
\begin{itemize}
	\item il linguaggio delle parole in cui l'ultimo blocco è uguale a uno dei precedenti:
	      \begin{equation*}
		      K_n := \set{x_1\cdots x_kx \mid k>0, x_1,\dots,x_k\in\set{0,1}^n, \exists j\in\set{1,\dots,k}: x_j=x}
	      \end{equation*}
	\item il linguaggio delle parole in cui due blocchi qualsiasi sono uguali:
	      \begin{equation*}
		      E_n := \set{x_1\cdots x_k \mid k>0, x_1,\dots,x_k\in\set{0,1}^n,\exists i,j\in\set{1,\dots,k},i<j: x_i=x_j}
	      \end{equation*}
	\item il linguaggio delle parole in cui $n$ blocchi sono uguali:
	      \begin{align*}
		      L_n := \{ & x_1x_2\cdots x_k\mid k\geq0, x_1,x_2,\dots,x_k\in \set{0,1}^n,                                  \\
		                & \exists i_1,i_2,\dots,i_n\in\set{1,\dots,k},i_1<i_2<\dots<i_n: x_{i_1}=x_{i_2}=\dots=x_{i_n} \}
	      \end{align*}
\end{itemize}

Per $L_n$ sono stati descritti nel paragrafo \ref{subs:a1l:low} un \la1 riconoscitore e il lower bound sul numero di stati di ogni 1DFA, \la1, e 1NFA, 2NFA, 2DFA, D\la1 che riconosce $L_n$. Gli stessi risultati possono essere facilmente adattati a $K_n$ e $E_n$, usando una variante dell'algoritmo \ref{alg:a1l:lowLn:3f} per il riconoscimento e la distinguibilità per i lower bound.
I tre linguaggi possono quindi essere usati per dimostrare che esistono casi in cui per la simulazione di un \la1 di $n$ stati da parte di un 1NFA, 2NFA, 2DFA, D\la1 è necessario un numero esponenziale in $n$ di stati e da parte di un 1DFA un numero doppiamente esponenziale.


\subsection{Riconoscitori}
Descriviamo ora i riconoscitori mancanti per questi linguaggi, cioè 1DFA, 1NFA, 2DFA, 2NFA e D\la1. Usiamo ancora una volta l'esempio di $L_n$ poiché le tecniche sono molto simili tra i diversi linguaggi.

\subsubsection{1DFA}
Un 1DFA che riconosce $L_n$ può contare le occorrenze di ogni possibile blocco durante la scansione dell'input. Per fare ciò, ogni stringa $x\in\set{0,1}^n$ ha un contatore da $0$ a $n-1$ associato. Per l'identificazione di un blocco dell'input, gli stati sono organizzati ad albero, in cui ogni nodo rappresenta un prefisso di una possibile stringa. Giunta a una foglia la macchina incrementa il contatore della stringa associata e inizia a percorrere l'albero successivo, la cui radice coincide con la foglia. Trovata la $n$-esima occorrenza di una stringa, l'automa si limita a contare modulo $n$ la restante lunghezza della parola di input per verificare la struttura a blocchi. Per la prima fase vengono usati $n^{2^n}$ stati per i contatori; ogni incremento di un contatore dipende da un albero binario completo di $2^n-1$ stati. Per la seconda fase sono sufficienti $n$ stati, per un totale di $(2^n-1)\cdot n^{2^n}+n$ stati.

\subsubsection{1NFA}
La figura \ref{img:wit:LnNFA} mostra un NFA che riconsce $L_n$. All'inizio l'automa prova a indovinare nondeterministicamente una stringa $x\in\set{0,1}^n$ che ritiene essere il blocco ripetuto, quindi l'automa conta le occorrenze di $x$ nell'input. Per fare ciò, utilizza $2n-1$ stati per ciascuna occorrenza (per ciascuno dei blocchi possibili). Le transizioni contrassegnate con $\ok$ indicano un confronto positivo tra il simbolo corrente e il rispettivo nel blocco candidato, quelle con $\nok$ un confronto negativo e quelle con $\any$ non dipendono dal confronto e contano semplicemente i simboli di input. I valori effettivi di $\ok$ e $\nok$ dipendono ovviamente dal blocco scelto all'inizio.

Si supponga di aver contato l'$i$-esima occorrenza del candidato, con $i\in\set{0,\dots,n-1}$, e di star verificando l'$(i+1)$-esima.
\begin{itemize}
	\item Finché il blocco corrente coincide con il candidato, vengono effettuate le transizioni $\ok$, proseguendo negli stati $x_{i,j}$, dove $j\in\set{0,\dots,n-1}$ è l'indice del simbolo che viene confrontato. Verificata la coincidenza dell'ultimo simbolo, cioè certificata l'occorrenza $i+1$, si passa al confronto del blocco successivo incrementando il contatore $i$ e passando quindi agli stati $x_{i+1,j}$;
	\item se i due blocchi non coincidono, il primo simbolo diverso tra i due porta l'automa a prendere una transizione $\nok$, proseguendo poi negli stati $\bar x_{i,j}$, che contano i simboli fino alla fine del blocco senza confrontare. Al termine di questa serie si riprende la computazione dallo stato $x_{i,j}$ alla ricerca della $(i+1)$-esima occorrenza nel blocco successivo.
\end{itemize}
Una volta trovate $n$ occorrenze, l'automa passa in una serie di stati $f_n,f_1,\dots,f_{n-1}$, comuni a tutti i blocchi candidati, che contano modulo $n$ i simboli rimanenti, accettando se sono in numero multiplo di $n$. Il numero di stati è quindi in totale $1+2^n\cdot (2n-1)\cdot n+n$.

\begin{figure}
	\centering
	\begin{tikzpicture}[shorten >=1pt,initial text=,near/.style={node distance=5mm}]
	\scriptsize
	\node[state,initial] (q0) {\small $q_0$};
	\node[state,node distance=12mm] (x00) [right=of q0] {$x_{0,0}$};
	\node[node distance=8mm] (dots1) [above=of x00] {$\dots$};
	\node (y1) [above=of dots1]{$x^{(1)}$};
	\node (dots2) [below=of x00] {$\dots$};
	\node (y2n) [below=of dots2] {$x^{(2^n)}$};

	\path[->]
	(q0) edge[bend left] node [above] {$\emptyword$} (y1.west)
	(q0) edge[bend left] node [above] {$\emptyword$} (dots1.west)
	(q0) edge[bend right] node [below] {$\emptyword$} (x00.west)
	(q0) edge[bend right] node [below] {$\emptyword$} (dots2.west)
	(q0) edge[bend right] node [below] {$\emptyword$} (y2n.west);

	\node[state,near] (x01) [right=of x00] {$x_{0,1}$};
	\node[state,near] (x02) [right=of x01] {$x_{0,2}$};
	\node[near] (x02d) [right=of x02] {$\dots$};
	\node[state,near] (x0n-1) [right=of x02d] {\tiny $x_{0,n-1}$};

	\path[->]
	(x00)	edge node [above] {$\ok$} (x01)
	(x01)	edge node [above] {$\ok$} (x02)
	(x02)	edge node [above] {$\ok$} (x02d)
	(x02d)	edge node [above] {$\ok$} (x0n-1);

	\node[state,node distance=8mm] (bx01) [below right=of x00] {$\bar x_{0,1}$};
	\node[state,near] (bx02) [right=of bx01] {$\bar x_{0,2}$};
	\node[near] (bx02d) [right=of bx02] {$\dots$};
	\node[state,near] (bx0n-1) [right=of bx02d] {\tiny $\bar x_{0,n-1}$};

	\path[->]
	(x00)	edge node [left] {$\nok$} (bx01)
	(bx01)	edge node [below] {$\any$} (bx02)
	(bx02)	edge node [below] {$\any$} (bx02d)
	(bx02d)	edge node [below] {$\any$} (bx0n-1)
	(x01)	edge node [left] {$\nok$} (bx02)
	(x02)	edge node [left] {$\nok$} (bx02d);

	\node[state,near] (x10) [right=of x0n-1] {$x_{1,0}$};

	\path[->]
	(bx0n-1) edge[bend angle=70,bend left] node[below] {$\any$} (x00)
	(x0n-1) edge[bend angle=45,bend right] node[below] {$\nok$} (x00)
	(x0n-1) edge node [above] {$\ok$} (x10);

	\node[near]		(x10d)		[right=of x10] {$\dots$};
	\node[state,near]	(in-1n-1)	[right=of x10d] {\tiny $x_{n-1,n-1}$};
	\node[state]		(f1)		[right=of in-1n-1.north] {$f_1$};
	\node[state,near,accepting]	(fn)		[above=of f1] {$f_n$};
	\node[state,near]		(f2)		[below=of f1] {$f_2$};
	\node[near]	(fd)	[below=of f2] {$\dots$};
	\node[state,near]	(fn-1)	[below=of fd] {\tiny $f_{n-1}$};

	\path[->]
	(in-1n-1)	edge node[left] {$\ok$} (fn)
	(fn)		edge node[right] {$\any$} (f1)
	(f1)		edge node[right] {$\any$} (f2)
	(f2)		edge node[right] {$\any$} (fd)
	(fd)		edge node[right] {$\any$} (fn-1)
	(fn-1)		edge[bend right] node[right] {$\any$} (fn);
\end{tikzpicture}

	\caption{L'NFA che riconosce $L_n$. La prima mossa è rappresentata come $\emptyword$-mossa, cioè non legge alcun simbolo.}
	\label{img:wit:LnNFA}
\end{figure}

\subsubsection{2DFA}
Un 2DFA può riconoscere $L_n$ eseguendo innanzitutto una scansione preliminare che verifichi che la lunghezza dell'input sia multipla di $n$. Quindi, per ogni possibile blocco $x\in\set{0,\dots,n-1}$, effettua una scansione del nastro contando le occorrenze di $x$ (con una strategia simile a quella dell'1NFA descritto precedentemente). Trovate $n$ occorrenze di un blocco, accetta violando l'end-marker destro. Il numero di stati utilizzati è $n$ per la prima fase, $2^n\cdot (2n-1)\cdot n$ per la seconda e $1$ per accettare, per un totale di $n+2^n\cdot (2n-1)\cdot n+1$ come nel caso precedente.

Questa macchina è anche un 2NFA e D\la1, di cui non conosciamo riconoscitori migliori che, per riconoscere $L_n$, sfruttino le capacità in più dei rispettivi modelli.



\section{Linguaggi unari}\label{sec:wit:un}
I linguaggi unari possiedono la fondamentale proprietà secondo cui le classi dei context-free e dei regolari coincidono \cite{Ginsburg:62:unary}. Da ciò deriva il fatto che, nel caso unario, gli automi $d$-limited caratterizzano i linguaggi regolari, nonché i context-free, per qualunque $d$.

La relazione tra automi limited e linguaggi unari è stata studiata estensivamente da Pighizzini e Prigioniero in \cite{Pighizzini:19:limitedunary}. In particolare, i \la1 che riconoscono linguaggi unari possono fare uso di una tecnica basata sulla \eng{binary carry sequence}:
\begin{defin}
	La \emph{binary carry sequence} è la successione infinita di interi $\sigma_1\sigma_2\cdots\sigma_j\cdots$ in cui $\sigma_j$ è l'esponente della più alta potenza di $2$ che divide $j$, per ogni intero $j\geq1$.
\end{defin}

Definiamo inoltre la \eng{backward increasing sequence}, una funzione che trasforma sequenze e che ha proprietà interessanti in relazione alla binary carry sequence.
\begin{defin}
	Sia $s=k_1k_2\cdots k_j$ una sequenza finita di interi. La \emph{backward increasing sequence} di $s$, denotata con $\bis(s)$, è la più lunga successione ottenibile selezionando da destra verso sinistra un elemento di $s$ solo se è maggiore dell'ultimo selezionato. Formalmente, $\bis(k_1k_2\cdots k_j)=(i_1,i_2,\dots,i_r)$ se e solo se $i_1=k_{h_1},i_2=k_{h_2},\dots,i_r=k_{h_r}$ dove $h_1=j$ e $h_t=\max\set{h'<h_{t-1}\mid k_{h'}>k_{h_{t-1}}}$.
\end{defin}

Si verifica il seguente risultato, dimostrato in \cite{Pighizzini:19:limitedunary}:
\begin{lemma}\label{lem:wit:bis}
	Sia $\sigma_1\sigma_2\cdots\sigma_j$ il prefisso di lunghezza $j$ della binary carry sequence.
	\begin{itemize}
		\item \label{lem:wit:bis:1} Se $\bis(\sigma_1\sigma_2\cdots\sigma_j)=(i_1,i_2,\dots,i_r)$ allora
		      \begin{equation*}
			      j=\sum_{t=1}^r 2^{i_t}
		      \end{equation*}
		      Ossia, i valori della backward increasing sequence applicata al prefisso di lunghezza $j$ della binary carry sequence corrispondono alle posizioni dei bit a $1$ nella rappresentazione binaria di $j$.
		\item \label{lem:wit:bis:2} $\sigma_j$ è il minor numero naturale che non occorre in $\bis(\sigma_1\sigma_2\cdots\sigma_{j-1})$.
	\end{itemize}
\end{lemma}

Spieghiamo ora una tecnica che permette di riconoscere diversi linguaggi unari facendo uso del lemma \ref{lem:wit:bis} per contare i simboli dell'input. Si prenda in considerazione il linguaggio singoletto $\set{a^{2^n}}$, dove $n>0$ è un parametro intero fissato. Lo scopo della macchina è quello di scrivere il prefisso di lunghezza $2^n$ della binary carry sequence sul nastro, sostituendo le $a$.

Un D\la1 $A_n$ con alfabeto di lavoro $\set{a,0,1,\dots,n}$ può innanzitutto sovrascrivere il primo simbolo con $0$, primo elemento della binary carry sequence. Supponendo che a un certo punto della computazione la macchina abbia scritto il prefisso di lunghezza $j$ della binary carry sequence sovrascrivendo i primi $j$ simboli, il simbolo $\sigma_{j+1}$ può essere calcolato grazie al \hyperref[lem:wit:bis:2]{secondo punto} del lemma \ref{lem:wit:bis}: $A_n$ può effettuare visite in sola lettura dall'ultima cella scritta verso sinistra, individuando il più piccolo naturale che non occorre nella backward increasing sequence del prefisso scritto. Tale numero viene poi scritto nella successiva cella scrivibile e il procedimento viene ripetuto. Se a un certo punto della computazione l'automa scrive $n$ ($2^n$-esimo elemento della binary carry sequence) e la successiva cella contiene $\rem$, l'automa accetta. Se si raggiunge $\rem$ senza che $n$ venga scritto, allora la parola di input è troppo corta, mentre se viene scritto ma la successiva cella non contiene l'end-marker allora è troppo lunga. $A_n$ può essere implementato con $O(n)$ stati e usa $O(n)$ simboli (dettagli sull'implementazione e l'algoritmo utilizzato sono presenti in \cite{Pighizzini:19:limitedunary}).

Si può adattare $A_n$ ad accettare $\set{a^{2^n}}\star$, scrivendo al posto di $n$ un simbolo di reset $\reset$, in cui l'automa si comporta come sull'end-marker sinistro. La macchina accetta se $\reset$ precede $\rem$ o se riceve in input la stringa vuota. Un 1NFA necessita di $2^n$ stati per un contatore al fine di riconoscere questo linguaggio. Mereghetti e Pighizzini hanno dimostrato in \cite{Mereghetti:00:twoway} che lo stesso lower bound vale per i 2NFA. Questo linguaggio è quindi testimone del costo almeno esponenziale tra D\la1 a 2NFA (e quindi anche tra \la1 e 1NFA, 1DFA e 2DFA), anche per i linguaggi unari.

Una variante di questa tecnica può essere applicata per riconoscere il linguaggio $M_N:=\set{a^N}\star$, con $N>0$ un intero qualsiasi. Per fare ciò, un D\la1 $B_N$ può scrivere il prefisso di lunghezza $N-1$ della binary carry sequence con la tecnica descritta precedentemente, quindi scrivere un simbolo di reset $\reset$ che equivale a $\lem$ e impone alla macchina di ricominciare la scrittura dal simbolo $\sigma_1$. Questo procedimento viene ripetuto per tutto l'input: la stringa ha lunghezza multipla di $N$ se e solo se l'ultimo simbolo prima dell'end-marker destro è $\reset$ o $\lem$. Se $w$ è l'input il nastro verrà riscritto come segue:
\begin{equation*}
	\underbrace{\sigma_1\cdots\sigma_{N-1}\reset\cdots\reset\sigma_1\cdots\sigma_{N-1}\reset}_{\floor{\len w/N} \text{ volte}}\sigma_1\cdots\sigma_{\scriptscriptstyle \len w\mkern -11mu \mod N}\text.
\end{equation*}
Poiché $N$ non è necessariamente una potenza di $2$, rilevare quando è il momento di scrivere $\reset$ non è triviale. Per fare ciò $B_N$, durante il procedimento di identificazione del prossimo simbolo da scrivere, verifica inoltre se la backward increasing sequence del prefisso attuale è composta dagli indici dei bit a $1$ nella rappresentazione binaria di $N-1$. Se così fosse, il \hyperref[lem:wit:bis:1]{primo punto} del lemma \ref{lem:wit:bis} dimostra che il prefisso attuale ha lunghezza $N-1$, e che quindi il prossimo simbolo da scrivere è $\reset$.

L'implementazione di $B_N$ richiede un numero di stati e un alfabeto di lavoro lineari nel massimo elemento della binary carry sequence che può essere scritto, ossia $O(\log N)$.



\section{Linguaggi con reset}


\subsection{L'automa di Meyer e Fischer}
Con l'intenzione di studiare un witness language che non fosse né unario né a blocchi, Pighizzini, Prigioniero e Sádovský hanno studiato in \cite{Pighizzini:22:limitedwitness} il riconoscimento da parte di \la1 del linguaggio accettato dall'automa $S_N$, introdotto da Meyer e Fischer in \cite{Meyer:71:ecodescription} come testimone del costo esponenziale tra 1NFA e 1DFA. L'automa, rappresentato in figura \ref{img:wit:Sn}, ha una struttura ciclica: se si considerano solo le transizioni generate dal simbolo $a$ l'automa riconosce il linguaggio $\set{a^N}\star$. Durante il riconoscimento di una stringa, il ruolo di $b$ è scelto nondeterministicamente tra due: o viene ignorato, lasciando invariato lo stato, o impone alla macchina un reset, riportandola allo stato iniziale.

\begin{figure}
	\centering
	% \begin{tikzpicture}
\begin{tikzpicture}[shorten >=1pt,initial text=]
	\def\stateangle{30}
	\def\statedistance{2cm}
	\path[inner sep=0]
	(0,0) node[state](5){$q_{N-1}$}

	++(60:\statedistance) node[state,accepting,initial](0) {$q_0$}
	++(0:\statedistance) node[state](1){$q_1$}
	++(-60:\statedistance) node[state](2){$q_2$}

	(5)
	++(-60:\statedistance)  node[state](4){$q_{N-2}$}
	++(0:\statedistance)
	node[state](3){$q_3$};

	\path[->]
	(1) edge[loop above] node[near end,right] {$b$} (1)
	(2) edge[loop right] node[near end,below] {$b$} (2)
	(3) edge[loop below] node[near end,left] {$b$} (3)
	(4) edge[loop below] node[near end,left] {$b$} (4)
	(5) edge[loop left] node[near end,above] {$b$} (5)

	(0) edge[bend left] node[above] {$a$} (1)
	(1) edge[bend left] node[right] {$a$} (2)
	(2) edge[bend left] node[right] {$a$} (3)
	(3) edge[bend left,dashed]  (4)
	(4) edge[bend left] node[left] {$a$} (5)
	(5) edge[bend left] node[left] {$a$} (0)

	(1) edge[bend left] node[above] {$b$} (0)
	(2.170) edge[out=170,in=-45] node[above] {$b$} (0.-45)
	(3) edge[] node[above,near start] {$b$} (0)
	(4.70) edge[out=70,in=-75] node[left] {$b$} (0.-75)
	(5) edge[bend right] node[left] {$b$} (0)
	;
\end{tikzpicture}

	\caption{L'NFA $S_N$ di Meyer e Fischer.}
	\label{img:wit:Sn}
\end{figure}

È stato dimostrato in \cite{Meyer:71:ecodescription} che il minimo 1DFA equivalente a $S_N$ ha $2^N$ stati. In \cite{Pighizzini:22:limitedwitness} viene costruito un 2DFA di $N+2$ stati equivalente a $S_N$, il che testimonia il costo esponenziale tra D\la1 e 1DFA, e viene dimostrato che il minimo 2NFA equivalente a $S_{2^n}$, con $n>0$, ha almeno $2^n$ stati.

Per simulare con \la1 automi con un meccanismo di reset simile a quello di $S_N$, si può utilizzare una variante della tecnica basata sulla binary carry sequence presentata nel paragrafo \ref{sec:wit:un}. In particolare, per ogni $N>1$, $S_N$ può essere riconosciuto da un \la1 $C_N$ con $O(\log N)$ stati e un alfabeto di lavoro di $O(\log N)$ simboli.
Poiché, come accennato in precedenza, se $S_N$ riceve in input stringhe composte unicamente dal simbolo $a$ l'automa riconosce il linguaggio unario $\set{a^N}\star$, in questo caso $C_N$ può comportarsi esattamente come $B_N$, l'automa che riconosce tale linguaggio. $C_N$ costruisce quindi una serie di ripetizioni del prefisso di lunghezza $N-1$ della binary carry sequence, separandole con il carattere di reset $\reset$. Per quanto riguarda il comportamento di $C_N$ per il simbolo di input $b$, la macchina può scegliere nondeterministicamente di effettuare una di due mosse, ciascuna corrispondente a uno dei comportamenti di $S_N$ leggendo $b$:
\begin{itemize}
	\item per simulare le transizioni che non cambiano stato, $C_N$ sovrascrive $b$ con un simbolo $\neutr$ neutro, nel senso che il comportamento della macchina in esso sarà semplicemente di procedere senza tenerlo in considerazione;
	\item per simulare le transizioni di reset, $C_N$ sovrascrive $b$ con il simbolo di reset $\reset$. Si noti che sia in questo caso sia se il ciclo di $\set{a^N}\star$ viene completato il simbolo $\reset$ corrisponde agli istanti in cui $S_N$ passa allo stato $q_0$.
\end{itemize}
Se si incontra $b$ a destra di $\lem$ o di $\reset$, la mossa non è definita, così come non lo è in $q_0$ nell'automa $S_N$. $C_N$ accetta se e solo se $\reset$ o $\lem$ sono seguiti da $\rem$. L'incremento di stati e di simboli rispetto a $B_N$ è trascurabile ed entrambi rimangono in numero di $O(\log N)$.

In conclusione, visti i risultati precedenti, il linguaggio $\generated{S_N}$ è testimone del gap esponenziale tra \la1 e 1NFA e doppiamente esponenziale tra \la1 e 1DFA.


\subsection{L'automa \texorpdfstring{$R_N$}{R con N}}
Un automa che presenta un meccanismo di reset simile a quello di Meyer e Fischer è stato presentato da Moore in \cite{Moore:71:automatabounds}. Il meccanismo dell'automa $R_N$, rappresentato in figura \ref{img:wit:Rn}, è tuttavia leggermente diverso: questa volta le transizioni di reset possono avvenire solo dallo stato $q_N$, verso lo stato $q_1$ o verso lo stato $q_2$. Queste transizioni sono, tra l'altro, le uniche nondeterministiche della macchina.

\begin{figure}
	\centering
	\begin{tikzpicture}[shorten >=1pt,initial text=]
	\node[state,initial]	(q1)			{$q_1$};
	\node[state]		(q2)	[right=of q1]	{$q_2$};
	\node[state]		(q3)	[right=of q2]	{$q_3$};
	\node			(qd)	[right=of q3]	{$\dots$};
	\node[state]		(qn-1)	[right=of qd]	{\small $q_{n-1}$};
	\node[state]		(qn)	[right=of qn-1]	{$q_n$};
	\path[->]
	(q1)	edge			node[above]	{$a$} (q2)
		edge[loop above]	node[above] 	{$b$} ()
	(q2)	edge			node[above]	{$a,b$} (q3)
	(q3)	edge			node[above]	{$a,b$} (qd)
	(qd)	edge			node[above]	{$a,b$} (qn-1)
	(qn-1)	edge			node[above]	{$a,b$} (qn)
	(qn)	edge[bend angle=25,bend left]		node[below]	{$a$} (q2)
	(qn)	edge[bend angle=35,bend left]		node[below]	{$a$} (q1);
\end{tikzpicture}

	\caption{L'NFA $R_N$ di Moore.}
	\label{img:wit:Rn}
\end{figure}

Un \la1 $D_N$ può riconoscere l'automa $S_N$ con un adattamento della tecnica precedente.
\begin{itemize}
	\item Simulando la computazione tra $q_1$ e $q_N$, $D_N$ sovrascrive l'input con la binary carry sequence trovando, dopo ogni scrittura, il più piccolo naturale che non ricorre nella backward increasing sequence del prefisso corrente e, contemporaneamente, controllando se tale sequenza rappresenta il numero $N$. La macchina verifica che la successione dei simboli di input porti a transizioni valide (ossia non può leggere $b$ nello stato $q_N$).
	\item $D_N$ sceglie nondeterministicamente quale transizione simulare nel caso stia simulando lo stato $q_N$ e legga $a$ in input.
	\item Per simulare la transizione dallo stato $q_N$ verso lo stato $q_1$, $D_N$ scrive un simbolo di reset $\reset_1$ su cui la macchina si comporta come sull'end-marker sinistro, e dopo il quale ricomincia a scrivere la binary carry sequence a partire dal suo primo elemento. Lo stesso simbolo di reset viene scritto se simulando lo stato $q_1$, cioè dopo la scrittura di $\reset_1$ oppure dopo $\lem$, il simbolo di input successivo è $b$, simulando la transizione dallo stato $q_1$ in se stesso.
	\item Per simulare la transizione dallo stato $q_N$ verso lo stato $q_2$, $D_N$ scrive un simbolo di reset $\reset_2$. Su questo simbolo la macchina si comporta in modo leggermente diverso rispetto a $\reset_1$ e $\lem$, in quanto muovendosi alla sua destra l'automa scrive la binary carry sequence partendo dal suo secondo elemento (invece che dal primo). Questo simbolo porta l'automa a comportarsi diversamente anche nelle computazioni successive: leggendolo durante il tracciamento della backward increasing sequence, l'automa si comporta come se avesse letto $\reset_1$ e prima di esso (cioè alla sua destra) il primo elemento della binary carry sequence. In questo modo il conteggio dei simboli con la tecnica della binary carry sequence rimane valido, pur contando a partire dal suo secondo elemento.
\end{itemize}
L'automa $D_N$ può essere implementato con $O(\log N)$ stati e $O(\log N)$ simboli.

Moore ha dimostrato che $R_N$, oltre a essere minimo, riconosce un linguaggio che è testimone del costo esponenziale tra 1NFA a 1DFA. L'esistenza di un \la1 di descrizione logaritmica che riconosce $\generated{R_N}$ dimostra che tale linguaggio è testimone anche del costo esponenziale tra \la1 e 1NFA e doppiamente esponenziale tra \la1 e 1DFA.

\chapter{Altri risultati e problemi aperti}\label{cha:prob}
In questo capitolo concludiamo riassumendo i problemi aperti che riguardano gli automi $1$-limited e presentandone qualche ulteriore risultato. Diamo poi uno sguardo ai risultati relativi ad altre varianti degli automi limited.



\section{Problemi aperti}
Lo studio dei $1$-limited e del loro rapporto con altri riconoscitori, pur avendo portato a risultati soddisfacenti nei casi principali (1NFA, 1DFA), lascia diverse domande senza risposta.


\subsection{L'eliminazione del nondeterminismo}
Il più importante dei problemi aperti che riguardano i $1$-limited è il gap di complessità descrizionale tra \la1 e D\la1, di cui conosciamo un lower bound esponenziale (corollario \ref{cor:a1l:LAtoDLA}) e un upper bound doppiamente esponenziale, derivante dalla simulazione degli \la1 da parte dei 1DFA. Non abbiamo, tra l'altro, evidenze di una distanza maggiore tra \la1 e 2DFA, nonostante le restrizioni ulteriori di quest'ultimo modello rispetto ai D\la1.

Pighizzini, Prigioniero e Sádovský \cite{Pighizzini:22:limitedwitness} hanno proposto, con l'intenzione di introdurre dei witness language per una ipotetica distanza più che esponenziale tra \la1 e D\la1 (o tra \la1 e 2DFA), due linguaggi binari basati sulla parità (XOR). Il linguaggio $P_n$ è un linguaggio a blocchi, in cui lo XOR bit a bit di un certo numero di blocchi risulta nell'ultimo blocco. Il linguaggio $P'_n$ è una versione di $P_n$ in cui cade il vincolo dei blocchi, permettendo lo XOR di qualunque sottostringa che non si sovrapponga:
\begin{align*}
	P_n := \{  x_1\dots x_kx \mid ~ & k>0, x_1,\dots,x_k,x\in\{0,1\}^n,                                \\
	                                & \exists h>0,i_1,i_2,\dots,x_h\in\{1,\dots,k\},i_1<i_2<\dots<i_h: \\
	                                & x=x_{i_1}\oplus x_{i_2}\oplus\dots\oplus x_{i_h}\}
\end{align*}
\begin{align*}
	P'_n := \{  wx \mid ~ & w\in\{0,1\}^*,x\in\{0,1\}^n,                                                \\
	                      & \exists h>0,x_1,x_2,\dots,x_h\in\{0,1\}^n,y_0,y_1,\dots,y_h\in\{0,1\}^*:    \\
	                      & w=y_0x_1y_1\dots y_{n-1}x_hy_h \land x=x_1\oplus x_2\oplus\dots\oplus x_h\}
\end{align*}

Questi linguaggi possono essere riconosciuti da un \la1 di dimensione lineare in $n$ con un adattamento delle tecniche per i linguaggi a blocchi (paragrafo \ref{sec:wit:blk}), tuttavia è sconosciuto un lower bound per un 1DFA equivalente. Rimane una congettura, per il momento, che una dimensione esponenziale non sia sufficiente per un D\la1 o 2DFA per riconoscere questi linguaggi.


\subsection{Simulazioni mancanti}
Non si conoscono simulazioni dirette che permettano, a partire da un \la1, di rimuovere il nondeterminismo (\la1\tto D\la1), la capacità di riscrittura (\la1\tto 2NFA), o la combinazione delle due (\la1\tto 2DFA).

In effetti, non si conosce una costruzione che permetta di convertire un 2NFA direttamente in un 2DFA (è il problema aperto posto da Sakoda e Sipser in \cite{Sakoda:78:sizetwoway}), anche se di una potenziale simulazione si conosce un lower bound polinomiale e un upper bound esponenziale. Pur non risolvendo questo problema, sarebbe interessante capire se vi sia una relazione con la simulazione di \la1 da parte di D\la1. In particolare, l'eliminazione del nondeterminismo da una macchina two-way potrebbe sfruttare una tecnica adattabile al caso \la1\tto D\la1.

Per quanto riguarda l'eliminazione della capacità di riscrittura, ci sembra improbabile trovare una tecnica che permetta di codificare nel solo movimento two-way, anche se nondeterministico, i possibili contenuti del nastro modificato.

I risultati relativi alla complessità descrizionale dei vari riconoscitori di linguaggi regolari sono riassunti nella figura \ref{img:pro:sim}.

\begin{figure}
	\centering
	\begin{tikzpicture}
	\footnotesize
	\newcommand{\bounds}[4]{$\geq$ #1 \ref{itm:pro:#2}\\ $\leq$ #3 \ref{itm:pro:#4}}
	\newcommand{\boundsq}[3]{\bounds{#1}{#3:l}{#2}{#3:u}}

	\node (la)	{\la1};
	\node (dla)	[below=5cm of la] {D\la1};
	\node (nfa)	[below left=3cm of la] {1NFA/2NFA};
	\node (2dfa)	[right=6cm of nfa] {2DFA};
	\node (1dfa)	[right=2cm of 2dfa] {1DFA};

	\path[-latex]
	(la)  edge node[above left,align=left] {\boundsq{exp}{exp}{lanfa}} (nfa.north)
	(la) edge node[above right,align=left] {\boundsq{2exp}{2exp}{lanfa}} (1dfa.north)
	(la)  edge node[left,align=left] {\bounds{exp}{ladla:l}{2exp}{ladfa:u}} (dla.north)
	(la)  edge node[below left,align=left] {\bounds{exp}{lanfa:l}{2exp}{ladfa:u}} (2dfa.north)
	(dla) edge node[below left,align=left] {\bounds{exp}{dlanfa:l}{2exp}{lanfa:u}} (nfa.south)
	(dla)  edge node[below right,align=left] {\bounds{exp}{dlanfa:l}{2exp}{ladfa:u}} (1dfa.south)
	(dla) edge node[above left,align=left] {\bounds{exp}{dlanfa:l}{2exp}{ladfa:u}} (2dfa.south);
\end{tikzpicture}

	\captionsetup{singlelinecheck=off}
	\caption[]{Upper e lower bound di conversione tra $1$-limited e altri riconoscitori di linguaggi regolari.\label{img:pro:sim}
		\begin{enumerate}[(a)]
			\item \label{itm:pro:lanfa:u} simulazione \la1\tto 1NFA (paragrafo \ref{subs:a1l:up})
			\item \label{itm:pro:lanfa:l} witness language tra cui $L_n$ (paragrafo \ref{subs:a1l:low})
			\item \label{itm:pro:ladfa:u} simulazione \la1\tto 1DFA (paragrafo \ref{subs:a1l:up});
			\item \label{itm:pro:ladla:l} corollario \ref{cor:a1l:LAtoDLA} da witness language $L_n$;
			\item \label{itm:pro:dlanfa:l} witness language tra cui l'unario $\set{a^{2^n}}$ (paragrafo \ref{sec:wit:un}).
		\end{enumerate}
	}
\end{figure}



\subsection{Il caso unario}
Per quanto riguarda i linguaggi unari, lo sviluppo della tecnica basata sulla binary carry sequence (paragrafo \ref{sec:wit:un}) ha portato all'individuazione di testimoni della distanza almeno esponenziale tra \la1 e 1NFA o 2NFA, superando i precedenti risultati di lower bound quadratico \cite{Pighizzini:14:limitedRE}. Non si conosce una simulazione che limiti la distanza nel caso unario a semplicemente esponenziale, né sono stati trovati lower bound maggiori, pertanto il gap tra \la1 e 1DFA per gli unari rimane un problema aperto, la cui risposta è inclusa tra il singolo e il doppio esponenziale.



\section{Altri risultati sugli \texorpdfstring{$1$-limited}{1-limited}}


\subsection{Grammatiche}
Nei precedenti capitoli ci siamo concentrati sugli aspetti degli $1$-limited che li mettono in relazione con altri riconoscitori di linguaggi regolari; tuttavia sono noti anche risultati relativi al loro rapporto con grammatiche. Nel caso unario in particolare, in cui le grammatiche di tipo $2$ descrivono linguaggi regolari, Pighizzini e Prigioniero hanno dimostrato \cite{Pighizzini:19:limitedunary} che è possibile convertire una grammatica context-free in un \la1 che riconosce lo stesso linguaggio, con un incremento polinomiale della dimensione della descrizione.

Il risultato è stato esteso al caso generale tramite la nozione di Parikh-equivalenza. Due linguaggi (o due loro generatori o riconoscitori) si dicono Parikh-equivalenti se per ogni parola del primo esiste una parola nel secondo uguale a meno dell'ordine dei simboli, o alternativamente con lo stesso numero di occorrenze per ogni simbolo. Il teorema di Parikh dimostra che ogni linguaggio libero dal contesto ha un Parikh-equivalente regolare \cite{Parikh:66:CF}. Nell'articolo sopracitato viene dimostrato che è possibile convertire una grammatica context-free qualsiasi in un \la1 che riconosca un linguaggio regolare Parikh-equivalente, con un incremento polinomiale nella dimensione della grammatica.


\subsection{Complessità in tempo}
Abbiamo dimostrato che l'utilizzo di un \la1 per riconoscere un linguaggio regolare può evitare l'utilizzo di un 1NFA di complessità descrizionale esponenziale, o un 1DFA doppiamente esponenziale . Questo risparmio, tuttavia, va a pari passo con una potenziale perdita nella complessità in tempo. Infatti, mentre un 1NFA o 1DFA effettua una transizione per carattere di input, effettuando una computazione in un tempo lineare nella dimensione dell'input, un \la1, in quanto macchina two-way, può ripercorrere l'input più volte in una computazione che supera il lineare. Guillon e Prigioniero hanno dimostrato in \cite{Guillon:19:linearlimited} che dato un qualsiasi \la1 è possibile costruire un \la1 equivalente che lavori in tempo lineare nella dimensione dell'input, con un incremento polinomiale nella complessità descrizionale e preservando il determinismo.



\section{Altri automi limited}
Lo studio degli automi limited in letteratura va ben oltre il caso degli $1$-limited, trattando sia il caso $d>1$ sia varianti del modello, con particolare attenzione alla complessità descrizionale. In \cite{Pighizzini:19:limited}, Pighizzini presenta la maggior parte dei risultati conseguiti su questa famiglia di riconoscitori, tra cui riportiamo qui alcuni dei più interessanti.


\subsection{\texorpdfstring{$d$-limited con $d>1$}{d-limited con d>1}}
Come già detto, Hibbard \cite{Hibbard:67:CFdet} ha dimostrato che i $d$-limited nondeterministici caratterizzano i linguaggi context-free se $d\geq2$. Nel caso deterministico, esiste una gerarchia stretta in cui i D\la2 caratterizzano i linguaggi liberi dal contesto deterministici e per ogni $d\geq2$ esiste un linguaggio riconosciuto da un D\la{(d+1)} non riconosciuto da alcun D\la d. Nonostante ciò, esistono linguaggi context-free che non possono essere riconosciuti da D\la d, per alcun $d$.

Pighizzini e Pisoni hanno dimostrato in \cite{Pighizzini:14:limitedCF} che la trasformazione da \la2 ad automa a pila è esponenziale, mentre la trasformazione inversa (che non è banale come nel caso 1DFA\tto\la1) è polinomiale. Kutrib, Pighizzini e Wendlandt hanno dimostrato che per $d>2$ rimane un upper bound esponenziale alla conversione in PDA \cite{Kutrib:18:complexlimited}.


\subsection{Varianti di automi limited}
In \cite{Wechsung:79:complexities} Wechsung ha studiato una variante di automa limited in cui, invece di un $d$ costante, la limitazione all'ultima visita con capacità di scrittura dipende da una funzione della lunghezza dell'input.

Più recentemente è stato introdotto il modello degli \eng{strongly limited automata}, che possiedono ulteriori restrizioni ispirate agli automi limited che riconoscono linguaggi di Dyck (sequenze di parentesi correttamente bilanciate). Pighizzini ha dimostrato in \cite{Pighizzini:16:stronglylimited} che questo modello ha dimensione al più polinomiale nella dimensione di una grammatica context-free o di un PDA equivalenti, in contrapposizione con le conversioni esponenziali ottenute nel caso degli automi $d$-limited.

Numerosi altri modelli e varianti sono stati studiati, tra cui automi limited con capacità di scrittura nelle \emph{ultime} $d$ visite \cite{Wechsung:79:complexities} e anche estensioni probabilistiche \cite{Yamakami:19:limitedmodels}.


\chapter*{Ringraziamenti}
Vorrei ringraziare Luca Prigioniero e Giovanni Pighizzini per essere stati ottimi relatori quanto ricercatori. In particolare ringrazio il dottor Prigioniero per la sua grande competenza e disponibilità, nel correggere le mie bozze e nel rispondere alle mie infinite domande, premiando la mia curiosità.
Ringrazio la mia famiglia, i miei amici e in generale chi mi ha supportato nella mia vita e nella mia carriera universitaria. Il supporto degli altri è un dono che non bisogna dare per scontato.
Ringrazio i professori che si contraddistinguono nell'insegnare trasmettendo la propria passione agli studenti. Sono loro che plasmano il cuore dell'università e di chi la vive.
Infine ringrazio tutti i ricercatori, che hanno il rarissimo dono di guardare al futuro.


\printbibliography[heading=bibintoc]

\end{document}
