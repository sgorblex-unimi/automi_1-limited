\documentclass[a4paper,twoside]{thesis}
% \documentclass[a4paper]{thesis}
% \documentclass[a4paper,compact]{thesis}	% compact is used for drafts

\usepackage{Automi_1-Limited}

\begin{document}
\title{Automi 1-Limited}
\author{Alessandro Clerici Lorenzini}
\matr{941784}
\logo{img/unimi_logo}
\university{Università degli Studi di Milano}
\dept{Dipartimento di Informatica}
\degr{Corso di Laurea in Informatica}
\superv{Prof. Giovanni Pighizzini}
\cosuperv{Dr. Luca Prigioniero}
\date{Anno Accademico 2021/2022}

\pagestyle{plain}
\maketitle
\thispagestyle{empty}
\cleardoublepage
\pagenumbering{roman}
\tableofcontents
\preface

\section*{Storia e motivazioni}
Lo studio moderno degli automi limited nasce dall'unione di due rami dell'informatica teorica: la classificazione dei linguaggi formali e la complessità descrizionale.

I modelli computazionali con limitazioni sono un classico oggetto di studio dell'informatica teorica. Gli automi limited sono stati introdotti da Hibbard nel 1967 (\cite{Hibbard:67:CFdet}) come strumento per costruire una gerarchia che caratterizzasse il determinismo all'interno della classe dei linguaggi liberi dal contesto.

La complessità descrizionale, che ha avuto origine con l'articolo fondatore di Meyer e Fischer (\cite{Meyer:71:ecodescription}), studia quanto una descrizione di un linguaggio formale, ad esempio una grammatica o un riconoscitore, può essere succinta, e il rapporto tra le dimensioni di diverse descrizioni equivalenti. Il classico esempio è la costruzione per sottoinsiemi di Rabin e Scott (\cite{Rabin:59:NFA}) che permette a un automa a stati finiti deterministico di simularne uno nondeterministico pagando un costo esponenziale nel numero di stati.

Un automa $d$-limited è una macchina di Turing nondeterministica con alcune limitazioni: lo spazio di lavoro è limitato alle celle che inizialmente contengono l'input (delimitate da $\lem$ e $\rem$) e la scrittura di una cella è possibile solo durante le sue prime $d$ visite. Sebbene Hibbard abbia dimostrato che gli automi $d$-limited nondeterministici caratterizzano i linguaggi liberi dal contesto quando $d>1$, gli automi $0$-limited (cioè con $d=0$) corrispondono alla definizione di automi a stati finiti nondeterministici \eng{two-way}, riconoscitori di linguaggi regolari. Wagner e Wechsung (\cite{Wagner:86:compCompl}) hanno dimostrato che permettendo la scrittura alla prima visita di una cella la classe riconosciuta è la stessa, ossia gli automi $1$-limited caratterizzano i linguaggi regolari. Pighizzini e Pisoni hanno ottenuto in \cite{Pighizzini:14:limitedRE} l'upper bound del numero di stati necessari a un automa a stati finiti per simulare un automa $1$-limited di $n$ stati, ottenendo un numero di stati esponenziale in $n$ per gli automi a stati finiti nondeterministici e doppiamente esponenziale per i deterministici. Lo studio dei lower bound tramite \eng{witness languages}, iniziato nello stesso articolo, ha dimostrato che per qualunque $n$ esistono casi in cui tale costo non può essere evitato. In altre parole, gli automi $1$-limited possono fornire descrizioni molto più compatte rispetto ai riconoscitori standard per la classe dei regolari. Negli anni successivi sono stati studiati altri linguaggi testimoni per i bound che riguardano gli $1$-limited (ad esempio \cite{Pighizzini:22:limitedwitness}), ma anche altre varianti di automi limited, con risultati interessanti specialmente in relazione alla complessità descrizionale.

\section*{In questa tesi}
Nel capitolo \ref{cha:prel} riassumiamo le conoscenze che consistono in prerequisiti per la comprensione del cuore della tesi, toccando le principali definizioni della teoria dei linguaggi formali. Il capitolo \ref{cha:a1l} introduce il modello degli automi $d$-limited e degli $1$-limited in particolare, quindi ne studia la potenza computazionale e descrizionale, presentando le principali simulazioni da parte degli altri riconoscitori di linguaggi regolari e dimostrando l'ottimalità dei rispettivi costi tramite un linguaggio a blocchi testimone del lower bound. Approfondiamo nel capitolo \ref{cha:wit} upper e lower bound di alcuni witness languages, studiando alcune tecniche applicabili agli $1$-limited per riconoscere famiglie di linguaggi con caratteristiche comuni: linguaggi a blocchi, unari (cioè costruiti su un alfabeto di un solo simbolo) o derivanti da automi con meccanismi di reset. Infine, il capitolo \ref{cha:prob} riassume i risultati sulla materia, evidenziandone i problemi ancora aperti e i le possibili ricerche future, e parla brevemente degli altri risultati che riguardano gli automi limited e le loro varianti.

\pagenumbering{arabic}
\pagestyle{headings}

\chapter{Nozioni preliminari}
La teoria dei linguaggi formali è un campo fondamentale dell'informatica teorica e studia i linguaggi, cioè insiemi di parole, e i loro generatori o riconoscitori. Questo campo, sebbene a prima vista appaia strettamente legato alla linguistica, ha implicazioni enormi nell'informatica e nella matematica, per esempio nei campi della computabilità, della programmazione, della crittografia e della logica. In questo capitolo ricordiamo al lettore le nozioni fondamentali della teoria dei linguaggi, delle grammatiche e dei principali riconoscitori, di cui studieremo il rapporto con gli automi \eng{$1$-limited}.


\subsection*{Convenzioni di notazione}
Dato un insieme $S$, indicheremo con $\card{S}$ la sua cardinalità e con $\subsets{S}$ l'insieme dei suoi sottoinsiemi.
% TODO: aggiungere qui eventuali altre notazioni
La restante notazione è descritta nelle sezioni seguenti.


\section{Linguaggi}
In questa sezione introduciamo le definizioni fondamentali della teoria dei linguaggi, inclusi alfabeti, parole e linguaggi stessi. La maggior parte dei concetti è espressa seguendo la notazione usata in \cite{Hopcroft:79:introLFA}, che invitiamo il lettore a consultare per approfondimenti.

\paragraph{Alfabeti} Un \emph{alfabeto} è un insieme finito e non vuoto arbitrario, i cui elementi sono detti \emph{simboli}. Indicheremo alfabeti mediante simboli come $\Sigma$, $\Gamma$ o esplicitando direttamente l'insieme ($\set{a,b}$). I simboli di un alfabeto sono spesso indicati con lettere minuscole o cifre numeriche, tuttavia può essere utile usare altri simboli quando la semantica ne viene semplificata. Chiamiamo unario un alfabeto di esattamente un elemento.

\paragraph{Parole} Una \emph{parola} (o \emph{stringa}) $w$ su un alfabeto $\Sigma$ è una sequenza finita di simboli appartenenti a $\Sigma$: $w:=x_1 x_2 \cdots x_n$ con $x_1,x_2,\dots,x_n\in\Sigma$. Le parole si indicano comunemente con lettere latine minuscole. La parola non contenente simboli è detta parola vuota e si indica con $\emptyword$. La lunghezza di una parola $w$ è il numero di simboli che la compongono e viene indicata con $\len w$. Si indica con $\Sigma^n$, con $n\in\N$, l'insieme di parole di lunghezza $n$ su $\Sigma$, con $\Sigma^0:=\set{\emptyword}$ per qualunque $\Sigma$. Spesso si usa semplicemente $\Sigma$ per indicare l'insieme $\Sigma^1$ delle parole di lunghezza $1$ su $\Sigma$. Si indica con $\Sigma\star$ l'insieme di tutte le parole sull'alfabeto $\Sigma$, cioè l'unione di tutte le sue potenze, e con $\Sigma^+$ l'insieme delle parole non vuote su $\Sigma$: $\Sigma^+ := \Sigma\star\setminus\set{\emptyword}$. Date due parole $v=x_1\cdots x_n$ e $w=y_1\cdots y_m$ si dice concatenazione di $v$ e $w$ la parola $vw:=x_1\cdots x_n y_1\cdots y_m$ composta dai simboli di $v$ seguiti da quelli di $w$. Si noti che $\len{vw}=\len v+\len w$.

\paragraph{Linguaggi} Un linguaggio $L$ su un alfabeto $\Sigma$ è un insieme di parole su $\Sigma$, ossia $L\subseteq\Sigma\star$ (le potenze di un alfabeto sono dunque linguaggi). Il simbolo $\emptyset$ indica il linguaggio vuoto, da non confondere con il linguaggio $\set{\emptyword}$.  A volte omettiamo le parentesi graffe per indicare un linguaggio singoletto ($a$ al posto di $\set{a}$). Chiamiamo unario un linguaggio su un alfabeto unario. Il prodotto di due linguaggi $L_1$ e $L_2$ è il linguaggio in cui ogni parola è la concatenazione di una parola di $L_1$ e una di $L_2$:
\begin{equation*}
	L_1\cdot L_2 := \set{xy\mid x\in L_1 \land y\in L_2}
\end{equation*}
Dato un linguaggio $L$, viene indicato con $L^0$ il linguaggio $\set{\emptyword}$, e con $L^n$ il prodotto di $L$ con se stesso $n$ volte. La chiusura (di Kleene) di un linguaggio $L$ è il linguaggio $L\star$ delle concatenazioni di un numero arbitrario di parole di $L$:
\begin{equation*}
	L\star := L^0\cup L^1\cup\dots=\bigcup_{k=0}^\infty L^k
\end{equation*}
Si indica inoltre con $L^+$ il linguaggio $\bigcup_{k=1}^\infty L^k$. Si noti che $L^+=L\star$ se e solo se $\emptyword\in L$.


\subsection*{Riconoscitori e generatori}
Esistono diversi modi di rappresentare un linguaggio $L$: se è finito è sufficiente elencarne le parole ($L:=\set{w_1,w_2,\dots,w_n}$), se è infinito e le sue parole possiedono una proprietà caratterizzante $P$, può essere descritto da essa ($L:=\set{w\mid P(w)}$). Tuttavia, non sempre è facile o rappresentativo usare una di queste rappresentazioni. I metodi generativo e riconoscitivo forniscono un ulteriore modo di descrivere i linguaggi.

% TODO: aggiungere un riferimento bibliografico che permetta al lettore di approfondire il concetto di sistema formale (e, volendo, calcolo logico)
\begin{description}
	\item[Generatori] un generatore per un linguaggio $L\subseteq\Sigma\star$ è un sistema formale che produce parole appartenenti a $L$, cioè un metodo per costruirle tramite regole.
	\item[Riconoscitori] un riconoscitore per un linguaggio $L\subseteq\Sigma\star$ è un algoritmo o una procedura che determina se una data parola $w\in\Sigma\star$ appartiene a $L$.
\end{description}

\noindent Dato un riconoscitore o generatore $M$, si indica con $\generated M$ il linguaggio riconosciuto o generato da $M$.



\section{Grammatiche}
Le \emph{grammatiche} sono il principale generatore studiato nella teoria dei linguaggi formali. Una grammatica su un alfabeto $\Sigma$ consiste in un insieme di regole che permettono di costruire le parole di un linguaggio trasformando in più passi un \emph{assioma} di partenza. Queste trasformazioni seguono delle regole, dette di produzione, nella forma $\alpha\to\beta$, che permettono di trasformare una parola $u\alpha v$ in $u\beta v$. Le parole derivate applicando le regole di produzione possono contenere simboli detti \emph{nonterminali} da un alfabeto ausiliario (a cui appartiene l'assioma), ma solo quelle composte esclusivamente da simboli \emph{terminali}, cioè appartenenti a $\Sigma$, fanno parte del linguaggio generato.


\subsection{La Classificazione di Chomsky}\label{subs:prel:chom}
La gerarchia Chomsky (\cite{Chomsky:56:hier}) è una classificazione delle grammatiche, basata sulla forma delle loro regole di produzione, da cui deriva una gerarchia di linguaggi fondamentale nella teoria dei linguaggi formali. La gerarchia si compone di quattro classi:
\begin{description}
	\item[Tipo 0] tutte le grammatiche sono di \emph{tipo 0};
	\item[Tipo 1] in una grammatica di tipo 1, ogni regola di produzione è nella forma $\alpha A\beta\to\alpha\gamma\beta$, dove $\gamma$ è non vuota e $A$ è un simbolo nonterminale. È ammessa la regola $S\to\emptyword$, se $S$ è l'assioma, solo se $S$ non compare nella parte destra di alcuna altra regola. I linguaggi che possono essere generati da grammatiche di tipo 1 sono detti \emph{dipendenti dal contesto} (\eng{context-sensitive});
	\item[Tipo 2] in una grammatica di tipo 2, ogni regola di produzione $\alpha\to\beta$ è tale che $\alpha$ è un simbolo nonterminale. Valgono le stesse restrizioni sull'assioma delle grammatiche di tipo 1. I linguaggi che possono essere generati da grammatiche di tipo 2 sono detti \emph{liberi dal contesto} (\eng{context-free});
	\item[Tipo 3] in una grammatica di tipo 3, ogni regola di produzione è in una delle forme $A\to\sigma B$, $A\to\sigma$, $A\to\emptyword$, dove $A$ e $B$ sono simboli nonterminali e $\sigma$ è un simbolo terminale. I linguaggi che possono essere generati da grammatiche di tipo 3 sono detti \emph{regolari}.
\end{description}
Esiste un'inclusione tra le classi di grammatiche di tipo più alto e quelle di tipo più basso, da cui deriva un'inclusione (propria) tra le rispettive classi di linguaggi. Come vedremo, queste classi corrispondono inoltre a classi di riconoscitori.



\section{Riconoscitori}
I riconoscitori sono modelli matematici che, data una parola in input, determinano se questa appartiene al linguaggio (accettazione) o non vi appartiene (rifiuto). Formalmente, data una macchina $M$ che lavora su un alfabeto $\Sigma$:
\begin{equation*}
	\generated M := \set{w\in\Sigma\star\mid M\text{ accetta }w}
\end{equation*}
Per questo motivo di ogni modello si definisce il concetto di accettazione.
Per approfondimenti, dimostrazioni e altri riconoscitori suggeriamo la lettura di \cite{Hopcroft:79:introLFA} e \cite{Shallit:09:secondLFA}.


\subsection{Macchine di Turing}
Una macchina di Turing si compone di un nastro, una testina e un controllo finito. Il nastro è infinito a destra ed è diviso in celle, le prime celle contenenti l'input (un simbolo per cella), le seguenti il simbolo vuoto. La testina punta a una cella del nastro (visita) ed è in grado di leggere e scrivere su di essa, scegliendo da un predeterminato alfabeto. Il controllo finito consiste in un insieme finito di stati di cui uno corrente. L'evoluzione della computazione di una macchina di Turing procede per istanti successivi, secondo una legge detta \emph{funzione di transizione}. Precisamente, la macchina, in funzione dello stato corrente e del simbolo letto dalla testina, sovrascrive il simbolo nella cella con un altro, cambia stato, e si muove in una direzione (sinistra o destra) in una cella adiacente. Lo stato corrente e i simboli sul nastro (più implicitamente la posizione della testina) consistono nell'unica memoria della macchina.

\begin{defin}[macchina di Turing]
	Una \emph{macchina di Turing} (TM) è una settupla $M:=\tuple{Q,\Sigma,\Gamma,\blank,\delta,q_0,F}$, dove:
	\begin{itemize}
		\item $Q$ è un insieme finito e non vuoto di stati;
		\item $\Sigma$ è l'alfabeto di input, cioè dei simboli che si possono trovare sul nastro all'inizio della computazione (insieme a $\blank$);
		\item $\Gamma\supseteq\Sigma\cup\set{\blank}$ è un alfabeto di simboli per il nastro;
		\item $\blank\in\Gamma$ è il simbolo vuoto (\emph{blank}), che ricorre in ogni cella del nastro a destra dell'input;
		\item $\delta:Q\times\Gamma\to Q\times\Gamma\times\set{\Left,\Right}$ è una funzione parziale detta di transizione. Se a un dato passo la macchina è nello stato $p$, la cella puntata dalla testina contiene $\sigma$, e $\delta(p,\sigma)=(q,\gamma,D)$, allora:
		      \begin{itemize}
			      \item $q$ è il prossimo stato;
			      \item $\gamma$ è il simbolo che verrà scritto in sostituzione a $\sigma$ nella cella corrente, allo spostamento della testina;
			      \item $D$ è la direzione in cui si muoverà la testina, $\Left$ se a sinistra e $\Right$ se a destra.
		      \end{itemize}
		\item $q_0\in Q$ è lo stato iniziale;
		\item $F\subseteq Q$ è l'insieme degli stati finali (o accettanti).
	\end{itemize}
	Una macchina di Turing accetta una parola $w\in\Sigma\star$ se e solo se la sua computazione, a partire dallo stato $q_0$ e con la testina sul primo simbolo di $w$, termina in uno stato finale.
\end{defin}
\noindent Sia l'arresto in uno stato non finale sia il non arresto sono considerati rifiuti.

\begin{figure}
	\centering
	\begin{tikzpicture}[cell/.style={minimum height=1.5em,minimum width=1.5em,outer sep=0pt,rectangle,draw,node distance=0pt}]
	\node[cell] (first) {$\sigma_0$};
	\node[cell] (pointed) [right=of first] {$\sigma_1$};
	\node[cell] (third) [right=of pointed] {$\sigma_2$};
	\node[cell, minimum width=2.5em] (worddots) [right=of third] {$\dots$};
	\node[cell] (last) [right=of worddots] {$\sigma_n$};
	\node[cell] (b1) [right=of last] {$b$};
	\node[cell] (b2) [right=of b1] {$b$};
	\node[node distance=0pt] (dots) [right=0.4 cm of b2] {$\dots$};
	\node[cell] (control) [above=0.75cm of pointed,thick] {$q$};
	\draw[-latex] (control) -- (pointed);
	\draw (b2.north east) -- ++(1.5cm,0) (b2.south east) -- ++ (1.5cm,0);
\end{tikzpicture}

	\caption{Rappresentazione di una macchina di Turing di esempio in un dato istante.}
\end{figure}

\subsubsection{Nondeterminismo}
È fondamentale citare il modello nondeterministico delle macchine di Turing e, in generale, delle macchine riconoscitrici. Una macchina di Turing nondeterministica (NTM) ha per funzione di transizione una relazione del tipo $\delta:Q\times\Gamma\to \subsets{Q\times\Gamma\times\set{\Left,\Right}}$. Un modello di questo tipo descrive multiple possibilità per un passo dell'evoluzione della macchina, ciascuna descritta da uno degli elementi di un'immagine della funzione. Una parola in input è accettata se esiste una computazione, tra quelle coerenti con la funzione di transizione, che termina in uno stato accettante.
In generale, una macchina riconoscitrice si dice nondeterministica quando la sua funzione di transizione fornisce più possibilità per un passo evolutivo, e l'accettazione corrisponde all'esistenza di una computazione che termini accettando.

Poiché per ogni TM nondeterministica è possibile costruire una TM deterministica che riconosce lo stesso linguaggio e viceversa, i due modelli riconoscono la stessa classe di linguaggi. La classe dei linguaggi accettati da macchine di Turing è quella dei linguaggi ricorsivamente enumerabili, che equivale alla classe dei linguaggi generabili da grammatiche di tipo 0 della classificazione di Chomsky (paragrafo \ref{subs:prel:chom}).

% TODO: aggiungere estensione con limitazione da una funzione lineare nell'input? Serve fonte
\subsubsection{Automi limitati linearmente}
Gli \emph{automi linearmente limitati} (LBA) sono macchine di Turing in cui la lunghezza del nastro, invece che essere infinita, è limitata dalla lunghezza dell'input. La classe dei linguaggi accettati da automi linearmente limitati è quella dei linguaggi dipendenti da contesto.


\subsection{Automi a pila}\label{subs:prel:PDA}
Un \emph{automa a pila} (PDA) è una macchina nondeterministica composta da un controllo finito, un nastro in sola lettura e una pila. Una pila è una struttura dati che permettere di leggere e scrivere in maniera LIFO (\eng{Last In, First Out}). A ogni passo l'automa può leggere un simbolo dell'input, procedendo da sinistra verso destra, oppure non leggere nulla ($\emptyword$-mossa). In funzione del simbolo letto sul nastro e di quello sulla cima della pila, l'automa cambia stato e sostituisce il simbolo in cima alla pila con una parola. Un PDA accetta una parola $w\in\Sigma\star$ se e solo se esiste una computazione che, a partire dallo stato iniziale, con la pila contenente il simbolo iniziale e con la lettura del primo simbolo di $w$, termina la lettura dell'input in uno stato finale. Si può definire in alternativa l'accettazione per pila vuota, cioè terminando la lettura dell'input con la pila non contenente simboli, che risulta equivalente nella potenza riconoscitiva.

Come le macchine di Turing, gli automi a pila hanno una controparte deterministica: gli \emph{automi a pila deterministici} (DPDA). Tuttavia, contrariamente alle TM, essa non equivale, nella potenza riconoscitiva, alla versione nondeterministica. Si può dimostrare che la classe di linguaggi riconosciuta da PDA coincide con la classe dei linguaggi liberi da contesto. I DPDA riconoscono una sottoclasse propria dei linguaggi liberi da contesto, i cui membri sono detti semplicemente linguaggi liberi da contesto deterministici. Rispetto alla classificazione di Chomsky questa classe è un sottoinsieme proprio della classe dei context-free e un soprainsieme di quella dei regolari. Una gerarchia più ampia di linguaggi context-free derivante dal determinismo è stata introdotta da Hibbard in \cite{Hibbard:67:CFdet} tramite gli automi \eng{$d$-limited}.


\subsection{Automi a stati finiti}\label{subs:prel:NFA}

\subsubsection{NFA e DFA}
\begin{defin}[automa a stati finiti nondeterministico]
	Un \emph{automa a stati finiti nondeterministico} (NFA) è una quintupla $A=\tuple{Q,\Sigma,\delta,q_0,F}$, dove:
	\begin{itemize}
		\item $Q$ è un insieme finito e non vuoto di stati;
		\item $\Sigma$ è l'alfabeto di input;
		\item $\delta:Q\times\Sigma\to\subsets{Q}$ è la funzione di transizione. Se a un dato passo l'automa è nello stato $p$, legge il simbolo $\sigma$, e $\delta(p,\sigma)\ni q$, allora l'automa può passare allo stato $q$ e alla lettura del simbolo successivo;
		\item $q_0\in Q$ è lo stato iniziale;
		\item $F\subseteq Q$ è l'insieme degli stati finali.
	\end{itemize}
	Un NFA accetta una parola $w\in\Sigma\star$ se e solo se esiste una computazione che, a partire dallo stato $q_0$ e dalla lettura del primo simbolo di $w$, termina in uno stato finale dopo aver letto tutti i simboli dell'input.
\end{defin}
\begin{defin}
	Un \emph{automa a stati finiti deterministico} (DFA) è un NFA in cui $\card{\delta(q,\sigma)}\leq 1 ~ \forall q\in Q,\sigma\in\Sigma$.
\end{defin}

\begin{figure}
	\centering
	\begin{tikzpicture}[shorten >=1pt,node distance=1.1cm,auto,initial text=]
	\node[state,initial]	(q_0)			{$q_0$};
	\node[state]		(q_1)	[right=of q_0]	{$q_1$};
	\node[state,accepting]	(q_2)	[right=of q_1]	{$q_2$};
	\path[->]
	(q_0)	edge			node		{$0$} (q_1)
		edge [loop above]	node 		{$0,1$} ()
	(q_1)	edge			node		{$1$} (q_2);
\end{tikzpicture}

	\caption{Diagramma di transizione per un NFA che accetta le stringhe che finiscono in $01$}
\end{figure}

Si può dimostrare che NFA e DFA riconoscono la stessa classe di linguaggi e che essa coincide con la classe dei linguaggi regolari. La tabella \ref{tab:prel:chomskyhier} riassume la classificazione di Chomsky completa di corrispondenza con le rispettive classi di linguaggi e riconoscitori.

\begin{table}
	\caption{Classificazione di Chomsky con corrispondenza con le rispettive classi di linguaggi e riconoscitori. $a$ è un simbolo terminale, $A$ e $B$ nonterminali, $\alpha$, $\beta$ e $\gamma$ parole qualunque, con $\gamma$ non vuota.}
	\label{tab:prel:chomskyhier}
	\centering
	\begin{tabularx}{\textwidth}{lXXl}
		\toprule
		\textbf{Grammatica} & \textbf{Linguaggi generabili} & \textbf{Riconoscitore}  & \textbf{Regole di produzione}         \\
		\midrule
		Tipo 0              & Ricorsivamente enumerabili    & Macchine di Turing      & $\gamma\to\alpha$                     \\
		Tipo 1              & Dipendenti dal contesto       & Automi lineari limitati & $\alpha A\beta\to\alpha\gamma\beta$   \\
		Tipo 2              & Liberi dal contesto           & Automi a pila           & $A\to\alpha$                          \\
		Tipo 3              & Regolari                      & Automi a stati finiti   & $A\to a$, $A\to aB$, $A\to\emptyword$ \\
		\bottomrule
	\end{tabularx}
\end{table}

\subsubsection{Automi \eng{two-way}}
Un altro sistema equivalente a quello degli NFA per riconoscere i linguaggi regolari è quello degli automi \eng{two-way}, simili ad essi ma con la capacità di muoversi in ambe le direzioni tra i simboli dell'input. Useremo la formalizzazione di \cite{Pighizzini:14:limitedRE}, poiché si avvicina molto a quella per gli automi \eng{limited}. L'equivalenza degli automi two-way deterministici e DFA è discussa in \cite{Shallit:09:secondLFA}.

Un automa a stati finiti two-way ha le stesse componenti di una macchina di Turing: controllo finito, nastro e testina. Tuttavia, esso non può effettuare operazioni di scrittura; inoltre, può visitare unicamente le celle che in origine contengono l'input. Questo è infatti delimitato da due simboli speciali, il \eng{left} ($\lem$) e il \eng{right} ($\rem$) \eng{end-marker}, oltre i quali la testina non può muoversi.
\begin{defin}
	Un \emph{automa a stati finiti \eng{2-way} nondeterministico} (2NFA) è una quintupla $A=\tuple{Q,\Sigma,\delta,q_0,F}$, dove:
	\begin{itemize}
		\item $Q$ è un insieme finito e non vuoto di stati;
		\item $\Sigma$ è l'alfabeto di input;
		\item $\delta:Q\times(\Sigma\cup\set{\lem,\rem})\to\subsets{Q\times\set{\Left,\Right}}$ è la funzione di transizione. Se a un dato passo l'automa è nello stato $p$, legge il simbolo $\sigma$, e $\delta(p,\sigma)\ni (q,D)$, allora l'automa può passare allo stato $q$ e alla lettura del simbolo alla sinistra di quello corrente se $D=\Left$, o alla sua destra se $D=\Right$. Non è possibile muovere la testina a sinistra di $\lem$ e a destra di $\rem$, se non per accettare;
		\item $q_0\in Q$ è lo stato iniziale;
		\item $F\subseteq Q$ è l'insieme degli stati finali.
	\end{itemize}
	I simboli speciali $\lem$ e $\rem$ circoscrivono l'input nonché lo spazio di lavoro sul nastro.

	Un 2NFA accetta una parola $w\in\Sigma\star$ se e solo se esiste una computazione che, a partire dallo stato $q_0$ e dalla lettura del primo simbolo dell'input (il nastro contenente $\lem w\rem$), termina in uno stato finale $q\in F$ violando il right end-marker.
\end{defin}

\begin{defin}
	Un \emph{automa a stati finiti \eng{two-way} deterministico} (2DFA) è un 2NFA in cui $\card{\delta(q,\sigma)}\leq 1 ~ \forall q\in Q,\sigma\in\Sigma\cup\set{\lem,\rem}$.
\end{defin}

Talvolta, per distinguerli dalle loro controparti two-way, gli NFA \eng{one-way} si abbreviano con 1NFA e, analogamente, i DFA con 1DFA.


\section{Complessità descrizionale}
Come visto, i linguaggi possono essere descritti da diversi sistemi: automi, grammatiche, etc. Una descrizione può essere codificata in simboli e misurata, ottenendo una rappresentazione della sua complessità. Il campo della complessità descrizionale studia le descrizioni da questo punto di vista, cercando quelle più concise e studiando quale sia il \eng{trade-off} di complessità quando si passa da una descrizione a un'altra equivalente.

Nella pratica, ogni sistema di descrizione ha caratteristiche che influenzano la sua complessità descrizionale. Per esempio, la complessità di un NFA o DFA è proporzionale al numero di stati che lo compongono. Dato un NFA di $n$ stati, la \eng{subset construction} (\cite{Rabin:59:NFA}) permette di ottenere un DFA equivalente con $2^n$ stati. Questo \eng{upper bound}, che limita la crescita in complessità nella conversione, è anche un \eng{lower bound}, cioè esistono linguaggi per cui è inevitabile che la simulazione raggiunga tale complessità. Ciò permette di concludere che la descrizione di un linguaggio da parte di un NFA può arrivare a essere esponenzialmente più efficace di quella di un DFA dal punto di vista della complessità descrizionale.

Una panoramica sulla complessità descrizionale, le sue declinazioni e i suoi sviluppi recenti è presentata in \cite{Kutrib:21:descriptional}.

\chapter{Automi \emph{1-limited}}



\section{Automi \emph{d-limited}}
\begin{defin}[automa \emph{$d$-limited}]
	Dato un intero $d\geq 0$, un automa \emph{$d$-limited} (abbreviato in $d$-LA) è una tupla $\tuple{Q,\Sigma,\Gamma,\delta,q_0,F}$ dove:
	\begin{itemize}
		\item $Q$ è un insieme finito e non vuoto di stati;
		\item $\Sigma$ è l'alfabeto di input;
		\item $\Gamma\supseteq\Sigma\cup\set{\lem,\rem}$ è l'alfabeto di lavoro (\emph{working alphabet}), dove $\lem$ e $\rem$ sono rispettivamente il \emph{left} e \emph{right end marker}, che circoscrivono l'input nonché lo spazio di lavoro sul nastro. $\Gamma$ è partizionato in $d+1$ sottoinsiemi $\Gamma_0,\Gamma_1,\dots,\Gamma_d$, con $\Gamma_0=\Sigma$ e $\lem,\rem\in\Gamma_d$. L'insieme $\Gamma_k$ rappresenta l'alfabeto a cui appartiene ogni simbolo alla $k$-esima visita. Dopo la $d$-esima visita, la cella rimane invariata (\emph{frozen});
		\item $\delta:Q\times\Gamma\to 2^{Q\times(\Gamma\setminus\set{\lem,\rem})\times\set{-1,+1}}$ è la funzione di transizione: a ogni passo la macchina, in funzione del simbolo corrente sul nastro e dello stato corrente, cambia stato, sovrascrive il simbolo e muove la testina a destra ($+1$) o a sinistra ($-1$). La scelta del simbolo è soggetta al partizionamento di $\Gamma$: un simbolo in $\Gamma_k$ viene sostituito con un simbolo in $\Gamma_{k+1}$ (ad eccezione di $\Gamma_d$). Le visite in cui si cambia direzione contano doppio\footnote{Ciò è una conseguenza del fatto che, in realtà, si contano per ogni cella non le visite, ma le scansioni da sinistra a destra (visite di numero dispari) e quelle da destra a sinistra (visite di numero pari), per cui un cambio di direzione comprende entrambe.}, sostituendo un simbolo in $\Gamma_k$ con un simbolo in $\Gamma_{k+2}$. La macchina ha, in generale, più possibilità per un passo evolutivo, operando in maniera nondeterministica.
		\item $F\subseteq Q$ è l'insieme degli stati finali.
	\end{itemize}
	Un automa $d$-limited accetta una parola $w$ se e solo se esiste una computazione che, a partire dallo stato $q_0$ con la testina sulla prima delle celle contenenti l'input (il nastro contenente $\lem w\rem$), termina in uno stato finale $q\in F$ violando il \emph{right end marker}.

	Un automa $d$-limited si dice deterministico se $\card{\delta(q,\gamma)}\leq 1 ~ \forall q\in Q,\gamma\in\Gamma$. Un automa si dice \emph{limited} se è \emph{$d$-limited} per qualche $d$.
\end{defin}
Si noti che gli 0-LA corrispondono esattamente ai 2NFA.



\section{Potenza riconoscitiva}



\section{Complessità descrizionale}

\chapter{\eng{Witness languages}}
Ai fini di ottenere o migliorare i lower bound, di risolvere i problemi aperti e in generale di ottenere una conoscenza più approfondita di un modello oggetto di studio, vengono studiati dei \eng{witness languages} ("linguaggi testimoni"), linguaggi la cui esistenza dimostra una congettura. Nella sezione \ref{subs:a1l:low} abbiamo studiato la famiglia di linguaggi $L_n$, testimone del lower bound di conversione da \la1 e 1DFA doppiamente esponenziale e da \la1 a 1NFA, 2NFA, 2DFA, D\la1 semplicemente esponenziale. In questo capitolo espandiamo lo studio di questi linguaggi introducendone di nuovi, costruendone altri riconoscitori e dimostrando l'ottimalità dei bound che li riguardano. Classifichiamo tali linguaggi in base a una caratteristica comune che ne rende lo studio (almeno in parte) uniforme: a blocchi, unari, con reset. I linguaggi unari, in particolare, sono di interesse in quanto considerati un caso speciale nello studio dei linguaggi formali (poiché non c'è distinzione tra unari regolari e unari context-free), ed è oggetto di problemi aperti.



\section{Linguaggi a blocchi}\label{sec:wit:blk}
In un linguaggio a blocchi di parametro $n$, ogni parola è composta dalla concatenazione di stringhe di lunghezza $n$, dette blocchi. Condizioni diverse sulla relazione tra i blocchi danno origine a diverse famiglie di linguaggi, ad esempio:
\begin{itemize}
	\item il linguaggio delle parole in cui l'ultimo blocco è uguale a uno dei precedenti:
	      \begin{equation*}
		      K_n := \{ x_1\cdots x_kx \mid k>0, x_1,\dots,x_k\in\{a,b\}^n, \exists j\in\{1,\dots,k\},x_j=x\}
	      \end{equation*}
	\item il linguaggio delle parole in cui due blocchi qualsiasi sono uguali:
	      \begin{equation*}
		      E_n := \{x_1\cdots x_k \mid k>0, x_1,\dots,x_k\in\{a,b\}^n,\exists i,j\in\{1,\dots,k\},i<j,x_i=x_j\}
	      \end{equation*}
	\item il linguaggio in cui $n$ blocchi sono uguali:
	      \begin{align*}
		      L_n := \{ & x_1x_2\cdots x_k\mid k\geq0, x_1,x_2,\dots,x_k\in\{0,1\}^n,                                  \\
		                & \exists i_1,i_2,\dots,i_n\in\{1,\dots,k\},i_1<i_2<\dots<i_n, x_{i_1}=x_{i_2}=\dots=x_{i_n}\}
	      \end{align*}
\end{itemize}

Per $L_n$ sono stati descritti al paragrafo \ref{subs:a1l:low} un \la1 riconoscitore e il lower bound sul numero di stati di un 1DFA, un \la1, e 1NFA, 2NFA, 2DFA o D\la1 che riconoscano $L_n$. Gli stessi risultati possono essere facilmente adattati agli altri due linguaggi presentati, usando una variante dell'algoritmo \ref{alg:a1l:lowLn:3f} per il riconoscimento e la distinguibilità per i lower bound. Questo dimostra che i tre linguaggi sono testimoni della distanza doppiamente esponenziale di complessità tra \la1 e 1DFA, di quella semplicemente esponenziale tra \la1 e NFA, 2NFA, 2DFA, e di quella almeno esponenziale tra \la1 e D\la1.


\subsection{Riconoscitori}
Descriviamo ora gli upper bound mancanti per il riconoscimento di questi linguaggi, cioè D\la1, 2NFA, 2DFA e 1NFA. Usiamo ancora una volta l'esempio di $L_n$ poiché le tecniche sono molto simili tra i diversi linguaggi.

\subsubsection{1DFA}
Un 1DFA che riconosca $L_n$ può, scansionando l'input da sinistra verso destra, contare le occorrenze di ogni possibile blocco. Per fare ciò, ogni blocco $x\in\set{0,1}^n$ ha un contatore associato. Per l'identificazione di un blocco, gli stati sono organizzati ad albero, in cui ogni ramo è prefisso di un blocco e una foglia coincide con l'incremento del contatore, nonché con la radice dell'albero successivo. Trovata la $n$-esima occorrenza di un blocco, l'automa si limita a contare modulo $n$ la restante lunghezza della parola di input per verificare la struttura a blocchi. Per la prima fase sono richiesti $n^{2^n}$ stati per i contatori, ciascuno dipendente da un albero binario completo di $2^n-1$ stati, mentre per la seconda fase sono sufficienti $n$ stati, per un totale di $(2^n-1)\cdot n^{2^n}+n$ stati.

\subsubsection{1NFA}
La figura \ref{img:wit:LnNFA} mostra un NFA che riconsce $L_n$. All'inizio l'automa prova a indovinare nondeterministicamente una stringa $x^{(i)}\in\set{0,1}^n$ che ritiene essere il blocco ripetuto. Questa mossa è rappresentata per semplicità come $\emptyword$-transizione, cioè senza leggere alcun simbolo, ma può essere convertita in una transizione normale. Scelto un candidato blocco $x$, l'automa conta le occorrenze di $x$ nell'input. Per fare ciò, utilizza $2n-1$ stati per ciascuna delle $n$ occorrenze (per ciascuno dei blocchi possibili). Le transizioni contrassegnate con $\ok$ indicano un confronto positivo tra il simbolo corrente e il rispettivo nel blocco candidato, quelle con $\nok$ un confronto negativo e quelle con $\any$ non dipendono dal confronto e contano semplicemente i simboli di input. I valori effettivi di $\ok$ e $\nok$ dipendono ovviamente dal blocco scelto all'inizio.
\begin{itemize}
	\item finché il blocco corrente coincide con il candidato, vengono effettuate le transizioni $\ok$, proseguendo nella serie di stati $x_{i,j}$, dove $i$ è il contatore di occorrenze allo stato attuale (si sta confrontando per l'$i+1$-esima) e $j$ è l'indice del simbolo che viene confrontato. Verificata la coincidenza dell'ultimo simbolo, cioè certificata l'occorrenza $i+1$, si passa al confronto del blocco successivo incrementando il contatore $i$ e passando quindi alla serie $x_{i+1,j}$;
	\item se i due blocchi non coincidono, il primo simbolo diverso tra i due porta l'automa a prendere una transizione $\nok$, proseguendo poi per la serie di stati $\bar x_{i,j}$, che contano i simboli fino alla fine del blocco senza confrontare. Al termine di questa serie si riprende la computazione dallo stato $x_{i,j}$ alla ricerca della $i+1$-esima occorrenza nel blocco successivo.
\end{itemize}
Una volta trovate $n$ occorrenze, l'automa passa in una serie di stati $f_n,f_1,\dots,f_{n-1}$, comuni a tutti i blocchi candidati, che contano modulo $n$ i simboli rimanenti, accettando se sono in numero multiplo di $n$. Il numero di stati è quindi in totale $1+2^n\cdot (2n-1)\cdot n+n$.

\begin{figure}
	\centering
	\begin{tikzpicture}[shorten >=1pt,initial text=,near/.style={node distance=5mm}]
	\scriptsize
	\node[state,initial] (q0) {\small $q_0$};
	\node[state,node distance=12mm] (x00) [right=of q0] {$x_{0,0}$};
	\node[node distance=8mm] (dots1) [above=of x00] {$\dots$};
	\node (y1) [above=of dots1]{$x^{(1)}$};
	\node (dots2) [below=of x00] {$\dots$};
	\node (y2n) [below=of dots2] {$x^{(2^n)}$};

	\path[->]
	(q0) edge[bend left] node [above] {$\emptyword$} (y1.west)
	(q0) edge[bend left] node [above] {$\emptyword$} (dots1.west)
	(q0) edge[bend right] node [below] {$\emptyword$} (x00.west)
	(q0) edge[bend right] node [below] {$\emptyword$} (dots2.west)
	(q0) edge[bend right] node [below] {$\emptyword$} (y2n.west);

	\node[state,near] (x01) [right=of x00] {$x_{0,1}$};
	\node[state,near] (x02) [right=of x01] {$x_{0,2}$};
	\node[near] (x02d) [right=of x02] {$\dots$};
	\node[state,near] (x0n-1) [right=of x02d] {\tiny $x_{0,n-1}$};

	\path[->]
	(x00)	edge node [above] {$\ok$} (x01)
	(x01)	edge node [above] {$\ok$} (x02)
	(x02)	edge node [above] {$\ok$} (x02d)
	(x02d)	edge node [above] {$\ok$} (x0n-1);

	\node[state,node distance=8mm] (bx01) [below right=of x00] {$\bar x_{0,1}$};
	\node[state,near] (bx02) [right=of bx01] {$\bar x_{0,2}$};
	\node[near] (bx02d) [right=of bx02] {$\dots$};
	\node[state,near] (bx0n-1) [right=of bx02d] {\tiny $\bar x_{0,n-1}$};

	\path[->]
	(x00)	edge node [left] {$\nok$} (bx01)
	(bx01)	edge node [below] {$\any$} (bx02)
	(bx02)	edge node [below] {$\any$} (bx02d)
	(bx02d)	edge node [below] {$\any$} (bx0n-1)
	(x01)	edge node [left] {$\nok$} (bx02)
	(x02)	edge node [left] {$\nok$} (bx02d);

	\node[state,near] (x10) [right=of x0n-1] {$x_{1,0}$};

	\path[->]
	(bx0n-1) edge[bend angle=70,bend left] node[below] {$\any$} (x00)
	(x0n-1) edge[bend angle=45,bend right] node[below] {$\nok$} (x00)
	(x0n-1) edge node [above] {$\ok$} (x10);

	\node[near]		(x10d)		[right=of x10] {$\dots$};
	\node[state,near]	(in-1n-1)	[right=of x10d] {\tiny $x_{n-1,n-1}$};
	\node[state]		(f1)		[right=of in-1n-1.north] {$f_1$};
	\node[state,near,accepting]	(fn)		[above=of f1] {$f_n$};
	\node[state,near]		(f2)		[below=of f1] {$f_2$};
	\node[near]	(fd)	[below=of f2] {$\dots$};
	\node[state,near]	(fn-1)	[below=of fd] {\tiny $f_{n-1}$};

	\path[->]
	(in-1n-1)	edge node[left] {$\ok$} (fn)
	(fn)		edge node[right] {$\any$} (f1)
	(f1)		edge node[right] {$\any$} (f2)
	(f2)		edge node[right] {$\any$} (fd)
	(fd)		edge node[right] {$\any$} (fn-1)
	(fn-1)		edge[bend right] node[right] {$\any$} (fn);
\end{tikzpicture}

	\caption{L'NFA che riconosce $L_n$.}
	\label{img:wit:LnNFA}
\end{figure}

\subsubsection{2DFA}
Un 2DFA può riconoscere $L_n$ eseguendo innanzitutto una scansione preliminare che verifichi che la lunghezza dell'input sia multipla di $n$, quindi effettuando una scansione del nastro per ogni possibile blocco, contando le occorrenze di quello corrente (con una strategia simile a quella dell'1NFA descritto precedentemente), finché trovando $n$ occorrenze di un blocco può semplicemente accettare violando l'end-marker destro. Il numero di stati richiesti è $n$ per la prima fase, $2^n\cdot n\cdot (2n-1)$ per la seconda e $1$ per accettare, per un totale di $n+2^n\cdot (2n-1)\cdot n+1$ come nel caso precedente.

Questa macchina è anche un 2NFA e D\la1, di cui non conosciamo riconoscitori migliori che sfruttino le capacità in più dei rispettivi modelli per riconoscere $L_n$.



\section{Linguaggi unari}\label{sec:wit:un}
I linguaggi unari possiedono la fondamentale proprietà secondo cui le classi dei context-free e dei regolari collidono in un'unica classe. Da ciò deriva il fatto che un automa $d$-limited riconosce precisamente i linguaggi regolari, ossia i context-free, per qualunque $d$.

La relazione tra automi limited e linguaggi unari è stata studiata estensivamente da Pighizzini e Prigioniero in \cite{Pighizzini:19:limitedunary}. In particolare, i \la1 che riconoscono linguaggi unari possono fare uso di una tecnica basata sulla \eng{binary carry sequence}:
\begin{defin}
	La \emph{binary carry sequence} è la successione infinita di interi $\sigma_1\sigma_2\cdots\sigma_j\cdots$ in cui $\sigma_j$ è l'esponente della più alta potenza di $2$ che divide $j$, per ogni intero $j\geq1$.
\end{defin}

Definiamo inoltre la \eng{backward increasing sequence}, una funzione che trasforma sequenze che ha proprietà interessanti in relazione alla binary carry sequence.
\begin{defin}
	Sia $s=k_1k_2\cdots k_j$ una sequenza finita di interi. La \emph{backward increasing sequence} di $s$, denotata con $\bis(s)$, è la più lunga successione ottenibile selezionando da destra verso sinistra gli elementi di $s$ solo finché si susseguono in ordine crescente. Formalmente, $\bis(k_1k_2\cdots k_j)=(i_1,i_2,\dots,i_r)$ se e solo se $i_1=k_{h_1},i_2=k_{h_2}\dots i_r=k_{h_r}$ dove $h_1=j$ e $h_t=\max\set{h'<h_{t-1}\mid k_{h'}>k_{h_{t-1}}}$.
\end{defin}

Si verifica il seguente risultato, dimostrato in \cite{Pighizzini:19:limitedunary}:
\begin{lemma}\label{lem:wit:bis}
	Sia $\sigma_1\sigma_2\cdots\sigma_j$ il prefisso di lunghezza $j$ della binary carry sequence.
	\begin{itemize}
		\item \label{lem:wit:bis:1} Se $\bis(\sigma_1\sigma_2\cdots\sigma_j)=(i_1,i_2,\dots,i_r)$ allora
		      \begin{equation*}
			      j=\sum_{t=1}^r 2^{i_t}
		      \end{equation*}
		      Ossia, i valori della backward increasing sequence applicata al prefisso di lunghezza $j$ della binary carry sequence corrispondono alle posizioni dei bit a $1$ della rappresentazione binaria di $j$.
		\item \label{lem:wit:bis:2} $o_j$ è il minor numero naturale che non occorre in $\bis(\sigma_1\sigma_2\cdots\sigma_{j-1})$.
	\end{itemize}
\end{lemma}

Spieghiamo ora una tecnica che ci permette di riconoscere diversi linguaggi unari facendo uso del lemma per contare i simboli dell'input. Si prenda in considerazione il linguaggio singoletto $\set{a^{2^n}}$, dove $n>0$ è un parametro intero. Lo scopo della macchina è quello di scrivere il prefisso di lunghezza $2^n$ della binary carry sequence sul nastro, sostituendo le $a$.

Un D\la1 di alfabeto di lavoro $\set{a,0,1,\dots,n}$ può innanzitutto sovrascrivere il primo simbolo con $0$, primo elemento della binary carry sequence. Supponendo che a un certo punto della computazione la macchina abbia scritto il prefisso di lunghezza $j$ della binary carry sequence sovrascrivendo i primi $j$ simboli, il simbolo $\sigma_{j+1}$ può essere calcolato grazie al \hyperref[lem:wit:bis:2]{secondo punto} del lemma \ref{lem:wit:bis}: la macchina può effettuare visite in sola lettura verso sinistra, individuando il più piccolo naturale che non occorre nella backward increasing sequence del prefisso scritto. Tale numero è scritto nella successiva cella scrivibile e il procedimento viene ripetuto. Se a un certo punto della computazione l'automa scrive $n$ ($2^n$-esimo elemento della binary carry sequence) e la successiva cella contiene $\rem$ l'automa accetta. Se si raggiunge $\rem$ senza che $n$ venga scritto allora la parola è troppo corta, mentre se viene scritto ma la successiva cella non contiene l'end-marker la parola è troppo lunga. Questa macchina può essere implementata in $O(n)$ stati e fare uso di $O(n)$ simboli (dettagli sull'implementazione e l'algoritmo che essa utilizza sono presenti in \cite{Pighizzini:19:limitedunary}). Scrivendo al posto di $n$ un simbolo di reset $\reset$, in cui l'automa si comporta come sull'end-marker sinistro, si adatta la macchina ad accettare $\set{a^{2^n}}\star$ quando $\lem$ o $\reset$ precedono $\rem$. Un 1NFA necessità di $2^n$ stati per un contatore al fine di riconoscere questo linguaggio. Mereghetti e Pighizzini hanno dimostrato in \cite{Mereghetti:00:twoway} che lo stesso lower bound vale per i 2NFA. Questo linguaggio è quindi testimone della distanza almeno esponenziale da D\la1 a 2NFA (e quindi anche da \la1 e verso 1NFA, 1DFA e 2DFA) per i linguaggi unari.

Una variante di questa tecnica può essere applicata per riconoscere il linguaggio $M_N:=\set{a^N}\star$, con $N>0$ un intero qualsiasi. Per fare ciò, un D\la1 $_N$ può scrivere il prefisso di lunghezza $N-1$ della binary carry sequence con la tecnica descritta precedentemente, quindi scrivere un simbolo di reset $\reset$ che equivale a $\lem$ e impone alla macchina di ricominciare la scrittura dal simbolo $\sigma_1$. Questo procedimento viene ripetuto per tutto l'input: la stringa ha lunghezza multipla di $N$ se e solo se l'ultimo simbolo prima dell'end-marker destro è $\reset$ o $\lem$. Se $w$ è l'input il nastro verrà riscritto come segue:
\begin{equation*}
	\underbrace{\sigma_1\cdots\sigma_{N-1}\reset\cdots\reset\sigma_1\cdots\sigma_{N-1}\reset}_{\floor{\len w/N} \text{ volte}}\sigma_1\cdots\sigma_{\scriptscriptstyle \len w\mkern -11mu \mod N}
\end{equation*}
Poiché $N$ non è necessariamente una potenza di $2$, rilevare quando è il momento di scrivere $\reset$ non è triviale. Per fare ciò $B_N$, durante procedimento di identificazione del prossimo simbolo da scrivere, verifica inoltre se la backward increasing sequence del prefisso attuale rappresenta il numero $N-1$. Se così fosse, il \hyperref[lem:wit:bis:1]{primo punto} del lemma \ref{lem:wit:bis} dimostra che il prefisso attuale ha lunghezza $N-1$, e che quindi il prossimo simbolo da scrivere è $\reset$.

L'implementazione di $B_N$ richiede un numero di stati e un alfabeto di lavoro lineari nel massimo elemento della binary carry sequence che può essere scritto, ossia $O(\log N)$.



\section{Linguaggi con reset}


\subsection{L'automa di Meyer e Fischer}
Con l'intenzione di studiare un witness language che non fosse né unario né a blocchi, Pighizzini, Prigioniero e Sádovský hanno studiato in \cite{Pighizzini:22:limitedwitness} il riconoscimento da parte di \la1 del linguaggio accettato dall'automa $S_N$, introdotto da Meyer e Fischer in \cite{Meyer:71:ecodescription} come testimone della distanza esponenziale tra 1NFA e 1DFA. L'automa, rappresentato in figura \ref{img:wit:Sn}, ha una struttura ciclica: se si considerano solo le transizioni generate dal simbolo $a$ l'automa riconosce il linguaggio $\set{a^N}\star$. Il ruolo di $b$ è scelto nondeterministicamente tra due: o viene ignorato, lasciando invariato lo stato, o impone alla macchina un reset, riportandola allo stato iniziale.

\begin{figure}
	\centering
	% \begin{tikzpicture}
\begin{tikzpicture}[shorten >=1pt,initial text=]
	\def\stateangle{30}
	\def\statedistance{2cm}
	\path[inner sep=0]
	(0,0) node[state](5){$q_{N-1}$}

	++(60:\statedistance) node[state,accepting,initial](0) {$q_0$}
	++(0:\statedistance) node[state](1){$q_1$}
	++(-60:\statedistance) node[state](2){$q_2$}

	(5)
	++(-60:\statedistance)  node[state](4){$q_{N-2}$}
	++(0:\statedistance)
	node[state](3){$q_3$};

	\path[->]
	(1) edge[loop above] node[near end,right] {$b$} (1)
	(2) edge[loop right] node[near end,below] {$b$} (2)
	(3) edge[loop below] node[near end,left] {$b$} (3)
	(4) edge[loop below] node[near end,left] {$b$} (4)
	(5) edge[loop left] node[near end,above] {$b$} (5)

	(0) edge[bend left] node[above] {$a$} (1)
	(1) edge[bend left] node[right] {$a$} (2)
	(2) edge[bend left] node[right] {$a$} (3)
	(3) edge[bend left,dashed]  (4)
	(4) edge[bend left] node[left] {$a$} (5)
	(5) edge[bend left] node[left] {$a$} (0)

	(1) edge[bend left] node[above] {$b$} (0)
	(2.170) edge[out=170,in=-45] node[above] {$b$} (0.-45)
	(3) edge[] node[above,near start] {$b$} (0)
	(4.70) edge[out=70,in=-75] node[left] {$b$} (0.-75)
	(5) edge[bend right] node[left] {$b$} (0)
	;
\end{tikzpicture}

	\caption{L'NFA $S_N$ di Meyer e Fischer.}
	\label{img:wit:Sn}
\end{figure}

Viene dimostrato in \cite{Meyer:71:ecodescription} che il minimo 1DFA equivalente a $S_N$ ha $2^N$ stati. In \cite{Pighizzini:22:limitedwitness} viene costruito un 2DFA di $N+2$ stati equivalente a $S_N$, il che testimonia la distanza esponenziale tra D\la1 e 1DFA, e viene dimostrato che il minimo 2NFA equivalente a $S_{2^n}$, con $n>0$, ha almeno $2^n$ stati.

Per simulare con \la1 automi con un meccanismo di reset simile a quello di $S_N$, si può sfruttare una variante della tecnica basata sulla binary carry sequence di cui al paragrafo \ref{sec:wit:un}. In particolare, per ogni $N>1$, $S_N$ può essere riconosciuto da un \la1 $C_N$ con $O(\log N)$ stati e un alfabeto di lavoro di $O(\log N)$ simboli.
Poiché, come accennato in precedenza, restringendo il comportamento di $S_N$ al solo input $a$ si riconosce il linguaggio unario $\set{a^N}\star$, in questo caso $C_N$ può comportarsi esattamente come $B_N$, l'automa che riconosce tale linguaggio. $C_N$ costruisce quindi una serie di ripetizioni del prefisso di lunghezza $N-1$ della binary carry sequence, separandole con il carattere di reset $\reset$. Per quanto riguarda il comportamento di $C_N$ per il simbolo di input $b$, la macchina può scegliere nondeterministicamente di effettuare una di due mosse, ciascuna corrispondente con uno dei comportamenti di $S_N$ leggendo $b$:
\begin{itemize}
	\item per simulare le transizioni che non cambiano stato, $C_N$ sovrascrive $b$ con un simbolo $\neutr$ neutrale, nel senso che il comportamento della macchina in esso sarà semplicemente di procedere senza tenerlo in considerazione;
	\item per simulare le transizioni di reset, $C_N$ sovrascrive $b$ con il simbolo di reset $\reset$. Si noti che sia in questo caso sia se il ciclo di $\set{a^N}\star$ viene completato il simbolo $\reset$ corrisponde agli istanti in cui $S_N$ passa allo stato $q_0$.
\end{itemize}
Se si incontra $b$ a destra di $\lem$ o di $\reset$, la mossa non è definita, così come non lo è in $q_0$ per $S_N$. $C_N$ accetta se e solo se $\reset$ o $\lem$ sono seguiti da $\rem$. L'incremento di stati e di simboli rispetto a $B_N$ è trascurabile ed entrambi rimangono in numero di $O(\log N)$.

In conclusione il linguaggio $\generated{S_N}$ è testimone del gap esponenziale tra \la1 e 1NFA e doppiamente esponenziale tra \la1 e 1DFA.


\subsection{L'automa di Moore}
Un automa simile a quello di Meyer e Fischer è stato presentato da E. F. Moore in \cite{Moore:56:gendanken} e studiato da F. R. Moore in \cite{Moore:71:automatabounds}. L'automa $R_N$, rappresentato in figura \ref{img:wit:Rn}, presenta un meccanismo di reset, ma questa volta le transizioni di reset possono avvenire solo dallo stato $q_N$ verso lo stato $q_1$ o verso lo stato $q_2$. Queste transizioni sono, tra l'altro, l'unica fonte di nondeterminismo della macchina.

\begin{figure}
	\centering
	\begin{tikzpicture}[shorten >=1pt,initial text=]
	\node[state,initial]	(q1)			{$q_1$};
	\node[state]		(q2)	[right=of q1]	{$q_2$};
	\node[state]		(q3)	[right=of q2]	{$q_3$};
	\node			(qd)	[right=of q3]	{$\dots$};
	\node[state]		(qn-1)	[right=of qd]	{\small $q_{N-1}$};
	\node[state,accepting]	(qn)	[right=of qn-1]	{$q_N$};
	\path[->]
	(q1)	edge			node[above]	{$a$} (q2)
		edge[loop above]	node[above] 	{$b$} ()
	(q2)	edge			node[above]	{$a,b$} (q3)
	(q3)	edge			node[above]	{$a,b$} (qd)
	(qd)	edge			node[above]	{$a,b$} (qn-1)
	(qn-1)	edge			node[above]	{$a,b$} (qn)
	(qn)	edge[bend angle=25,bend left]		node[below]	{$a$} (q2)
	(qn)	edge[bend angle=35,bend left]		node[below]	{$a$} (q1);
\end{tikzpicture}

	\caption{L'NFA $S_N$ di Moore.}
	\label{img:wit:Rn}
\end{figure}

Un \la1 $D_N$ può riconoscere l'automa di Moore con un adattamento della tecnica precedente:
\begin{itemize}
	\item simulando la computazione tra $q_1$ e $q_N$, $D_N$ sovrascrive l'input con la binary carry sequence trovando, scritto ogni simbolo, il più piccolo naturale che non ricorre nella backward increasing sequence del prefisso corrente e, contemporaneamente, controllando se tale sequenza rappresenta il numero $N$. La macchina verifica che la successione dei simboli di input porti a transizioni legali (ossia non può ricevere $b$ nello stato $q_N$);
	\item $D_N$ sceglie nondeterministicamente quale transizione simulare nel caso stia simulando lo stato $q_N$ e riceva $a$ in input;
	\item per simulare la transizione dallo stato $q_n$ verso lo stato $q_1$, $D_N$ utilizza un simbolo di reset $\reset_1$ che si comporta come l'end-marker sinistro, e impone all'automa di ricominciare a scrivere la binary carry sequence a partire dal suo primo elemento $\sigma_1$. Lo stesso simbolo viene scritto se simulando lo stato $q_1$, cioè dopo la scrittura di $\reset_1$ oppure dopo $\lem$, il simbolo di input successivo è $b$, simulando la transizione dallo stato $q_1$ in se stesso;
	\item per simulare la transizione dallo stato $q_N$ verso lo stato $q_2$, $D_N$ utilizza un simbolo di reset $\reset_2$. Questo simbolo si comporta in modo leggermente diverso rispetto a $\reset_1$ e $\lem$, in quanto muovendosi alla sua destra l'automa scrive la binary carry sequence partendo dal suo secondo elemento $\sigma_2$ (invece che dal primo). Questo simbolo porta l'automa a comportarsi diversamente anche nelle computazioni successive: leggendolo durante il tracciamento della backward increasing sequence, l'automa si comporta come se avesse letto $\reset_1$ e prima di esso $\sigma_1$. In questo modo il conteggio degli stati con la tecnica della binary carry sequence rimane valido, pur contando a partire dal suo secondo elemento.
\end{itemize}
L'automa $D_N$ può essere implementato con $O(\log N)$ stati e $O(\log N)$ simboli.

F. R. Moore ha dimostrato che $R_N$, oltre a essere minimo, riconosce un linguaggio che è testimone della distanza esponenziale da 1NFA a 1DFA. L'esistenza di un \la1 di complessità logaritmica che riconosce $\generated{R_N}$ dimostra che il linguaggio riconosciuto è testimone anche della distanza esponenziale tra \la1 e 1NFA e doppiamente esponenziale tra \la1 e 1DFA.

\chapter{Altri risultati e problemi aperti}\label{cha:prob}
In questo capitolo concludiamo riassumendo i problemi aperti che riguardano gli automi $1$-limited e presentandone qualche ulteriore risultato. Diamo poi uno sguardo ai risultati relativi ad altre varianti degli automi limited.



\section{Problemi aperti}
Lo studio dei $1$-limited e del loro rapporto con altri riconoscitori, pur avendo portato a risultati soddisfacenti nei casi principali (1NFA, 1DFA), lascia diverse domande senza risposta.


\subsection{L'eliminazione del nondeterminismo}
Il più importante dei problemi aperti che riguardano i $1$-limited è il gap di complessità descrizionale tra \la1 e D\la1, di cui conosciamo un lower bound esponenziale (corollario \ref{cor:a1l:LAtoDLA}) e un upper bound doppiamente esponenziale, derivante dalla simulazione degli \la1 da parte dei 1DFA. Non abbiamo, tra l'altro, evidenze di una distanza maggiore tra \la1 e 2DFA, nonostante le restrizioni ulteriori di quest'ultimo modello rispetto ai D\la1.

Pighizzini, Prigioniero e Sádovský (\cite{Pighizzini:22:limitedwitness}) hanno proposto, con l'intenzione di introdurre dei witness language per una ipotetica distanza più che esponenziale tra \la1 e D\la1 (o tra \la1 e 2DFA), due linguaggi binari basati sulla parità (XOR). Il linguaggio $P_n$ è un linguaggio a blocchi, in cui lo XOR bit a bit di un certo numero di blocchi risulta nell'ultimo blocco. Il linguaggio $P'_n$ è una versione di $P_n$ in cui cade il vincolo dei blocchi, permettendo lo XOR di qualunque sottostringa che non si sovrapponga:
\begin{align*}
	P_n := \{  x_1\dots x_kx \mid ~ & k>0, x_1,\dots,x_k,x\in\{0,1\}^n,                                \\
	                                & \exists h>0,i_1,i_2,\dots,x_h\in\{1,\dots,k\},i_1<i_2<\dots<i_h: \\
	                                & x=x_{i_1}\oplus x_{i_2}\oplus\dots\oplus x_{i_h}\}
\end{align*}
\begin{align*}
	P'_n := \{  wx \mid ~ & w\in\{0,1\}^*,x\in\{0,1\}^n,                                                \\
	                      & \exists h>0,x_1,x_2,\dots,x_h\in\{0,1\}^n,y_0,y_1,\dots,y_h\in\{0,1\}^*:    \\
	                      & w=y_0x_1y_1\dots y_{n-1}x_hy_h \land x=x_1\oplus x_2\oplus\dots\oplus x_h\}
\end{align*}

Questi linguaggi possono essere riconosciuti da un \la1 lineare in $n$ con un adattamento delle tecniche per i linguaggi a blocchi (paragrafo \ref{sec:wit:blk}), tuttavia è sconosciuto un lower bound per un 1DFA equivalente. Rimane una congettura, per il momento, che questi linguaggi non siano accettati da D\la1 o 2DFA di dimensione semplicemente esponenziale.


\subsection{Simulazioni mancanti}
Non si conoscono simulazioni che permettano, a partire da un \la1, di rimuovere il nondeterminismo (\la1\tto D\la1), la capacità di riscrittura (\la1\tto 2NFA), o la combinazione delle due (\la1\tto 2DFA), se non quelle in 1NFA e 1DFA.

In effetti, non si conosce una costruzione che permetta di convertire un 2NFA in un 2DFA (è il problema aperto posto da Sakoda e Sipser in \cite{Sakoda:78:sizetwoway}), di cui si conosce un lower bound polinomiale e un upper bound esponenziale. Pur non risolvendo questo problema, sarebbe interessante investigare la sua potenziale correlazione con la simulazione di \la1 da parte di D\la1. In particolare, l'eliminazione del nondeterminismo da una macchina two-way potrebbe sfruttare una tecnica adattabile al caso \la1\tto D\la1.

Per quanto riguarda l'eliminazione della capacità di riscrittura, ci sembra improbabile trovare una tecnica che permetta di codificare nel solo movimento two-way, anche se nondeterministico, i possibili stati del nastro modificato.

I risultati relativi alla complessità descrizionale dei vari riconoscitori di linguaggi regolari sono riassunti nella figura \ref{img:pro:sim}.

\begin{figure}
	\centering
	\begin{tikzpicture}
	\footnotesize
	\newcommand{\bounds}[4]{$\geq$ #1 \ref{itm:pro:#2}\\ $\leq$ #3 \ref{itm:pro:#4}}
	\newcommand{\boundsq}[3]{\bounds{#1}{#3:l}{#2}{#3:u}}

	\node (la)	{\la1};
	\node (dla)	[below=5cm of la] {D\la1};
	\node (nfa)	[below left=3cm of la] {1NFA/2NFA};
	\node (2dfa)	[right=6cm of nfa] {2DFA};
	\node (1dfa)	[right=2cm of 2dfa] {1DFA};

	\path[-latex]
	(la)  edge node[above left,align=left] {\boundsq{exp}{exp}{lanfa}} (nfa.north)
	(la) edge node[above right,align=left] {\boundsq{2exp}{2exp}{lanfa}} (1dfa.north)
	(la)  edge node[left,align=left] {\bounds{exp}{ladla:l}{2exp}{ladfa:u}} (dla.north)
	(la)  edge node[below left,align=left] {\bounds{exp}{lanfa:l}{2exp}{ladfa:u}} (2dfa.north)
	(dla) edge node[below left,align=left] {\bounds{exp}{dlanfa:l}{2exp}{lanfa:u}} (nfa.south)
	(dla)  edge node[below right,align=left] {\bounds{exp}{dlanfa:l}{2exp}{ladfa:u}} (1dfa.south)
	(dla) edge node[above left,align=left] {\bounds{exp}{dlanfa:l}{2exp}{ladfa:u}} (2dfa.south);
\end{tikzpicture}

	\captionsetup{singlelinecheck=off}
	\caption[]{Upper e lower bound di conversione tra $1$-limited e altri riconoscitori di linguaggi regolari.\label{img:pro:sim}
		\begin{enumerate}[(a)]
			\item \label{itm:pro:lanfa:u} simulazione \la1\tto 1NFA (paragrafo \ref{subs:a1l:up})
			\item \label{itm:pro:lanfa:l} witness language tra cui $L_n$ (paragrafo \ref{subs:a1l:low})
			\item \label{itm:pro:ladfa:u} simulazione \la1\tto 1DFA (paragrafo \ref{subs:a1l:up});
			\item \label{itm:pro:ladla:l} corollario \ref{cor:a1l:LAtoDLA} da witness language $L_n$;
			\item \label{itm:pro:dlanfa:l} witness language tra cui l'unario $\set{a^{2^n}}$ (paragrafo \ref{sec:wit:un}).
		\end{enumerate}
	}
\end{figure}



\subsection{Il caso unario}
Per quanto riguarda i linguaggi unari, lo sviluppo della tecnica basata sulla binary carry sequence (paragrafo \ref{sec:wit:un}) ha portato all'individuazione di testimoni della distanza almeno esponenziale tra \la1 e 1NFA o 2NFA, superando i precedenti risultati di lower bound quadratico (\cite{Pighizzini:14:limitedRE}). Non si conosce una simulazione che limiti la distanza nel caso unario a semplicemente esponenziale, né sono stati trovati lower bound maggiori, pertanto il gap tra \la1 e 1DFA per gli unari rimane un problema aperto, la cui risposta è inclusa tra il singolo e il doppio esponenziale.



\section{Altri risultati sugli \texorpdfstring{$1$-limited}{1-limited}}


\subsection{Grammatiche}
Negli scorsi capitoli ci siamo concentrati sugli aspetti degli $1$-limited che li mettono in relazione con altri riconoscitori di linguaggi regolari; tuttavia sono noti anche risultati relativi al loro rapporto con grammatiche. Nel caso unario in particolare, in cui le grammatiche di tipo $2$ descrivono linguaggi regolari, Pighizzini e Prigioniero hanno dimostrato (\cite{Pighizzini:19:limitedunary}) che è possibile convertire una grammatica context-free in un \la1 che riconosce lo stesso linguaggio, con un incremento polinomiale della dimensione della descrizione.

Il risultato è stato esteso al caso generale tramite la nozione di Parikh-equivalenza. Due linguaggi (o due loro generatori o riconoscitori) si dicono Parikh-equivalenti se per ogni parola del primo esiste una parola nel secondo uguale a meno dell'ordine dei simboli, o alternativamente con lo stesso numero di occorrenze per ogni simbolo. Il teorema di Parikh dimostra che ogni linguaggio libero dal contesto ha un Parikh-equivalente regolare. Nell'articolo di cui sopra viene dimostrato che è possibile convertire una grammatica context-free qualsiasi in un \la1 che riconosca il regolare Parikh-equivalente, con un incremento polinomiale nella dimensione della grammatica.


\subsection{Complessità temporale}
Abbiamo dimostrato che l'utilizzo di un \la1 per riconoscere un linguaggio regolare può fornire un vantaggio esponenziale di complessità descrizionale rispetto all'utilizzo di un 1NFA, o doppiamente esponenziale rispetto a un 1DFA. Questo vantaggio, tuttavia, va a pari passo con una potenziale perdita nella complessità temporale. Infatti, mentre un 1NFA o 1DFA effettua una transizione per carattere di input, risultando in un tempo lineare nella sua dimensione, un \la1, in quanto macchina two-way, può ripercorrere l'input un numero di volte ben oltre il lineare. Guillon e Prigioniero hanno dimostrato in \cite{Guillon:19:linearlimited} che dato un qualsiasi \la1 è possibile costruire un \la1 equivalente che lavori in tempo lineare nella dimensione dell'input, con un incremento polinomiale nella complessità descrizionale e preservando lo stato di determinismo.



\section{Altri automi limited}
Lo studio degli automi limited va ben oltre il caso degli $1$-limited, trattando sia il caso $d>1$ sia varianti del modello, con particolare attenzione alla complessità descrizionale. In \cite{Pighizzini:19:limited} Pighizzini presenta la maggior parte dei risultati conseguiti su questa famiglia di riconoscitori, tra cui riportiamo qui alcuni dei più interessanti.


\subsection{\texorpdfstring{$d$-limited con $d>1$}{d-limited con d>1}}
Come detto, già Hibbard (\cite{Hibbard:67:CFdet}) ha dimostrato che i $d$-limited nondeterministici caratterizzano i linguaggi context-free se $d\geq2$. Nel caso deterministico, esiste una gerarchia stretta in cui i D\la2 caratterizzano i linguaggi liberi da contesto deterministici e per ogni $d\geq2$ esiste un linguaggio riconosciuto da un \la{(d+1)} non riconosciuto da alcun \la d, pur non raggiungendo l'interezza della classe dei context-free per alcun $d$.

Pighizzini e Pisoni hanno dimostrato in \cite{Pighizzini:14:limitedCF} che la trasformazione da \la2 ad automa a pila è esponenziale, mentre la trasformazione inversa (che non è banale come nel caso 1DFA\tto\la1) è polinomiale. Kutrib, Pighizzini e Wendlandt hanno dimostrato che per $d>2$ rimane un upper bound esponenziale alla conversione in PDA (\cite{Kutrib:18:complexlimited}).


\subsection{Varianti di automi limited}
In \cite{Wechsung:79:complexities} Wechsung studia una variante di automa limited in cui, invece di un $d$ costante, la limitazione all'ultima visita con capacità di scrittura dipende da una funzione $f(n)$ della dimensione dell'input.

Più recentemente è stato introdotto il modello degli \eng{strongly limited automata}, che possiedono un numero di ulteriori restrizioni ispirate agli automi limited che riconoscono linguaggi di Dyck (espressioni ben parentesizzate). Pighizzini dimostra in \cite{Pighizzini:16:stronglylimited} che questo modello ha dimensione al più polinomiale nella dimensione di una grammatica context-free o di un PDA equivalenti, in contrapposizione con i risultati esponenziali del modello classico.

Numerosi altri modelli e varianti sono stati studiati, tra cui automi limited con capacità di scrittura nelle \emph{ultime} $d$ visite (\cite{Wechsung:79:complexities}) e anche estensioni probabilistiche (\cite{Yamakami:19:limitedmodels}).


\chapter*{Ringraziamenti}
Vorrei ringraziare Luca Prigioniero e Giovanni Pighizzini per essere stati ottimi relatori quanto ricercatori. In particolare ringrazio il dottor Prigioniero per la sua grande competenza e disponibilità, nel correggere le mie bozze e nel rispondere alle mie infinite domande, premiando la mia curiosità.
Ringrazio la mia famiglia, i miei amici e in generale chi mi ha supportato nella mia vita e nella mia carriera universitaria. Il supporto degli altri è un dono che non bisogna dare per scontato.
Ringrazio i professori che si contraddistinguono nell'insegnare trasmettendo la propria passione agli studenti. Sono loro che plasmano il cuore dell'università e di chi la vive.
Infine ringrazio tutti i ricercatori, che hanno il rarissimo dono di guardare al futuro.


\printbibliography[heading=bibintoc]

\end{document}
