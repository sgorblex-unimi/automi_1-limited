\chapter{Altri risultati e problemi aperti}
In questo capitolo concludiamo riassumendo i problemi aperti che riguardano gli automi $1$-limited e presentandone qualche ulteriore risultato. Diamo poi uno sguardo ai risultati relativi ad altre varianti degli automi limited.



\section{Problemi aperti}
Lo studio dei $1$-limited e del loro rapporto con altri riconoscitori, pur avendo portato a risultati soddisfacenti nei casi principali (1NFA, 1DFA), lascia diverse domande senza risposta.


\subsection{L'eliminazione del nondeterminismo}
Il più importante dei problemi aperti che riguardano i $1$-limited è il gap di complessità descrizionale tra \la1 e D\la1, di cui conosciamo un lower bound esponenziale (corollario \ref{cor:a1l:LAtoDLA}) e un upper bound doppiamente esponenziale, derivante dalla simulazione degli \la1 da parte dei 1DFA. Non abbiamo, tra l'altro, evidenze di una distanza maggiore tra \la1 e 2DFA, nonostante le restrizioni ulteriori di quest'ultimo modello rispetto ai D\la1.

Pighizzini, Prigioniero e Sádovský (\cite{Pighizzini:22:limitedwitness}) hanno proposto, con l'intenzione di introdurre dei witness language per una ipotetica distanza più che esponenziale tra \la1 e D\la1 (o tra \la1 e 2DFA), due linguaggi binari basati sulla parità (XOR). Il linguaggio $P_n$ è un linguaggio a blocchi, in cui lo XOR bit a bit di un certo numero di blocchi risulta nell'ultimo blocco. Il linguaggio $P'_n$ è una versione di $P_n$ in cui cade il vincolo dei blocchi, permettendo lo XOR di qualunque sottostringa che non si sovrapponga:
\begin{align*}
	P_n := \{  x_1\dots x_kx \mid ~ & k>0, x_1,\dots,x_k,x\in\{0,1\}^n,                                \\
	                                & \exists h>0,i_1,i_2,\dots,x_h\in\{1,\dots,k\},i_1<i_2<\dots<i_h: \\
	                                & x=x_{i_1}\oplus x_{i_2}\oplus\dots\oplus x_{i_h}\}
\end{align*}
\begin{align*}
	P'_n := \{  wx \mid ~ & w\in\{0,1\}^*,x\in\{0,1\}^n,                                                \\
	                      & \exists h>0,x_1,x_2,\dots,x_h\in\{0,1\}^n,y_0,y_1,\dots,y_h\in\{0,1\}^*:    \\
	                      & w=y_0x_1y_1\dots y_{n-1}x_hy_h \land x=x_1\oplus x_2\oplus\dots\oplus x_h\}
\end{align*}

Questi linguaggi possono essere riconosciuti da un \la1 lineare in $n$ con un adattamento delle tecniche per i linguaggi a blocchi (paragrafo \ref{sec:wit:blk}), tuttavia è sconosciuto un lower bound per un 1DFA equivalente. Rimane una congettura, per il momento, che questi linguaggi non siano accettati da D\la1 o 2DFA di dimensione semplicemente esponenziale.


\subsection{Simulazioni mancanti}
Non si conoscono simulazioni che permettano, a partire da un \la1, di rimuovere il nondeterminismo (\la1\tto D\la1), la capacità di riscrittura (\la1\tto 2NFA), o la combinazione delle due (\la1\tto 2DFA), se non quelle in 1NFA e 1DFA.

In effetti, non si conosce una costruzione che permetta di convertire un 2NFA in un 2DFA (o un 1NFA in un 2DFA), di cui si conosce un lower bound polinomiale e un upper bound esponenziale. Pur non risolvendo questo problema, sarebbe interessante investigare la sua potenziale correlazione con la simulazione di \la1 da parte di D\la1. In particolare, l'eliminazione del nondeterminismo da una macchina two-way potrebbe sfruttare una tecnica adattabile al caso \la1\tto D\la1.

Per quanto riguarda l'eliminazione della capacità di riscrittura, ci sembra improbabile trovare una tecnica che permetta di codificare nel solo movimento two-way, anche se nondeterministico, i possibili stati del nastro modificato.


\subsection{Il caso unario}
Per quanto riguarda i linguaggi unari, lo sviluppo della tecnica basata sulla binary carry sequence (paragrafo \ref{sec:wit:un}) ha portato all'individuazione di testimoni della distanza almeno esponenziale tra \la1 e 1NFA o 2NFA, superando i precedenti risultati di lower bound quadratico (\cite{Pighizzini:14:limitedRE}). Non si conosce una simulazione che limiti la distanza nel caso unario a semplicemente esponenziale, né sono stati trovati lower bound maggiori, pertanto il gap tra \la1 e 1DFA per gli unari rimane un problema aperto, la cui risposta è inclusa tra il singolo e il doppio esponenziale.



\section{Altri risultati sugli \texorpdfstring{$1$-limited}{1-limited}}


\subsection{Grammatiche}
Negli scorsi capitoli ci siamo concentrati sugli aspetti degli $1$-limited che li mettono in relazione con altri riconoscitori di linguaggi regolari; tuttavia sono noti anche risultati relativi al loro rapporto con grammatiche. Nel caso unario in particolare, in cui le grammatiche di tipo $2$ descrivono linguaggi regolari, Pighizzini e Prigioniero hanno dimostrato (\cite{Pighizzini:19:limitedunary}) che è possibile convertire una grammatica context-free in un \la1 che riconosce lo stesso linguaggio, con un incremento polinomiale della dimensione della descrizione.

Il risultato è stato esteso al caso generale tramite la nozione di Parikh-equivalenza. Due linguaggi (o due loro generatori o riconoscitori) si dicono Parikh-equivalenti se per ogni parola del primo esiste una parola nel secondo uguale a meno dell'ordine dei simboli, o alternativamente con lo stesso numero di occorrenze per ogni simbolo. Il teorema di Parikh dimostra che ogni linguaggio libero da contesto ha un Parikh-equivalente regolare. Nell'articolo di cui sopra viene dimostrato che è possibile convertire una grammatica context-free qualsiasi in un \la1 che riconosca il regolare Parikh-equivalente, con un incremento polinomiale nella dimensione della grammatica.


\subsection{Complessità temporale}
Abbiamo dimostrato che l'utilizzo di un \la1 per riconoscere un linguaggio regolare può fornire un vantaggio esponenziale di complessità descrizionale rispetto all'utilizzo di un 1NFA, o doppiamente esponenziale rispetto a un 1DFA. Questo vantaggio, tuttavia, va a pari passo con una potenziale perdita nella complessità temporale. Infatti, mentre un 1NFA o 1DFA effettua una transizione per carattere di input, risultando in un tempo lineare nella sua dimensione, un \la1, in quanto macchina two-way, può ripercorrere l'input un numero di volte ben oltre il lineare. Guillon e Prigioniero hanno dimostrato in \cite{Guillon:19:linearlimited} che dato un qualsiasi \la1 è possibile costruire un \la1 equivalente che lavori in tempo lineare nella dimensione dell'input, con un incremento polinomiale nella complessità descrizionale e preservando lo stato di determinismo.



\section{Altri automi limited}
Lo studio degli automi limited va ben oltre il caso degli $1$-limited, trattando sia il caso $d>1$ sia varianti del modello, con particolare attenzione alla complessità descrizionale. In \cite{Pighizzini:19:limited} Pighizzini presenta la maggior parte dei risultati conseguiti su questa famiglia di riconoscitori, tra cui riportiamo qui alcuni dei più interessanti.


\subsection{\texorpdfstring{$d$-limited con $d>1$}{d-limited con d>1}}
Come detto, già Hibbard (\cite{Hibbard:67:CFdet}) ha dimostrato che i $d$-limited nondeterministici riconoscono esattamente i linguaggi context-free se $d\geq2$. Nel caso deterministico, esiste una gerarchia stretta in cui i D\la2 riconoscono i linguaggi liberi da contesto deterministici e per ogni $d\geq2$ esiste un linguaggio riconosciuto da un \la{(d+1)} non riconosciuto da alcun \la d, pur non raggiungendo l'interezza della classe dei context-free per alcun $d$.

Pighizzini e Pisoni hanno dimostrato in \cite{Pighizzini:14:limitedCF} che la trasformazione da \la2 ad automa a pila è esponenziale, mentre la trasformazione inversa (che non è banale come nel caso 1DFA\tto\la1) è polinomiale. Kutrib, Pighizzini e Wendlandt hanno dimostrato che per $d>2$ rimane un upper bound esponenziale alla conversione in PDA (\cite{Kutrib:18:complexlimited}).


\subsection{Varianti di automi limited}
In \cite{Wechsung:79:complexities} Wechsung studia una variante di automa limited in cui, invece di un $d$ costante, la limitazione all'ultima visita con capacità di scrittura dipende da una funzione $f(n)$ della dimensione dell'input.

Più recentemente è stato introdotto il modello degli \eng{strongly limited automata}, che possiedono un numero di ulteriori restrizioni ispirate agli automi limited che riconoscono linguaggi di Dyck (espressioni ben parentesizzate). Pighizzini dimostra in \cite{Pighizzini:16:stronglylimited} che questo modello ha dimensione al più polinomiale nella dimensione di una grammatica context-free o di un PDA equivalenti, in contrapposizione con i risultati esponenziali del modello classico.

Numerosi altri modelli e varianti sono stati studiati, tra cui automi limited con capacità di scrittura nelle \emph{ultime} $d$ visite (\cite{Wechsung:79:complexities}) e anche estensioni probabilistiche (\cite{Yamakami:19:limitedmodels}).
