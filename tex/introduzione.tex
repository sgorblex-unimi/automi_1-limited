\preface

\section*{Storia e motivazioni}
Lo studio moderno degli automi limited nasce dall'unione di due rami dell'informatica teorica: la classificazione dei linguaggi formali e la complessità descrizionale.

I modelli computazionali con limitazioni sono un classico oggetto di studio dell'informatica teorica. Gli automi limited sono stati introdotti da Hibbard nel 1967 (\cite{Hibbard:67:CFdet}) come strumento per costruire una gerarchia che caratterizzasse il determinismo all'interno della classe dei linguaggi liberi dal contesto.

La complessità descrizionale, che ha avuto origine con l'articolo fondatore di Meyer e Fischer (\cite{Meyer:71:ecodescription}), studia quanto una descrizione di un linguaggio formale, ad esempio una grammatica o un riconoscitore, può essere succinta, e il rapporto tra le dimensioni di diverse descrizioni equivalenti. Il classico esempio è la costruzione per sottoinsiemi di Rabin e Scott (\cite{Rabin:59:NFA}) che permette a un automa a stati finiti deterministico di simularne uno nondeterministico pagando un costo esponenziale nel numero di stati.

Un automa $d$-limited è una macchina di Turing nondeterministica con alcune limitazioni: lo spazio di lavoro è limitato alle celle che inizialmente contengono l'input (delimitate da $\lem$ e $\rem$) e la scrittura di una cella è possibile solo durante le sue prime $d$ visite. Sebbene Hibbard abbia dimostrato che gli automi $d$-limited nondeterministici caratterizzano i linguaggi liberi dal contesto quando $d>1$, gli automi $0$-limited (cioè con $d=0$) corrispondono alla definizione di automi a stati finiti nondeterministici \eng{two-way}, riconoscitori di linguaggi regolari. Wagner e Wechsung (\cite{Wagner:86:compCompl}) hanno dimostrato che permettendo la scrittura alla prima visita di una cella la classe riconosciuta è la stessa, ossia gli automi $1$-limited caratterizzano i linguaggi regolari. Pighizzini e Pisoni hanno ottenuto in \cite{Pighizzini:14:limitedRE} l'upper bound del numero di stati necessari a un automa a stati finiti per simulare un automa $1$-limited di $n$ stati, ottenendo un numero di stati esponenziale in $n$ per gli automi a stati finiti nondeterministici e doppiamente esponenziale per i deterministici. Lo studio dei lower bound tramite \eng{witness languages}, iniziato nello stesso articolo, ha dimostrato che per qualunque $n$ esistono casi in cui tale costo non può essere evitato. In altre parole, gli automi $1$-limited possono fornire descrizioni molto più compatte rispetto ai riconoscitori standard per la classe dei regolari. Negli anni successivi sono stati studiati altri linguaggi testimoni per i bound che riguardano gli $1$-limited (ad esempio \cite{Pighizzini:22:limitedwitness}), ma anche altre varianti di automi limited, con risultati interessanti specialmente in relazione alla complessità descrizionale.

\section*{In questa tesi}
Nel capitolo \ref{cha:prel} riassumiamo le conoscenze che consistono in prerequisiti per la comprensione del cuore della tesi, toccando le principali definizioni della teoria dei linguaggi formali. Il capitolo \ref{cha:a1l} introduce il modello degli automi $d$-limited e degli $1$-limited in particolare, quindi ne studia la potenza computazionale e descrizionale, presentando le principali simulazioni da parte degli altri riconoscitori di linguaggi regolari e dimostrando l'ottimalità dei rispettivi costi tramite un linguaggio a blocchi testimone del lower bound. Approfondiamo nel capitolo \ref{cha:wit} upper e lower bound di alcuni witness languages, studiando alcune tecniche applicabili agli $1$-limited per riconoscere famiglie di linguaggi con caratteristiche comuni: linguaggi a blocchi, unari (cioè costruiti su un alfabeto di un solo simbolo) o derivanti da automi con meccanismi di reset. Infine, il capitolo \ref{cha:prob} riassume i risultati sulla materia, evidenziandone i problemi ancora aperti e i le possibili ricerche future, e parla brevemente degli altri risultati che riguardano gli automi limited e le loro varianti.
