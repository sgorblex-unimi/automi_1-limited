\preface
La teoria dei linguaggi formali è un campo dell'informatica teorica che studia i linguaggi, cioè insiemi di stringhe, e i modi di rappresentarli finitamente. In quest'ambito si studiano diversi modelli di calcolo, in grado di riconoscere varie classi di linguaggi. Un classico oggetto di studio di questo campo sono i modelli computazionali con risorse limitate, di cui si studia la potenza riconoscitiva e il rapporto con altri modelli.

Gli automi limited sono stati introdotti da Hibbard nel 1967 \cite{Hibbard:67:CFdet} come strumento per costruire una gerarchia che caratterizzasse il determinismo all'interno della classe dei linguaggi liberi dal contesto.
Un automa $d$-limited è una macchina di Turing nondeterministica in cui lo spazio di lavoro è limitato alle celle che inizialmente contengono l'input e la scrittura di una cella è possibile solo durante le sue prime $d$ visite. Sebbene Hibbard abbia dimostrato che gli automi $d$-limited nondeterministici caratterizzano i linguaggi liberi dal contesto quando $d>1$, gli automi $0$-limited (cioè con $d=0$) corrispondono alla definizione di automi a stati finiti nondeterministici \eng{two-way}, riconoscitori di linguaggi regolari. Wagner e Wechsung \cite{Wagner:86:compCompl} hanno dimostrato che permettendo la scrittura alla prima visita di una cella la classe riconosciuta è la stessa, ossia gli automi $1$-limited caratterizzano i linguaggi regolari.

L'individuazione di diversi riconoscitori equivalenti per la stessa classe di linguaggi ha portato la ricerca a chiedersi quale sia il costo delle diverse rappresentazioni. La complessità descrizionale, che ha avuto origine con l'articolo fondatore di Meyer e Fischer \cite{Meyer:71:ecodescription}, studia quanto una descrizione di un linguaggio formale, ad esempio una grammatica o un automa, può essere succinta, e le relazioni tra le dimensioni di diverse descrizioni equivalenti. Per esempio, la costruzione per sottoinsiemi di Rabin e Scott \cite{Rabin:59:NFA} ha dimostrato che pagando un costo esponenziale nel numero di stati è possibile rimuovere il nondeterminismo da un automa a stati finiti. Meyer e Fischer hanno dimostrato che esistono linguaggi per cui tale costo è inevitabile \cite{Meyer:71:ecodescription}.

Nel 2014 Pighizzini e Pisoni hanno ripreso lo studio degli automi limited trattandoli dal punto di vista della complessità descrizionale, ottenendo in \cite{Pighizzini:14:limitedRE} l'upper bound del numero di stati necessari a un automa a stati finiti per simulare un automa $1$-limited. Tale simulazione, a partire da un automa $1$-limited di $n$ stati, produce un automa a stati finiti nondeterministico con un numero di stati esponenziale in $n$, o uno deterministico con un numero di stati doppiamente esponenziale. Lo stesso articolo ha dimostrato che per qualunque $n$ esistono casi in cui tale costo non può essere evitato. In altre parole, gli automi $1$-limited possono fornire descrizioni molto più compatte rispetto ai riconoscitori standard per la classe dei regolari. Negli anni successivi sono state sviluppate diverse tecniche applicabili agli automi $1$-limited che hanno portato allo studio di altri linguaggi che dimostrano i bound che li riguardano (ad esempio \cite{Pighizzini:22:limitedwitness}). Al contempo sono state presentate e studiate altre varianti di automi limited, con risultati interessanti specialmente in relazione alla complessità descrizionale.

In questo elaborato riassumiamo innanzitutto (capitolo \ref{cha:prel}) le conoscenze che consistono in prerequisiti per la comprensione del cuore della tesi, toccando le principali definizioni della teoria dei linguaggi formali. Il capitolo \ref{cha:a1l} introduce il modello degli automi $d$-limited e degli $1$-limited in particolare, quindi ne studia la potenza computazionale e descrizionale, presentando le principali simulazioni da parte degli altri riconoscitori di linguaggi regolari e dimostrando l'ottimalità dei rispettivi costi. Il capitolo \ref{cha:wit} approfondisce alcune tecniche applicabili agli $1$-limited per riconoscere linguaggi con particolari caratteristiche. Per i linguaggi a blocchi, cioè composti da concatenazioni di stringhe di uguale lunghezza legate da una particolare relazione, viene costruito un riconoscitore che implementa un algoritmo basato sul confronto simbolo a simbolo, mentre per i lower bound viene usata una dimostrazione basata su distinguibilità (tecnica nota in letteratura). Per i linguaggi unari, cioè costruiti su un alfabeto di un solo simbolo, viene sviluppata una tecnica basata sulla binary carry sequence (la successione dei massimi esponenti di $2$ che dividono ogni naturale) che permette a un automa $1$-limited di verificare se la lunghezza di una stringa è multipla di un naturale $n$ usando un numero logaritmico di stati in $n$. Infine viene studiato un adattamento della stessa tecnica utile a simulare alcuni automi a stati finiti che hanno una transizione che riporta allo stato iniziale. Il capitolo \ref{cha:prob} conclude con una panoramica dei risultati sul modello, evidenziandone i problemi ancora aperti e le possibili ricerche future, e discute brevemente gli altri risultati che riguardano gli automi limited e le loro varianti.
