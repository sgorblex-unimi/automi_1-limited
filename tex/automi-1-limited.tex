\chapter{Automi \emph{1-limited}}



\section{Automi \emph{d-limited}}
\begin{defin}[automa \emph{$d$-limited}]
	Dato un intero $d\geq 0$, un automa \emph{$d$-limited} (abbreviato in $d$-LA) è una tupla $\tuple{Q,\Sigma,\Gamma,\delta,q_0,F}$ dove:
	\begin{itemize}
		\item $Q$ è un insieme finito e non vuoto di stati;
		\item $\Sigma$ è l'alfabeto di input;
		\item $\Gamma\supseteq\Sigma\cup\set{\lem,\rem}$ è l'alfabeto di lavoro (\emph{working alphabet}), dove $\lem$ e $\rem$ sono rispettivamente il \emph{left} e \emph{right end marker}, che circoscrivono l'input nonché lo spazio di lavoro sul nastro. $\Gamma$ è partizionato in $d+1$ sottoinsiemi $\Gamma_0,\Gamma_1,\dots,\Gamma_d$, con $\Gamma_0=\Sigma$ e $\lem,\rem\in\Gamma_d$. L'insieme $\Gamma_k$ rappresenta l'alfabeto a cui appartiene ogni simbolo alla $k$-esima visita. Dopo la $d$-esima visita, la cella rimane invariata (\emph{frozen});
		\item $\delta:Q\times\Gamma\to 2^{Q\times(\Gamma\setminus\set{\lem,\rem})\times\set{-1,+1}}$ è la funzione di transizione: a ogni passo la macchina, in funzione del simbolo corrente sul nastro e dello stato corrente, cambia stato, sovrascrive il simbolo e muove la testina a destra ($+1$) o a sinistra ($-1$). La scelta del simbolo è soggetta al partizionamento di $\Gamma$: un simbolo in $\Gamma_k$ viene sostituito con un simbolo in $\Gamma_{k+1}$ (ad eccezione di $\Gamma_d$). Le visite in cui si cambia direzione contano doppio\footnote{Ciò è una conseguenza del fatto che, in realtà, si contano per ogni cella non le visite, ma le scansioni da sinistra a destra (visite di numero dispari) e quelle da destra a sinistra (visite di numero pari), per cui un cambio di direzione comprende entrambe.}, sostituendo un simbolo in $\Gamma_k$ con un simbolo in $\Gamma_{k+2}$. La macchina ha, in generale, più possibilità per un passo evolutivo, operando in maniera nondeterministica.
		\item $F\subseteq Q$ è l'insieme degli stati finali.
	\end{itemize}
	Un automa $d$-limited accetta una parola $w$ se e solo se esiste una computazione che, a partire dallo stato $q_0$ con la testina sulla prima delle celle contenenti l'input (il nastro contenente $\lem w\rem$), termina in uno stato finale $q\in F$ violando il \emph{right end marker}.

	Un automa $d$-limited si dice deterministico se $\card{\delta(q,\gamma)}\leq 1 ~ \forall q\in Q,\gamma\in\Gamma$. Un automa si dice \emph{limited} se è \emph{$d$-limited} per qualche $d$.
\end{defin}
Si noti che gli 0-LA corrispondono esattamente ai 2NFA.



\section{Potenza riconoscitiva}



\section{Complessità descrizionale}
