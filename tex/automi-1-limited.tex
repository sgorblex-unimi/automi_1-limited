\chapter{Automi \eng{\texorpdfstring{$1$}{1}-limited}}
Come accennato al paragrafo \ref{subs:prel:PDA}, il fatto che gli automi a pila deterministici riconoscano una classe intermedia tra i linguaggi liberi da contesto e quelli regolari ha portato Hibbard a voler definire una generalizzazione del determinismo nei linguaggi liberi da contesto e i loro riconoscitori. Per fare ciò, egli introdusse in \cite{Hibbard:67:CFdet} gli \eng{scan limited automata}, che oggi, con qualche piccola modifica al modello, chiamiamo automi \eng{limited}. Lo studio degli automi limited ha ormai superato il suo scopo iniziale: non riguarda più unicamente il caso deterministico e si presta particolarmente a risultati nell'ambito della complessità descrizionale.

In questo capitolo presentiamo innanzitutto gli automi \eng{$d$-limited} e il loro potere riconoscitivo, poi entriamo nel merito degli automi \eng{$1$-limited}, il nostro oggetto di studio, con una panoramica dei principali risultati che li riguardano.



\section{Automi \eng{d-limited}}
Gli automi \eng{$d$-limited} sono un caso particolare di automi linearmente limitati. Un automa \eng{$d$-limited}, con $d\in\N$, è una macchina di Turing nondeterministica in cui lo spazio di lavoro, che all'inizio contiene l'input, è circoscritto dagli \eng{end-marker} $\lem$ e $\rem$ (come già visto per i 2NFA al paragrafo \ref{subs:prel:NFA}). Inoltre, la capacità di scrittura della macchina è limitata, potendo scrivere su una cella solo durante le prime $d$ visite.
\begin{defin}[automa \eng{$d$-limited}]
	Dato un intero $d\geq 0$, un automa \eng{$d$-limited} ($d$-LA) è una tupla $A=\tuple{Q,\Sigma,\Gamma,\delta,q_0,F}$ dove:
	\begin{itemize}
		\item $Q$ è un insieme finito e non vuoto di stati;
		\item $\Sigma$ è l'alfabeto di input;
		\item $\Gamma\supseteq\Sigma\cup\set{\lem,\rem}$ è l'alfabeto di lavoro (\eng{working alphabet}), dove $\lem$ e $\rem$ circoscrivono l'input nonché lo spazio di lavoro sul nastro. $\Gamma$ è partizionato in $d+1$ sottoinsiemi $\Gamma_0,\Gamma_1,\dots,\Gamma_d$, con $\Gamma_0=\Sigma$ e $\lem,\rem\in\Gamma_d$. L'insieme $\Gamma_k$ rappresenta l'alfabeto a cui appartiene ogni simbolo alla $k$-esima visita della cella che lo contiene. Dopo la $d$-esima visita, la cella rimane invariata (\eng{frozen});
		\item $\delta:Q\times\Gamma\to \subsets{Q\times(\Gamma\setminus\set{\lem,\rem})\times\set{\Left,\Right}}$ è la funzione di transizione. Se a un dato passo l'automa è nello stato $p$, legge il simbolo $\sigma$, e $\delta(p,\sigma)\ni (q,\gamma,D)$, allora:
		      \begin{itemize}
			      \item $q$ è il prossimo stato;
			      \item $\gamma$ è il simbolo che verrà scritto in sostituzione a $\sigma$ nella cella corrente, allo spostamento della testina;
			      \item $D$ e la direzione in cui si muoverà la testina, $\Left$ se a sinistra e $\Right$ se a destra.
		      \end{itemize}
		      La natura del simbolo $\gamma$ è soggetta al partizionamento di $\Gamma$: un simbolo in $\Gamma_k$ viene sostituito con un simbolo in $\Gamma_{k+1}$ (ma un simbolo in $\Gamma_d$ viene sostituito con se stesso). Le visite in cui si cambia direzione contano doppio\footnote{Ciò è una conseguenza del fatto che, in realtà, si contano per ogni cella non le visite, ma le scansioni da sinistra a destra (visite di numero dispari) e quelle da destra a sinistra (visite di numero pari), per cui un cambio di direzione comprende entrambe.}, sostituendo un simbolo in $\Gamma_k$ con un simbolo in $\Gamma_{k+2}$. Le celle contenenti gli \eng{end-marker} non sono mai modificate, né è possibile muoversi a sinistra di $\lem$ e a destra di $\rem$, se non per accettare.
		\item $q_0\in Q$ è lo stato iniziale;
		\item $F\subseteq Q$ è l'insieme degli stati finali.
	\end{itemize}
	Una \emph{configurazione} di un automa $d$-limited si indica con $xqy$, dove $x$ è la parola prima della cella corrente (può essere omessa se $x=\emptyword$), $q$ è lo stato corrente, e $y$ è la parola che inizia con il simbolo nella cella corrente (si omettono $\lem$ e $\rem$). Si scrive $xpy\trans zqw$ se esiste una transizione che porta dalla configurazione $xpy$ a $zqw$ e $xpy\transs zqw$ se esiste una computazione in zero o più passi che porta dalla configurazione $xpy$ a $zqw$.

	Un automa $d$-limited accetta una parola $w\in\Sigma\star$ se e solo se esiste una computazione che, a partire dalla configurazione $q_0w$ (il nastro contenente $\lem w\rem$), termina in uno stato finale $q\in F$ violando il \eng{right end-marker}.

	Un automa $d$-limited si dice deterministico (D$d$-LA) se $\card{\delta(q,\gamma)}\leq 1 ~ \forall q\in Q,\gamma\in\Gamma$. Un automa si dice \eng{limited} se è \eng{$d$-limited} per qualche $d$.
\end{defin}
Sebbene formalmente il simbolo corrente viene sempre sovrascritto, nella pratica ciò non accade. Quando un simbolo deve rimanere invariato, esso viene sostituito con un simbolo equivalente rispetto alla funzione di transizione (si immagini che, ad esempio, $a$ venga sostituito con $a'$ ma comunque rappresentato con $a$). In questo modo si preserva l'informazione originale ma si mantiene la correttezza rispetto al modello. Una tecnica analoga viene usata quando si segna (\eng{mark}) una cella, per esempio sostituendo $a$ con $\bar a$, preservando il simbolo ma dandovi un valore aggiuntivo.

Si noti che gli 0-LA corrispondono esattamente ai 2NFA.

\begin{figure}
	\centering
	\begin{tikzpicture}[cell/.style={minimum height=1.5em,minimum width=1.5em,outer sep=0pt,rectangle,draw,node distance=0pt}]
	\node (lem) {\Large $\lem$};
	\node[cell] (0) [right=of lem]{$\sigma_0$};
	\node[cell] (1) [right=of 0] {$\sigma_1$};
	\node[cell] (2) [right=of 1] {$\sigma_2$};
	\node[cell] (3) [right=of 2] {$\sigma_3$};
	\node[cell, minimum width=2.5em] (worddots) [right=of 3] {$\dots$};
	\node[cell] (last) [right=of worddots] {$\sigma_n$};
	\node[node distance=0pt] (rem) [right=of last]{\Large $\rem$};
	\node[cell] (control) [above=0.75cm of 2,thick] {$q$};
	\draw[-latex] (control) -- (2);
\end{tikzpicture}

	\caption{Rappresentazione di un \la d di esempio in un dato istante.}
\end{figure}



\section{Potenza riconoscitiva}
In \cite{Hibbard:67:CFdet} Hibbard studia il diverso potere riconoscitivo dei D\la d al variare di $d$, con i D\la2 che riconoscono esattamente i linguaggi liberi da contesto deterministici. Nonostante ciò, e nonostante gli \la d siano una restrizione dei LBA, che riconoscono i \eng{context-sensitive}, è noto che i \la d nondeterministici riconoscono esattamente la classe dei linguaggi \eng{context-free}, per qualunque $d\geq2$. Questo risultato deriva dalla costruzione di alcune trasformazioni da modelli equivalenti agli automi a pila ai \la2 e viceversa, combinati con trasformazioni da \la{d+1} a \la d, per $d\geq2$. Una dimostrazione più semplice del potere riconoscitivo dei \la2 si ottiene dal teorema di Chomsky-Schützenberger (\cite{Chomsky:63:algebraCF}) e un \la2 molto semplice che riconosce i linguaggi di Dyck (dettagli in \cite{Pighizzini:19:limited}).

Per quanto riguarda $d<2$, poiché gli \la0 sono esattamente i 2NFA è chiaro che questi riconoscano esattamente i linguaggi regolari. Wagner e Wechsung hanno dimostrato in \cite{Wagner:86:compCompl} che la possibilità di riscrivere una cella durante la sua prima visita non aumenta il potere riconoscitivo, ergo anche gli \la1 riconoscono i linguaggi regolari. Come vedremo, infatti, il vero potere degli \la1 non è nella classe riconosciuta, ma nella complessità della loro descrizione, notevolmente ridotta rispetto ai riconoscitori standard per i linguaggi regolari.

Una panoramica più approfondita dei risultati ottenuti per gli automi $d$-limited, specialmente per $d\geq2$ che qui non trattiamo, si può trovare in \cite{Pighizzini:19:limited}.



\section{Complessità descrizionale}
