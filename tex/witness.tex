\chapter{\eng{Witness languages}}
Ai fini di ottenere o migliorare i lower bound, di risolvere i problemi aperti e in generale di ottenere una conoscenza più approfondita di un modello oggetto di studio, vengono studiati dei \eng{witness languages} ("linguaggi testimoni"), linguaggi la cui esistenza dimostra una congettura. Nella sezione \ref{subs:a1l:low} abbiamo studiato la famiglia di linguaggi $L_n$, testimone del lower bound di conversione da \la1 e 1DFA doppiamente esponenziale e da \la1 a 1NFA, 2NFA, 2DFA, D\la1 semplicemente esponenziale. In questo capitolo espandiamo lo studio di questi linguaggi introducendone di nuovi, costruendone altri riconoscitori e dimostrando l'ottimalità dei bound che li riguardano. Classifichiamo tali linguaggi in base a una caratteristica comune che ne rende lo studio (almeno in parte) uniforme: a blocchi, unari, con reset. I linguaggi unari, in particolare, sono di interesse in quanto considerati un caso speciale nello studio dei linguaggi formali (poiché non c'è distinzione tra unari regolari e unari context-free), ed è oggetto di problemi aperti.



\section{Linguaggi a blocchi}\label{sec:wit:blk}
In un linguaggio a blocchi di parametro $n$, ogni parola è composta dalla concatenazione di stringhe di lunghezza $n$, dette blocchi. Condizioni diverse sulla relazione tra i blocchi danno origine a diverse famiglie di linguaggi, ad esempio:
\begin{itemize}
	\item il linguaggio delle parole in cui l'ultimo blocco è uguale a uno dei precedenti:
	      \begin{equation*}
		      K_n := \{ x_1\cdots x_kx \mid k>0, x_1,\dots,x_k\in\{a,b\}^n, \exists j\in\{1,\dots,k\},x_j=x\}
	      \end{equation*}
	\item il linguaggio delle parole in cui due blocchi qualsiasi sono uguali:
	      \begin{equation*}
		      E_n := \{x_1\cdots x_k \mid k>0, x_1,\dots,x_k\in\{a,b\}^n,\exists i,j\in\{1,\dots,k\},i<j,x_i=x_j\}
	      \end{equation*}
	\item il linguaggio in cui $n$ blocchi sono uguali:
	      \begin{align*}
		      L_n := \{ & x_1x_2\cdots x_k\mid k\geq0, x_1,x_2,\dots,x_k\in\{0,1\}^n,                                  \\
		                & \exists i_1,i_2,\dots,i_n\in\{1,\dots,k\},i_1<i_2<\dots<i_n, x_{i_1}=x_{i_2}=\dots=x_{i_n}\}
	      \end{align*}
\end{itemize}

Per $L_n$ sono stati descritti al paragrafo \ref{subs:a1l:low} un \la1 riconoscitore e il lower bound sul numero di stati di un 1DFA, un \la1, e 1NFA, 2NFA, 2DFA o D\la1 che riconoscano $L_n$. Gli stessi risultati possono essere facilmente adattati agli altri due linguaggi presentati, usando una variante dell'algoritmo \ref{alg:a1l:lowLn:3f} per il riconoscimento e la distinguibilità per i lower bound. Questo dimostra che i tre linguaggi sono testimoni della distanza doppiamente esponenziale di complessità tra \la1 e 1DFA, di quella semplicemente esponenziale tra \la1 e NFA, 2NFA, 2DFA, e di quella almeno esponenziale tra \la1 e D\la1.


\subsection{Riconoscitori}
Descriviamo ora gli upper bound mancanti per il riconoscimento di questi linguaggi, cioè D\la1, 2NFA, 2DFA e 1NFA. Usiamo ancora una volta l'esempio di $L_n$ poiché le tecniche sono molto simili tra i diversi linguaggi.

\subsubsection{1DFA}
Un 1DFA che riconosca $L_n$ può, scansionando l'input da sinistra verso destra, contare le occorrenze di ogni possibile blocco. Per fare ciò, ogni blocco $x\in\set{0,1}^n$ ha un contatore associato. Per l'identificazione di un blocco, gli stati sono organizzati ad albero, in cui ogni ramo è prefisso di un blocco e una foglia coincide con l'incremento del contatore, nonché con la radice dell'albero successivo. Trovata la $n$-esima occorrenza di un blocco, l'automa si limita a contare modulo $n$ la restante lunghezza della parola di input per verificare la struttura a blocchi. Per la prima fase sono richiesti $n^{2^n}$ stati per i contatori, ciascuno dipendente da un albero binario completo di $2^n-1$ stati, mentre per la seconda fase sono sufficienti $n$ stati, per un totale di $(2^n-1)\cdot n^{2^n}+n$ stati.

\subsubsection{1NFA}
La figura \ref{img:wit:LnNFA} mostra un NFA che riconsce $L_n$. All'inizio l'automa prova a indovinare nondeterministicamente una stringa $x^{(i)}\in\set{0,1}^n$ che ritiene essere il blocco ripetuto. Questa mossa è rappresentata per semplicità come $\emptyword$-transizione, cioè senza leggere alcun simbolo, ma può essere convertita in una transizione normale. Scelto un candidato blocco $x$, l'automa conta le occorrenze di $x$ nell'input. Per fare ciò, utilizza $2n-1$ stati per ciascuna delle $n$ occorrenze (per ciascuno dei blocchi possibili). Le transizioni contrassegnate con $\ok$ indicano un confronto positivo tra il simbolo corrente e il rispettivo nel blocco candidato, quelle con $\nok$ un confronto negativo e quelle con $\any$ non dipendono dal confronto e contano semplicemente i simboli di input. I valori effettivi di $\ok$ e $\nok$ dipendono ovviamente dal blocco scelto all'inizio.
\begin{itemize}
	\item finché il blocco corrente coincide con il candidato, vengono effettuate le transizioni $\ok$, proseguendo nella serie di stati $x_{i,j}$, dove $i$ è il contatore di occorrenze allo stato attuale (si sta confrontando per l'$i+1$-esima) e $j$ è l'indice del simbolo che viene confrontato. Verificata la coincidenza dell'ultimo simbolo, cioè certificata l'occorrenza $i+1$, si passa al confronto del blocco successivo incrementando il contatore $i$ e passando quindi alla serie $x_{i+1,j}$;
	\item se i due blocchi non coincidono, il primo simbolo diverso tra i due porta l'automa a prendere una transizione $\nok$, proseguendo poi per la serie di stati $\bar x_{i,j}$, che contano i simboli fino alla fine del blocco senza confrontare. Al termine di questa serie si riprende la computazione dallo stato $x_{i,j}$ alla ricerca della $i+1$-esima occorrenza nel blocco successivo.
\end{itemize}
Una volta trovate $n$ occorrenze, l'automa passa in una serie di stati $f_n,f_1,\dots,f_{n-1}$, comuni a tutti i blocchi candidati, che contano modulo $n$ i simboli rimanenti, accettando se sono in numero multiplo di $n$. Il numero di stati è quindi in totale $1+2^n\cdot (2n-1)\cdot n+n$.

\begin{figure}
	\centering
	\begin{tikzpicture}[shorten >=1pt,initial text=,near/.style={node distance=5mm}]
	\scriptsize
	\node[state,initial] (q0) {\small $q_0$};
	\node[state,node distance=12mm] (x00) [right=of q0] {$x_{0,0}$};
	\node[node distance=8mm] (dots1) [above=of x00] {$\dots$};
	\node (y1) [above=of dots1]{$x^{(1)}$};
	\node (dots2) [below=of x00] {$\dots$};
	\node (y2n) [below=of dots2] {$x^{(2^n)}$};

	\path[->]
	(q0) edge[bend left] node [above] {$\emptyword$} (y1.west)
	(q0) edge[bend left] node [above] {$\emptyword$} (dots1.west)
	(q0) edge[bend right] node [below] {$\emptyword$} (x00.west)
	(q0) edge[bend right] node [below] {$\emptyword$} (dots2.west)
	(q0) edge[bend right] node [below] {$\emptyword$} (y2n.west);

	\node[state,near] (x01) [right=of x00] {$x_{0,1}$};
	\node[state,near] (x02) [right=of x01] {$x_{0,2}$};
	\node[near] (x02d) [right=of x02] {$\dots$};
	\node[state,near] (x0n-1) [right=of x02d] {\tiny $x_{0,n-1}$};

	\path[->]
	(x00)	edge node [above] {$\ok$} (x01)
	(x01)	edge node [above] {$\ok$} (x02)
	(x02)	edge node [above] {$\ok$} (x02d)
	(x02d)	edge node [above] {$\ok$} (x0n-1);

	\node[state,node distance=8mm] (bx01) [below right=of x00] {$\bar x_{0,1}$};
	\node[state,near] (bx02) [right=of bx01] {$\bar x_{0,2}$};
	\node[near] (bx02d) [right=of bx02] {$\dots$};
	\node[state,near] (bx0n-1) [right=of bx02d] {\tiny $\bar x_{0,n-1}$};

	\path[->]
	(x00)	edge node [left] {$\nok$} (bx01)
	(bx01)	edge node [below] {$\any$} (bx02)
	(bx02)	edge node [below] {$\any$} (bx02d)
	(bx02d)	edge node [below] {$\any$} (bx0n-1)
	(x01)	edge node [left] {$\nok$} (bx02)
	(x02)	edge node [left] {$\nok$} (bx02d);

	\node[state,near] (x10) [right=of x0n-1] {$x_{1,0}$};

	\path[->]
	(bx0n-1) edge[bend angle=70,bend left] node[below] {$\any$} (x00)
	(x0n-1) edge[bend angle=45,bend right] node[below] {$\nok$} (x00)
	(x0n-1) edge node [above] {$\ok$} (x10);

	\node[near]		(x10d)		[right=of x10] {$\dots$};
	\node[state,near]	(in-1n-1)	[right=of x10d] {\tiny $x_{n-1,n-1}$};
	\node[state]		(f1)		[right=of in-1n-1.north] {$f_1$};
	\node[state,near,accepting]	(fn)		[above=of f1] {$f_n$};
	\node[state,near]		(f2)		[below=of f1] {$f_2$};
	\node[near]	(fd)	[below=of f2] {$\dots$};
	\node[state,near]	(fn-1)	[below=of fd] {\tiny $f_{n-1}$};

	\path[->]
	(in-1n-1)	edge node[left] {$\ok$} (fn)
	(fn)		edge node[right] {$\any$} (f1)
	(f1)		edge node[right] {$\any$} (f2)
	(f2)		edge node[right] {$\any$} (fd)
	(fd)		edge node[right] {$\any$} (fn-1)
	(fn-1)		edge[bend right] node[right] {$\any$} (fn);
\end{tikzpicture}

	\caption{L'NFA che riconosce $L_n$.}
	\label{img:wit:LnNFA}
\end{figure}

\subsubsection{2DFA}
Un 2DFA può riconoscere $L_n$ eseguendo innanzitutto una scansione preliminare che verifichi che la lunghezza dell'input sia multipla di $n$, quindi effettuando una scansione del nastro per ogni possibile blocco, contando le occorrenze di quello corrente (con una strategia simile a quella dell'1NFA descritto precedentemente), finché trovando $n$ occorrenze di un blocco può semplicemente accettare violando l'end-marker destro. Il numero di stati richiesti è $n$ per la prima fase, $2^n\cdot n\cdot (2n-1)$ per la seconda e $1$ per accettare, per un totale di $n+2^n\cdot (2n-1)\cdot n+1$ come nel caso precedente.

Questa macchina è anche un 2NFA e D\la1, di cui non conosciamo riconoscitori migliori che sfruttino le capacità in più dei rispettivi modelli per riconoscere $L_n$.



\section{Linguaggi unari}\label{sec:wit:un}
I linguaggi unari possiedono la fondamentale proprietà secondo cui le classi dei context-free e dei regolari collidono in un'unica classe. Da ciò deriva il fatto che un automa $d$-limited riconosce precisamente i linguaggi regolari, ossia i context-free, per qualunque $d$.

La relazione tra automi limited e linguaggi unari è stata studiata estensivamente da Pighizzini e Prigioniero in \cite{Pighizzini:19:limitedunary}. In particolare, i \la1 che riconoscono linguaggi unari possono fare uso di una tecnica basata sulla \eng{binary carry sequence}:
\begin{defin}
	La \emph{binary carry sequence} è la successione infinita di interi $\sigma_1\sigma_2\cdots\sigma_j\cdots$ in cui $\sigma_j$ è l'esponente della più alta potenza di $2$ che divide $j$, per ogni intero $j\geq1$.
\end{defin}

Definiamo inoltre la \eng{backward increasing sequence}, una funzione che trasforma sequenze che ha proprietà interessanti in relazione alla binary carry sequence.
\begin{defin}
	Sia $s=k_1k_2\cdots k_j$ una sequenza finita di interi. La \emph{backward increasing sequence} di $s$, denotata con $\bis(s)$, è la più lunga successione ottenibile selezionando da destra verso sinistra gli elementi di $s$ solo finché si susseguono in ordine crescente. Formalmente, $\bis(k_1k_2\cdots k_j)=(i_1,i_2,\dots,i_r)$ se e solo se $i_1=k_{h_1},i_2=k_{h_2}\dots i_r=k_{h_r}$ dove $h_1=j$ e $h_t=\max\set{h'<h_{t-1}\mid k_{h'}>k_{h_{t-1}}}$.
\end{defin}

Si verifica il seguente risultato, dimostrato in \cite{Pighizzini:19:limitedunary}:
\begin{lemma}\label{lem:wit:bis}
	Sia $\sigma_1\sigma_2\cdots\sigma_j$ il prefisso di lunghezza $j$ della binary carry sequence.
	\begin{itemize}
		\item \label{lem:wit:bis:1} Se $\bis(\sigma_1\sigma_2\cdots\sigma_j)=(i_1,i_2,\dots,i_r)$ allora
		      \begin{equation*}
			      j=\sum_{t=1}^r 2^{i_t}
		      \end{equation*}
		      Ossia, i valori della backward increasing sequence applicata al prefisso di lunghezza $j$ della binary carry sequence corrispondono alle posizioni dei bit a $1$ della rappresentazione binaria di $j$.
		\item \label{lem:wit:bis:2} $o_j$ è il minor numero naturale che non occorre in $\bis(\sigma_1\sigma_2\cdots\sigma_{j-1})$.
	\end{itemize}
\end{lemma}

Spieghiamo ora una tecnica che ci permette di riconoscere diversi linguaggi unari facendo uso del lemma per contare i simboli dell'input. Si prenda in considerazione il linguaggio singoletto $\set{a^{2^n}}$, dove $n>0$ è un parametro intero. Lo scopo della macchina è quello di scrivere il prefisso di lunghezza $2^n$ della binary carry sequence sul nastro, sostituendo le $a$.

Un D\la1 di alfabeto di lavoro $\set{a,0,1,\dots,n}$ può innanzitutto sovrascrivere il primo simbolo con $0$, primo elemento della binary carry sequence. Supponendo che a un certo punto della computazione la macchina abbia scritto il prefisso di lunghezza $j$ della binary carry sequence sovrascrivendo i primi $j$ simboli, il simbolo $\sigma_{j+1}$ può essere calcolato grazie al \hyperref[lem:wit:bis:2]{secondo punto} del lemma \ref{lem:wit:bis}: la macchina può effettuare visite in sola lettura verso sinistra, individuando il più piccolo naturale che non occorre nella backward increasing sequence del prefisso scritto. Tale numero è scritto nella successiva cella scrivibile e il procedimento viene ripetuto. Se a un certo punto della computazione l'automa scrive $n$ ($2^n$-esimo elemento della binary carry sequence) e la successiva cella contiene $\rem$ l'automa accetta. Se si raggiunge $\rem$ senza che $n$ venga scritto allora la parola è troppo corta, mentre se viene scritto ma la successiva cella non contiene l'end-marker la parola è troppo lunga. Questa macchina può essere implementata in $O(n)$ stati e fare uso di $O(n)$ simboli (dettagli sull'implementazione e l'algoritmo che essa utilizza sono presenti in \cite{Pighizzini:19:limitedunary}). Scrivendo al posto di $n$ un simbolo di reset $\reset$, in cui l'automa si comporta come sull'end-marker sinistro, si adatta la macchina ad accettare $\set{a^{2^n}}\star$ quando $\lem$ o $\reset$ precedono $\rem$. Un 1NFA necessità di $2^n$ stati per un contatore al fine di riconoscere questo linguaggio. Mereghetti e Pighizzini hanno dimostrato in \cite{Mereghetti:00:twoway} che lo stesso lower bound vale per i 2NFA. Questo linguaggio è quindi testimone della distanza almeno esponenziale da D\la1 a 2NFA (e quindi anche da \la1 e verso 1NFA, 1DFA e 2DFA) per i linguaggi unari.

Una variante di questa tecnica può essere applicata per riconoscere il linguaggio $M_N:=\set{a^N}\star$, con $N>0$ un intero qualsiasi. Per fare ciò, un D\la1 $_N$ può scrivere il prefisso di lunghezza $N-1$ della binary carry sequence con la tecnica descritta precedentemente, quindi scrivere un simbolo di reset $\reset$ che equivale a $\lem$ e impone alla macchina di ricominciare la scrittura dal simbolo $\sigma_1$. Questo procedimento viene ripetuto per tutto l'input: la stringa ha lunghezza multipla di $N$ se e solo se l'ultimo simbolo prima dell'end-marker destro è $\reset$ o $\lem$. Se $w$ è l'input il nastro verrà riscritto come segue:
\begin{equation*}
	\underbrace{\sigma_1\cdots\sigma_{N-1}\reset\cdots\reset\sigma_1\cdots\sigma_{N-1}\reset}_{\floor{\len w/N} \text{ volte}}\sigma_1\cdots\sigma_{\scriptscriptstyle \len w\mkern -11mu \mod N}
\end{equation*}
Poiché $N$ non è necessariamente una potenza di $2$, rilevare quando è il momento di scrivere $\reset$ non è triviale. Per fare ciò $B_N$, durante procedimento di identificazione del prossimo simbolo da scrivere, verifica inoltre se la backward increasing sequence del prefisso attuale rappresenta il numero $N-1$. Se così fosse, il \hyperref[lem:wit:bis:1]{primo punto} del lemma \ref{lem:wit:bis} dimostra che il prefisso attuale ha lunghezza $N-1$, e che quindi il prossimo simbolo da scrivere è $\reset$.

L'implementazione di $B_N$ richiede un numero di stati e un alfabeto di lavoro lineari nel massimo elemento della binary carry sequence che può essere scritto, ossia $O(\log N)$.



\section{Linguaggi con reset}
Con l'intenzione di studiare un witness language che non fosse né unario né a blocchi, Pighizzini, Prigioniero e Sádovský hanno studiato in \cite{Pighizzini:22:limitedwitness} il riconoscimento da parte di \la1 del linguaggio accettato dall'automa $S_N$, introdotto da Meyer e Fischer in \cite{Meyer:71:ecodescription} come testimone della distanza esponenziale tra 1NFA e 1DFA. L'automa, rappresentato in figura \ref{img:wit:Sn}, ha una struttura ciclica: se si considerano solo le transizioni generate dal simbolo $a$ l'automa riconosce il linguaggio $\set{a^N}\star$. Il ruolo di $b$ è scelto nondeterministicamente tra due: o viene ignorato, lasciando invariato lo stato, o impone alla macchina un reset, riportandola allo stato iniziale.

\begin{figure}
	\centering
	% \begin{tikzpicture}
\begin{tikzpicture}[shorten >=1pt,initial text=]
	\def\stateangle{30}
	\def\statedistance{2cm}
	\path[inner sep=0]
	(0,0) node[state](5){$q_{N-1}$}

	++(60:\statedistance) node[state,accepting,initial](0) {$q_0$}
	++(0:\statedistance) node[state](1){$q_1$}
	++(-60:\statedistance) node[state](2){$q_2$}

	(5)
	++(-60:\statedistance)  node[state](4){$q_{N-2}$}
	++(0:\statedistance)
	node[state](3){$q_3$};

	\path[->]
	(1) edge[loop above] node[near end,right] {$b$} (1)
	(2) edge[loop right] node[near end,below] {$b$} (2)
	(3) edge[loop below] node[near end,left] {$b$} (3)
	(4) edge[loop below] node[near end,left] {$b$} (4)
	(5) edge[loop left] node[near end,above] {$b$} (5)

	(0) edge[bend left] node[above] {$a$} (1)
	(1) edge[bend left] node[right] {$a$} (2)
	(2) edge[bend left] node[right] {$a$} (3)
	(3) edge[bend left,dashed]  (4)
	(4) edge[bend left] node[left] {$a$} (5)
	(5) edge[bend left] node[left] {$a$} (0)

	(1) edge[bend left] node[above] {$b$} (0)
	(2.170) edge[out=170,in=-45] node[above] {$b$} (0.-45)
	(3) edge[] node[above,near start] {$b$} (0)
	(4.70) edge[out=70,in=-75] node[left] {$b$} (0.-75)
	(5) edge[bend right] node[left] {$b$} (0)
	;
\end{tikzpicture}

	\caption{L'NFA $S_N$ di Meyer e Fischer.}
	\label{img:wit:Sn}
\end{figure}

Viene dimostrato in \cite{Meyer:71:ecodescription} che il minimo 1DFA equivalente a $S_N$ ha $2^N$ stati. In \cite{Pighizzini:22:limitedwitness} viene costruito un 2DFA di $N+2$ stati equivalente a $S_N$, il che testimonia la distanza esponenziale tra D\la1 e 1DFA, e viene dimostrato che il minimo 2NFA equivalente a $S_{2^n}$, con $n>0$, ha almeno $2^n$ stati.

Per simulare con \la1 automi con un meccanismo di reset simile a quello di $S_N$, si può sfruttare una variante della tecnica basata sulla binary carry sequence di cui al paragrafo \ref{sec:wit:un}. In particolare, per ogni $N>1$, $S_N$ può essere riconosciuto da un \la1 $C_N$ con $O(\log N)$ stati e un alfabeto di lavoro di $O(\log N)$ simboli.
Poiché, come accennato in precedenza, restringendo il comportamento di $S_N$ al solo input $a$ si riconosce il linguaggio unario $\set{a^N}\star$, in questo caso $C_N$ può comportarsi esattamente come $B_N$, l'automa che riconosce tale linguaggio. $C_N$ costruisce quindi una serie di ripetizioni del prefisso di lunghezza $N-1$ della binary carry sequence, separandole con il carattere di reset $\reset$. Per quanto riguarda il comportamento di $C_N$ per il simbolo di input $b$, la macchina può scegliere nondeterministicamente di effettuare una di due mosse, ciascuna corrispondente con uno dei comportamenti di $S_N$ leggendo $b$:
\begin{itemize}
	\item per simulare le transizioni che non cambiano stato, $C_N$ sovrascrive $b$ con un simbolo $\neutr$ neutrale, nel senso che il comportamento della macchina in esso sarà semplicemente di procedere senza tenerlo in considerazione;
	\item per simulare le transizioni di reset, $C_N$ sovrascrive $b$ con il simbolo di reset $\reset$. Si noti che sia in questo caso sia se il ciclo di $\set{a^N}\star$ viene completato il simbolo $\reset$ corrisponde agli istanti in cui $S_N$ passa allo stato $q_0$.
\end{itemize}
Se si incontra $b$ a destra di $\lem$ o di $\reset$, la mossa non è definita, così come non lo è in $q_0$ per $S_N$. $C_N$ accetta se e solo se $\reset$ o $\lem$ sono seguiti da $\rem$. L'incremento di stati e di simboli rispetto a $B_N$ è trascurabile ed entrambi rimangono in numero di $O(\log N)$.

In conclusione il linguaggio $\generated{S_N}$ è testimone del gap esponenziale tra \la1 e 1NFA e doppiamente esponenziale tra \la1 e 1DFA.
